%!TEX root = lec02_sql.tex

%
% ------------------------------------------------------------------------------
%

\begin{frame}{SQL Equivalence with the Relational Algebra}

Although very different in many respects, SQL and the algebra share many commonalities. In fact, we can write in the algebra many queries that are also expressible in SQL.

This is a good thing because there is an efficient (and easy to optimize) algorithm to compute answers to algebra expressions. 

Thus, we can answer a SQL query by: (1) translating it into an equivalent algebra expression; and (2) evaluating that expression efficiently.

\textbf{This is in fact what most DBMSs do!}

CMPUT391 is not a theory course, so we will not prove any equivalence results. Instead, only some hints are given.

\end{frame}


%
% ------------------------------------------------------------------------------
%

\newsavebox\algebraBasicSelectFromWhere
\savebox{\algebraBasicSelectFromWhere}{
    \begin{tikzpicture}[align=center,node distance=0.825cm,every node/.style={inner sep=0.5,outer sep=0.5,font=\footnotesize}]
    \node (P1) at (0,0) {$\pi_{a_1,\ldots,a_n}$};
    \node (P2) [below of=P1] {$\sigma_{c_1\ \wedge\ \cdots\ \wedge c_k}$};
    \node (P3) [below of=P2] {$\times$};
    \node (P4) [below left=0.25cm and 0.25cm of P3] {$\times$};
    \node (P5) [below right=0.25cm and 0.25cm of P3] {$T_m$};
    \node (P6) [below right=0.25cm and 0.25cm of P4] {$T_{m-1}$};
    \node (P7) [inner sep=4,below left=0.25cm and 0.125cm of P4] {$\cdots$};
    \node (P8) [below left=0.25cm and 0.125cm of P7] {$\times$};
    \node (P9) [below left=0.25cm and 0.125cm of P8] {$T_1$};
    \node (P10) [below right=0.25cm and 0.125cm of P8] {$T_2$};
    \path[commutative diagrams/.cd, every arrow, every label]
        (P2) edge node {} (P1)
        (P3) edge node {} (P2)
        (P4) edge node {} (P3)
        (P5) edge node {} (P3)
        (P6) edge node {} (P4)
        (P7) edge node {} (P4)
        (P8) edge node {} (P7)
        (P9) edge node {} (P8)
        (P10) edge node {} (P8);
\end{tikzpicture}%
}


\begin{frame}[fragile] %%%% equivalence of SQL and algebra
\label{AlgebraSQLEquivalencePartI}
\vskip1em
\begin{BOX}{Algebra and SQL equivalence -- take 1}
It should now be obvious that one can easily write an algebra expression equivalent to a SQL query where:\\
 - value expressions are attribute names\\
 - table expressions are base tables\\
 - predicates are comparisons involving attributes and/or constants
\end{BOX}

\vskip1em

\begin{columns}%[onlytextwidth]
\begin{column}{0.275\textwidth}
\begin{lstlisting}[style=cmput391,language=SQL,escapeinside={(*}{*)},frame=single]
SELECT (*$a_1, a_2, \ldots, a_n$*)
FROM (*$T_1, T_2, \ldots, T_m$*)
WHERE (*$c_1, c_2, \ldots, c_k$*)
\end{lstlisting}
\end{column}

\begin{column}{0.45\textwidth}
\scalebox{1.1}{\usebox\algebraBasicSelectFromWhere}
\end{column}
\end{columns}
\end{frame}


%
% EXAMPLE SQL QUERY FOR NEXT SLIDE
%
\newsavebox\SQLqueryBMM
\begin{lrbox}{\SQLqueryBMM}
\begin{minipage}{0.44\textwidth}
\begin{lstlisting}[style=SQL]
SELECT m.director
FROM Movie m, 
  (SELECT title, year 
   FROM Cast
   WHERE actor='Bill Murray'
  ) AS bmm
WHERE m.year=bmm.year 
  AND m.title=bmm.title;
\end{lstlisting}
\end{minipage}
\end{lrbox}

%
% ------------------------------------------------------------------------------
%

\begin{frame}[fragile]
\label{AlgebraSQLEquivalencePartII}
\vskip1em
\begin{BOX}{Algebra and SQL equivalence -- take 2}
 - Renaming in the algebra ($\rho$) can be used to handle SQL's \lstinline{AS} clauses

 \vskip0.5em

 - Table expressions that look like the ``simple SQL'' in slide~\ref{AlgebraSQLEquivalencePartI} can be expressed as ``sub-queries'' under the main $\times$ operator
\end{BOX}

\vskip1em

\begin{columns}%[onlytextwidth]
\begin{column}{0.275\textwidth}
\scalebox{0.75}{\fbox{\usebox\SQLqueryBMM}}
\end{column}
\begin{column}{0.6\textwidth}
\scalebox{1.1}{%
\begin{tikzpicture}[align=center,node distance=0.825cm,every node/.style={inner sep=0.5,outer sep=0.5,font=\footnotesize}]
    \node (P1) at (0,0) {$\pi_{\text{m.director}}$};
    \node (P2) [below of=P1] {$\sigma_{\substack{\text{m.year=bmm.year}\\\wedge\ \text{m.title=bmm.title}}}$};
    \node (P3) [below of=P2] {$\times$};
    \node (P4) [below right=0.25cm and 0.25cm of P3] {$\rho_{\substack{\text{bmm(bmm.title,}\\%
           \text{bmm.year)}}}$};
    \node (P5) [below left=0.25cm and 0.25cm of P3] {$\rho_{\substack{\text{m(m.title, m.year}\\
           \text{m.imdb,m.director)}}}$};
    \node (P6) [below=0.5cm of P4] {$\pi_{\text{title,year}}$};           
    \node (P7) [below=0.5cm of P6] {$\sigma_{\text{actor='Bill Murray'}}$};
    \node (P8) [below=0.5cm of P7] {Cast};
    \node (P9) [below=0.5cm of P5] {Movie};
    \path[commutative diagrams/.cd, every arrow, every label]
        (P2) edge node {} (P1)
        (P3) edge node {} (P2)
        (P4) edge node {} (P3)
        (P5) edge node {} (P3)
        (P6) edge node {} (P4)
        (P7) edge node {} (P6)
        (P8) edge node {} (P7)
        (P9) edge node {} (P5);
\end{tikzpicture}%
}
\end{column}
\end{columns}
\end{frame}

%
% ------------------------------------------------------------------------------
%

\begin{frame}[fragile]

\begin{BOX}{What about aggregation?}

Most textbooks give an ``extended algebra'' with an operator \[\gamma_{\text{<grouping>,fn()}}(R)\]

The \lstinline{GROUP BY} clause goes into ``grouping'' and fn() is the set function. The \lstinline{HAVING} clause are handled with a selection.
\end{BOX}

\vskip1em

\begin{columns}[onlytextwidth]
\begin{column}{0.3\textwidth}
\begin{lstlisting}[style=SQL]
SELECT theater, COUNT(*) 
FROM Guide
WHERE start > 1100
GROUP BY theater
HAVING COUNT(*) > 4
\end{lstlisting}
\end{column}
\begin{column}{0.4\textwidth}
\scalebox{1}{
     \begin{tikzpicture}[align=center,node distance=0.825cm,every node/.style={inner sep=0.5,outer sep=0.5,font=\footnotesize}]
    \node (P1) at (0,0) {$\pi_{\text{theater},\text{COUNT(*)}}$};
    \node (P2) [below of=P1] {$\sigma_{\text{COUNT(*)}>4}$};
    \node (P3) [below of=P2] {$\gamma_{\text{theater},\text{COUNT(*)}}$};
    \node (P4) [below of=P3] {$\sigma_{\text{start}>1100}$};
    \node (P5) [below of=P4] {\lstinline{Guide}};
    \path[commutative diagrams/.cd, every arrow, every label]
        (P2) edge node {} (P1)
        (P3) edge node {} (P2)
        (P4) edge node {} (P3)
        (P5) edge node {} (P4);
\end{tikzpicture}%
}
\end{column}
\end{columns}
\end{frame}


\begin{frame}

\begin{BOX}{What about recursion?}
Recursion is normally left out of the algebra.
\end{BOX}

\vskip3em

\begin{BOX}{What about universal quantification?}
Many textbooks add a \textbf{division} operator to the algebra, specifically to compute ``for all'' queries. However, most relational engines do not use that operator in query plans.
\end{BOX}

\end{frame}