\documentclass[xcolor={usenames,dvipsnames}]{beamer}

%
% ugly hack referring to the slides folder. Works for me.
%
\makeatletter
\def\input@path{{/Users/denilson/teaching/cmput391/git-slides/}}
\makeatother

% \usepackage[T1]{fontenc} % font encoding
% \usepackage{fontspec}
\usepackage{fontspec}


\makeatletter%
\@ifclassloaded{beamer}{%
  \usepackage{lmodern} %%% modern beamer style
  %%beamer style
  \usetheme{metropolis}
  \usefonttheme{professionalfonts}
  \setbeamercolor{background canvas}{bg=background}
  \setbeamercolor{frametitle}{bg=snow,fg=black}
  \setbeamercolor{title separator}{fg=snow}
  \setbeamercolor{alerted text}{fg=accent}

  \let\oldtitle\title
  \renewcommand{\title}[1]{\oldtitle{CMPUT391\\#1}}
  \author{Instructor: Denilson Barbosa}
  \date{\small University of Alberta}
  \institute{\today\vskip4em Slides by D. Barbosa, with suggested corrections and improvements by (in alphabetical order) C. Bins, D. Caminhas, K. Guhzva, Q. Lautischer, E. Macdonald, M. A. Nascimento, K. Newbury, M. Strobl, D. Sunderman, K. Wang, and K. Wong.}
  
  \titlegraphic{\includegraphics[width=1.5cm]{../images/by-sa.png}~
  {\tiny This work is licensed under a Creative Commons Attribution 4.0 International License}}

  %%%% un-framed blocks
  \setbeamertemplate{blocks}[rounded][shadow=false]
}{}

\@ifpackageloaded{xcolor}{}{
  \usepackage{xcolor}
}
\makeatother


\setsansfont[BoldFont={Open Sans SemiBold}]{Open Sans}[Scale=0.85]
\setmonofont{Cousine}[Scale=0.8]
\setmathtt{Cousine}[Scale=0.8]

%%%% highlighting text 
\usepackage{xspace}
\def\highlight#1{\colorbox{accent}{\textcolor{white}{#1}}\xspace}

\def\blue#1{\textcolor{blue}{#1}}

%
%
%
\usepackage{parskip}

\usepackage{subcaption} % needed for subfigures

\usepackage{wrapfig} % for figures "to the side of the text"

\usepackage{anyfontsize} % for egregious warnings

%
% Fonts
%
% \usepackage{amsmath}
% \usepackage{mathspec}
% \setmathtt[Scale=0.875]{Cousine}

%
% various packages for tables and colored tables
%
\usepackage{xcolor,colortbl}
\usepackage{tcolorbox}

\definecolor{accent}{RGB}{225,68,85}
\definecolor{highlight}{RGB}{170,68,153}
\definecolor{background}{rgb}{253,252,252}

\definecolor{sqlColor}{RGB}{119,119,17}

\definecolor{Cfunction}{RGB}{51,51,153}

\definecolor{stringColor}{RGB}{51,51,153}
\definecolor{datatypeColor}{RGB}{170,68,153}
\definecolor{functionColor}{RGB}{51,134,116}

\definecolor{exampleColor}{RGB}{51,34,136}
\definecolor{commentColor}{RGB}{112,68,225}

\definecolor{fern}{RGB}{80,118,66}

\definecolor{snow}{RGB}{240,240,230}
\definecolor{Gray}{gray}{0.85}

\definecolor{tupleBoxColor}{gray}{0.85}

\definecolor{Maroon}{RGB}{139,0,0}

%
% coloring of SQL code
%
\def\ColorSymbol#1{\textcolor{functionColor}{#1}}

%
% hihglight box
%
\usepackage{xspace}
\def\highlight#1{\colorbox{accent}{\textcolor{white}{#1}}\xspace}

%
% Tables
%
\usepackage{multirow}
\usepackage{multicol}

%
% CODE LISTINGS
%
\usepackage{listings}

%
% catch-all style meant to provide a clean and "empty" style
% from which all others build upon
%
\lstdefinestyle{cmput391}{
  style=,
  frame=no,
  tabsize=2,
  numbers=none,
  showstringspaces=false,
  numberstyle=\footnotesize,
  basicstyle=\ttfamily\footnotesize\color{black},
  numbersep=4pt,
  keywords=[1]{},
  keywords=[2]{},
  keywords=[3]{sh_lock,xl_lock,unlock,read,write,commit,abort},
  keywordstyle=[1]\ttfamily\bfseries\footnotesize,
  keywordstyle=[2]\ttfamily\footnotesize,
  keywordstyle=[3]\ttfamily\footnotesize\color{Maroon},
  emph=[1]{},
  emph=[2]{},
  emph=[3]{},
  emphstyle=[1]\ttfamily\footnotesize,
  emphstyle=[2]\ttfamily\footnotesize,
  emphstyle=[3]\ttfamily\footnotesize,
  commentstyle=\ttfamily\itshape\color{commentColor},
  stringstyle=\ttfamily\color{stringColor},
  moredelim=**[l][\itshape\color{commentColor}]{--},
  moredelim=**[is][\color{highlight}]{-|}{|-},
  moredelim=[is][\bfseries\color{highlight}]{-:}{:-},
  morestring=[s][\color{stringColor}]{"}{"},
  morecomment=[l]{--}
}

\lstdefinelanguage{DTD}{
  style=cmput391,
  morekeywords=[1]{DOCTYPE,ELEMENT,ATTLIST},
  keywordstyle=[1]\color{accent},
  literate = {\#PCDATA}{{\textcolor{Cfunction}{\#PCDATA}}}7
             {\#IMPLIED}{{\textcolor{Cfunction}{\#IMPLIED}}}8
}

\lstdefinelanguage{XPath}{
  style=cmput391,
  morekeywords={xs,integer,current,date,distinct,values,avg,text,id,count,position,and,or,not,boolean,number,sum,floor,ceiling,round,doc,eq,ne,gt,lt,ge,le},
  keywordstyle=\color{red},
  literate = {last(}{{\textcolor{functionColor}{last}(}}5 {@<}{{<}}1  
}

\lstdefinestyle{XQuery}{
  style=cmput391,
  keywords=[1]{if,then,else,and,or,some,every,satisfies,for,in,let,where,group,order,by,return,ascending},
  keywords=[2]{gt,lt,eq,ne,ge,le,not,or},
  keywords=[3]{text,sum,min,max,avg,len,position,number,boolean,floor,ceiling,round,%
               id,doc,count,current-date,distinct-values,xs:integer,subsequence},
  keywordstyle=[1]\bfseries\color{sqlColor},
  keywordstyle=[2]\color{accent},
  keywordstyle=[3]\color{functionColor},
  moredelim =[is][\color{blue}]{-:}{:-},
  moredelim =[s][\color{accent}]{<}{>},
  moredelim=**[s][\color{black}]{/}{\ },
  moredelim=*[s][\color{blue}]{\$}{\ },
  alsoletter=:{:,-},
  literate = {(}{{\textcolor{black}{(}}}{1} {)}{{\textcolor{black}{)}}}{1} 
             {\{}{{\textcolor{black}{\{}}}{1} {\}}{{\textcolor{black}{\}}}}{1} 
             {last(}{{\textcolor{functionColor}{last}(}}5 {@<}{{<}}1
}


\lstdefinestyle{SQL}{
  style=cmput391,
  language=SQL,
  sensitive={True},
  deletekeywords={year,Cast,MIN,MAX},
  morekeywords={ABORT,AFTER,BEFORE,REFERENCES,REFERENCING,WITH,COMMITTED,UNCOMMITTED,REPEATABLE,SERIALIZABLE},
  keywords=[2]{AVG,MIN,MAX,COUNT,SUM,NEW,OLD,strftime,date,length,RAISE,rowid,substr},
  keywords=[3]{BIGINT,CHAR,DATE,DECIMAL,INT,FLOAT,TEXT},
  keywordstyle=[1]\ttfamily\bfseries\color{sqlColor},
  keywordstyle=[2]\ttfamily\bfseries\color{functionColor},
  keywordstyle=[3]\ttfamily\bfseries\color{datatypeColor},
  moredelim=**[is][\ttfamily\bfseries\color{functionColor}]{(@}{@)},
  morestring=[s][\color{stringColor}]{'}{'},
  moredelim=[is][\bfseries\color{highlight}]{-:}{:-}
}


%
% lstlisting for command line examples
%
\lstdefinestyle{commandLine}{
  basicstyle=cmput391,
  keywordstyle=[1]\ttfamily\bfseries\color{functionColor},
  keywordstyle=[2]\ttfamily\bfseries\color{datatypeColor},
  moredelim=**[is][\color{accent}\bf]{@}{@}
  keywords=[1]{%
    grep},
  keywords=[2]{%
    postings}
}

%
% lstlisting for C code
%
\lstdefinestyle{C}{
  language=C,
  tabsize=2,
  basicstyle=\ttfamily\footnotesize,
  stringstyle=\color{stringColor},
  keywords=[2]{sqlite3_prepare_v2,sqlite3_bind_int64,sqlite3_step,sqlite3_exec},
  keywordstyle=[1]\ttfamily\bfseries\color{datatypeColor},
  keywordstyle=[2]\ttfamily\bfseries\color{functionColor},
  commentstyle=\ttfamily\itshape\color{commentColor},
  moredelim=**[is][\color{Cfunction}\bf\ttfamily]{@}{@}
}


\lstdefinestyle{Python}{
  language=Python,
  tabsize=2,
  basicstyle=\ttfamily\footnotesize,
  stringstyle=\color{stringColor},
  morekeywords=[1]{self,yield,Class,true,false,None},
  keywords=[2]{satisfies,from,Iterator,EOF,concatenate,advance,join,rewind},
  deletekeywords=[1]{from},
  keywordstyle=[1]\ttfamily\bfseries\color{datatypeColor},
  keywordstyle=[2]\ttfamily\bfseries\color{functionColor},
  commentstyle=\ttfamily\itshape\color{commentColor},
  moredelim=**[is][\color{Cfunction}\bf\ttfamily]{@}{@}
}



%
% lstlisting style for RDF/SPARQL
%
\lstdefinestyle{SPARQL}{
  style=cmput391,
  sensitive={True},
  keywords=[1]{BASE,PREFIX,SELECT,WHERE,FILTER,MINUS,NOT,EXISTS},
  keywordstyle=[1]\ttfamily\bfseries\color{sqlColor},
  moredelim=*[s][\color{Cfunction}]{?}{\ },
  moredelim=*[s][\color{Cfunction}]{@}{\ },
  morestring=[s][\color{datatypeColor}]{"}{"},
  literate =* {|}{{|}}1 {//}{{//}}2 {< }{< }{2}
}

%
% coloring of markup code
%
\lstdefinestyle{markup}{
  style=cmput391,
  literate =* {=}{{\color{sqlColor}{=}}}1,
  moredelim =* [s][\color{accent}]{<}{>},
  moredelim =** [is][\color{accent}]{@}{@},
  moredelim =** [is][\color{blue}]{-:}{:-},
  morecomment=[s]{<!--}{-->}
}

\lstdefinestyle{RDF}{
  style=cmput391,
  keywords=[1]{BASE,PREFIX},
  keywordstyle=[1]\ttfamily\bfseries\color{functionColor},
  alsoletter=:{:,-,@},
  literate = {prefix}{{\bfseries\textcolor{functionColor}{prefix}}}6 
             {:}{{\textcolor{functionColor}{:}}}1 
             {@}{{\textcolor{functionColor}{@}}}1,
  moredelim =* [s][\color{accent}]{<}{>},
  moredelim =** [is][\color{functionColor}]{-:}{:-},
  morecomment=[s]{<!--}{-->}
}




















%
% Tikz stuff, for cummulative diagrams and algebra trees
%
\usepackage{tikz} %commutative diagrams
\usepackage{tikz-cd} %commutative diagrams
\usetikzlibrary{positioning}% To get more advances positioning options
\usetikzlibrary{arrows,shapes,automata}
\usetikzlibrary{calc} % for calculations and hard-coded coordinates
\usetikzlibrary{tikzmark, fit, shapes.misc}
\usetikzlibrary{decorations.pathreplacing}
\usetikzlibrary{backgrounds} % to draw underneath other elements
\usepackage{stackengine} % to overlay tikz pictures over other things

\usepackage{pgfplots}
\pgfplotsset{compat=1.17}

%
%
% PGF arrowtip that looks like an X
%
%
\pgfarrowsdeclare{X}{X}
{
  \arrowsize=0.25pt
  \advance\arrowsize by .5\pgflinewidth
  \pgfarrowsleftextend{-4\arrowsize-.5\pgflinewidth}
  \pgfarrowsrightextend{.5\pgflinewidth}
}
{
  \arrowsize=0.2pt
  \advance\arrowsize by .5\pgflinewidth
  \pgfsetdash{}{0pt} % do not dash
  \pgfsetroundjoin   % fix join
  \pgfsetroundcap    % fix cap
  \pgfpathmoveto{\pgfpointorigin}
  \pgfpathlineto{\pgfpoint{-5\arrowsize}{5\arrowsize}}
  \pgfusepathqstroke
  \pgfpathmoveto{\pgfpointorigin}
  \pgfpathlineto{\pgfpoint{5\arrowsize}{-5\arrowsize}}
  \pgfusepathqstroke
  \pgfpathmoveto{\pgfpointorigin}
  \pgfpathlineto{\pgfpoint{-5\arrowsize}{-5\arrowsize}}
  \pgfusepathqstroke
  \pgfpathmoveto{\pgfpointorigin}
  \pgfpathlineto{\pgfpoint{5\arrowsize}{5\arrowsize}}
  \pgfusepathqstroke
}


%
%%% for algorithms and pseudocode
%
\usepackage{algorithm}
% \usepackage{algpseudocode}
\usepackage[noend]{algpseudocode}
\renewcommand{\algorithmicrequire}{\textbf{Input:}}
\renewcommand{\algorithmicensure}{\textbf{Output:}}
\algnewcommand\algorithmicforeach{\textbf{for each}}
\algdef{S}[FOR]{ForEach}[1]{\algorithmicforeach\ #1\ \algorithmicdo}
% style for comments inside algorithms
\renewcommand{\algorithmiccomment}[1]{\hspace*{1em}\textcolor{commentColor}{\%~\textit{#1}}} 

%
% to create boxes with verbatim and code-like content
%
\usepackage{verbatimbox}

%
% to crop/trim boxes!
%
\usepackage{trimclip}

%
% colored and framed boxes
%
\newenvironment{BOX}[1]{\begin{tcolorbox}[colback=snow!25,colframe=snow!40!black,title=#1]\small}{\end{tcolorbox}}



%
% math and symbols
%
\usepackage{mathtools}
\DeclarePairedDelimiter{\ceil}{\lceil}{\rceil}
\DeclarePairedDelimiter{\floor}{\lfloor}{\rfloor}
\usepackage{amsbsy} 
\usepackage{amssymb}
\usepackage{ifsym} % needed to produce the common outerjoin symbols

%
% for nice-looking enumerations and inlined enumerations
%
\usepackage[shortlabels,inline]{enumitem}



%%% outer join symbols
\newcommand\TextbookOuterJoin{%
  \mathrel{\ooalign{\hss$\Join$\hss\cr%
  \kern0.5ex\raise1ex\hbox{\scalebox{0.7}{$\circ$}}}}}

\newcommand\RightOuterJoin{%
  \mathrel{%
  	\ooalign{%
  		$\Join$\cr\kern1.25ex\raise0.0975ex\hbox{\scalebox{0.25}[0.8275]{$\sqsubset$}}}
  	}
}

\newcommand\LeftOuterJoin{%
  \mathrel{%
  	\ooalign{%
  		$\Join$\cr\kern-0.0125ex\raise0.0975ex\hbox{\scalebox{0.25}[0.8275]{$\sqsupset$}}}%
	}
}

\newcommand\FullOuterJoin{%
  \mathrel{%
  	\ooalign{%
  		$\Join$\cr\kern-0.0125ex\raise0.0975ex\hbox{\scalebox{0.25}[0.8275]{$\sqsupset$}}}%
  		\kern-0.45ex\raise0.0975ex\hbox{\scalebox{0.25}[0.8275]{$\sqsubset$}}
	}
}






\title{Brief review of the Relational Algebra}

\begin{document}


% full relational schema

\newsavebox\FullMovieSchema
\begin{lrbox}{\FullMovieSchema}\begin{minipage}{\textwidth}
\begin{lstlisting}[style=SQL]
CREATE TABLE Movie (title CHAR(20), year INT, imdb FLOAT,
   PRIMARY KEY (title, year), CHECK(imdb >= 0 AND imdb <= 10));
CREATE TABLE Director (name CHAR(20), PRIMARY KEY (name));
CREATE TABLE Actor (name CHAR(20), PRIMARY KEY (name));
CREATE TABLE Cinema (name CHAR(20), address CHAR(100), 
   PRIMARY KEY(name));

CREATE TABLE MovieDirector (title CHAR(20), year INT, name CHAR(20)
  PRIMARY KEY (title, year, name),
  FOREIGN KEY (title, year) REFERENCES Movie(title, year),
  FOREIGN KEY (name) REFERENCES Director(name));

CREATE TABLE Cast (title CHAR(20), year INT, name CHAR(20), role CHAR(20), 
  PRIMARY KEY (title, year, name, role),
  FOREIGN KEY (title, year) REFERENCES Movie(title, year)
  FOREIGN KEY (name) REFERENCES Actor(name));

CREATE TABLE Guide (theater CHAR(20), film CHAR(20), year INT, start INT, 
  PRIMARY KEY (theater, film, year, start),
  FOREIGN KEY (theater) REFERENCES Cinema(name), 
  FOREIGN KEY (film, year) REFERENCES Movie(title, year));
\end{lstlisting}
\end{minipage}
\end{lrbox}


% simplified relational schema

\newsavebox\SimplifiedMovieSchema
\begin{lrbox}{\SimplifiedMovieSchema}\begin{minipage}{\textwidth}
\begin{lstlisting}[style=SQL]
CREATE TABLE Movie (title CHAR(20), year INT, imdb FLOAT, 
   director CHAR(20) NOT NULL, PRIMARY KEY (title, year), 
   CHECK(imdb >= 0 AND imdb <= 10));

CREATE TABLE Cinema (name CHAR(20), address CHAR(100), 
   PRIMARY KEY(name));

CREATE TABLE Cast (title CHAR(20), year INT, 
   actor CHAR(20) NOT NULL, role CHAR(20) NOT NULL, 
   FOREIGN KEY (title, year) REFERENCES Movie(title, year));

CREATE TABLE Guide (theater CHAR(20), film CHAR(20), year INT, start INT, 
  PRIMARY KEY (theater, film, year, start),
  FOREIGN KEY (theater) REFERENCES Cinema(name), 
  FOREIGN KEY (film, year) REFERENCES Movie(title, year));
\end{lstlisting}
\end{minipage}
\end{lrbox}

% simplified CREATE MOVIE example

\newsavebox\SimplifiedMovieTableDDL
\begin{lrbox}{\SimplifiedMovieTableDDL}\begin{minipage}{0.9\textwidth}
\begin{lstlisting}[style=SQL]
CREATE TABLE Movie (title CHAR(20), year INT, imdb FLOAT, 
   director CHAR(20), PRIMARY KEY (title, year), 
   CHECK(imdb >= 0 AND imdb <= 10));
\end{lstlisting}
\end{minipage}
\end{lrbox}

% simplified CREATE CAST example

\newsavebox\SimplifiedCastTableDDL
\begin{lrbox}{\SimplifiedCastTableDDL}\begin{minipage}{\textwidth}
\begin{lstlisting}[style=SQL]
CREATE TABLE Cast (title CHAR(20), year INT, 
   actor CHAR(20), role CHAR(20), 
   FOREIGN KEY (title, year) REFERENCES Movie(title, year);
\end{lstlisting}
\end{minipage}
\end{lrbox}
%
% this file has latex "boxes" with each of the tables
% and query results for the examples involving the
% movies database
%

\newsavebox{\MovieTable}
\savebox{\MovieTable}{%
\scriptsize%
\begin{tabular}{l | l | c | l }
\multicolumn{4}{l}{\textbf{Movie}}\\
\hline
\rowcolor{Gray}
\textbf{title} & \textbf{year} & \textbf{imdb} & \textbf{director} \\
\hline
Ghostbusters & 1984 & 7.8 & Ivan Reitman \\
\hline
Big & 1988 & 7.3 & Penny Marshall \\
\hline
Lost in Translation & 2003 & 7.8 & Sofia Coppola \\
\hline
Wadjda & 2012 & 8.1 & Haifaa al-Mansour \\
\hline
Ghostbusters & 2016 & 5.3 & Paul Feig \\
\hline
\end{tabular}
}

\newsavebox{\MovieTableTWO}
\savebox{\MovieTableTWO}{%
\scriptsize%
\begin{tabular}{l | l | c | l }
\multicolumn{4}{l}{\textbf{Movie}}\\
\hline
\rowcolor{Gray}
\textbf{title} & \textbf{year} & \textbf{imdb} & \textbf{director} \\
\hline
Lost in Translation & 2003 & 7.8 & Haifaa al-Mansour \\ % fake tuple :)
\hline
Groundhog day & 1993 & 8 & Harold Ramis \\
\hline
\end{tabular}
}


\newsavebox{\CinemaTable}
\savebox{\CinemaTable}{%
\scriptsize%
\begin{tabular}{l | l }
\multicolumn{2}{l}{\textbf{Cinema}}\\
\hline
\rowcolor{Gray}
\textbf{name} & \textbf{address}  \\
\hline
Garneau & 8712 109 St, Edmonton \\
\hline
Princess & 10337 82 Ave, Edmonton \\
\hline
Landmark & 10200 102 Ave, Edmonton\\
\hline
\end{tabular}
}

\newsavebox{\GuideTable}
\savebox{\GuideTable}{%
\scriptsize%
\begin{tabular}{l | l | l | l }
\multicolumn{4}{l}{\textbf{Guide}}\\
\hline
\rowcolor{Gray}
\textbf{theater} & \textbf{film} & \textbf{year} & \textbf{start}  \\
\hline
Garneau & Ghostbusters & 1984 & 1140 \\
\hline
Garneau & Ghostbusters & 2016 & 1290 \\
\hline
Princess & Wadjda & 2012 & 1260 \\
\hline
\end{tabular}
}


\newsavebox{\CastTable}
\savebox{\CastTable}{%
\scriptsize%
\begin{tabular}{l | l | l | l }
\multicolumn{4}{l}{\textbf{Cast}}\\
\hline
\rowcolor{Gray}
\textbf{title} & \textbf{year} & \textbf{actor} & \textbf{role}  \\
\hline
Ghostbusters & 1984 & Bill Murray & Dr. Venkman \\
\hline
Ghostbusters & 1984 & Sigourney Weaver & Dana Barret \\
\hline
Big & 1988 & Tom Hanks & Josh \\
\hline
Big & 1988 & Elisabeth Perkins & Susan \\
\hline
Lost in Translation & 2003 & Bill Murray & Bob Harris \\
\hline
Wadjda & 2012 & Waad Mohammed & Wadjda \\
\hline
Ghostbusters & 2016 & Dan Aykroyd & Cabbie \\
\hline
Ghostbusters & 2016 & Sigourney Weaver & Dana Barret \\
\hline
\end{tabular}
}

%
% box with all tables together.
%
%%%% NEED TO FIND A NICER WAY THAN THIS...
\newsavebox{\FullInstance}
\savebox{\FullInstance}{%
	\begin{tabular}{@{}l@{}}
		\resizebox{0.6\textwidth}{!}{\usebox{\MovieTable}}\\
		~\\[-0.5em]
		\resizebox{0.7\textwidth}{!}{\usebox{\CastTable}}\\
		~\\[-0.5em]
		\resizebox{0.95\textwidth}{!}{\usebox{\CinemaTable}~~~\usebox{\GuideTable}
		}%
	\end{tabular}%
}

%
% example movie column-oriented store
%
\newsavebox{\MovieTableColTitle}
\savebox{\MovieTableColTitle}{%
\scriptsize%
\begin{tabular}{l | l}
\hline
\rowcolor{Gray}
\ColorSymbol{\textbf{id}} &\textbf{title} \\
\hline
\ColorSymbol{1} & Ghostbusters \\
\hline
\ColorSymbol{2} & Big \\
\hline
\ColorSymbol{3} & Lost in Translation \\
\hline
\ColorSymbol{4} & Wadjda \\
\hline
\ColorSymbol{5} & Ghostbusters \\
\hline
\end{tabular}
}

\newsavebox{\MovieTableColYear}
\savebox{\MovieTableColYear}{%
\scriptsize%
\begin{tabular}{l | l }
\hline
\rowcolor{Gray}
\ColorSymbol{\textbf{id}} &\textbf{year} \\
\hline
\ColorSymbol{1} & 1984 \\
\hline
\ColorSymbol{2} & 1988 \\
\hline
\ColorSymbol{3} & 2003 \\
\hline
\ColorSymbol{4} & 2012 \\
\hline
\ColorSymbol{5} & 2016 \\
\hline
\end{tabular}
}

\newsavebox{\MovieTableColIMDB}
\savebox{\MovieTableColIMDB}{%
\scriptsize%
\begin{tabular}{l | l}
\hline
\rowcolor{Gray}
\ColorSymbol{\textbf{id}} &\textbf{imdb} \\
\hline
\ColorSymbol{1} & 7.8 \\
\hline
\ColorSymbol{2} & 7.3 \\
\hline
\ColorSymbol{3} & 7.8 \\
\hline
\ColorSymbol{4} & 8.1 \\
\hline
\ColorSymbol{5} & 5.3 \\
\hline
\end{tabular}
}

\newsavebox{\MovieTableColDirector}
\savebox{\MovieTableColDirector}{%
\scriptsize%
\begin{tabular}{l | l}
\hline
\rowcolor{Gray}
\ColorSymbol{\textbf{id}} &\textbf{director} \\
\hline
\ColorSymbol{1} & Ivan Reitman \\
\hline
\ColorSymbol{2} & Penny Marshall \\
\hline
\ColorSymbol{3} & Sofia Coppola \\
\hline
\ColorSymbol{4} & Haifaa al-Mansour  \\
\hline
\ColorSymbol{5} & Paul Feig \\
\hline
\end{tabular}
}


%
% Table with selection on Cast.actor='Bill Murray'
%
\newsavebox{\BillMurrayMovies}
\savebox{\BillMurrayMovies}{%
\scriptsize%
\begin{tabular}{l | l | l | l }
\rowcolor{Gray}
\hline
\textbf{title} & \textbf{year} & \textbf{actor} & \textbf{role}  \\
\hline
Ghostbusters & 1984 & Bill Murray & Dr. Venkman \\
\hline
Lost in Translation & 2003 & Bill Murray & Bob Harris \\
\hline
\end{tabular}
}

%
% Table with projection of director on Movie
%
\newsavebox{\MovieDirectors}
\savebox{\MovieDirectors}{%
\scriptsize%
\begin{tabular}{l }
\rowcolor{Gray}
\hline
\textbf{director}  \\
\hline
Ivan Reitman \\
\hline
Penny Marshall \\
\hline
Sofia Coppola \\
\hline
Haifaa al-Mansour  \\
\hline
Paul Feig \\
\hline
\end{tabular}
}

%
% Table with Movie \bowtie \sigma Cast.actor='Bill Murray'
%
\newsavebox{\JoinMoviesBillMurrayMovies}
\savebox{\JoinMoviesBillMurrayMovies}{%
\scriptsize%
\begin{tabular}{l | l | c | l | l | l }
\rowcolor{Gray}
\hline
\textbf{title} & \textbf{year} & \textbf{imdb} & \textbf{director} & \textbf{actor} & \textbf{role}  \\
\hline
Ghostbusters & 1984 & 7.8 & Ivan Reitman & Bill Murray & Dr. Venkman \\
\hline
Lost in Translation & 2003 & 7.8 & Sofia Coppola & Bill Murray & Bob Harris \\
\hline
\end{tabular}
}

%
% Left Outer Join of Cinema and Guide
%
\newsavebox{\LeftOuterJoinCinemaGuide}
\savebox{\LeftOuterJoinCinemaGuide}{%
\scriptsize%
\begin{tabular}{l | l | l | l | l | l}
\hline
\rowcolor{Gray}
\textbf{name} & \textbf{address} & \textbf{theater} & \textbf{film} & \textbf{year} & \textbf{start} \\
\hline
Garneau & 8712 109 St, Edmonton & Garneau & Ghostbusters & 1984 & 1140 \\
\hline
Garneau & 8712 109 St, Edmonton & Garneau & Ghostbusters & 2016 & 1290 \\
\hline
Princess & 10337 82 Ave, Edmonton & Princess & Wadjda & 2012 & 1260 \\
\hline
Landmark & 10200 102 Ave, Edmonton & NULL & NULL & NULL & NULL\\
\hline
\end{tabular}
}

%
% Semijoin of Cinema and Guide
%
\newsavebox{\SemijoinCinemaGuide}
\savebox{\SemijoinCinemaGuide}{%
\scriptsize%
\begin{tabular}{l | l }
\hline
\rowcolor{Gray}
\textbf{name} & \textbf{address}  \\
\hline
Garneau & 8712 109 St, Edmonton \\
\hline
Princess & 10337 82 Ave, Edmonton \\
\hline
\end{tabular}
}

%
% Single tuple in self-join of Cast to get actors in reruns
%
\newsavebox{\SelfJoinExampleSigourneyWeaver}
\savebox{\SelfJoinExampleSigourneyWeaver}{%
\scriptsize%
\begin{tabular}{l | l | l | l | l | l | l | l}
\hline
\rowcolor{Gray}
\textbf{title} & \textbf{year} & \textbf{actor} & \textbf{role}  & \textbf{title2} & \textbf{year2} & \textbf{actor2} & \textbf{role2} \\
\hline
Ghostbusters & 1984 & Sigourney Weaver & Dana Barret & Ghostbusters & 2016 & Sigourney Weaver & Dana Barret \\
\hline
\end{tabular}
}

%
% result of select theater, count(*) on Guide
%
\newsavebox{\CountStartOnGuideTable}
\savebox{\CountStartOnGuideTable}{%
\scriptsize%
\begin{tabular}{l | l }
\hline
\rowcolor{Gray}
\textbf{theater} & \textbf{COUNT} \\
\hline
Garneau & 2\\
\hline
Princess & 1\\
\hline
\end{tabular}
}

%
% This file contains latex boxes with generic tables R, S, ...
% used in algebra and SQL examples
%

%
% Tables R, S for cross product example
%
\newsavebox{\CrossProdExR}
\savebox{\CrossProdExR}{%
\scriptsize%
\begin{tabular}{l | l}
\multicolumn{2}{l}{\textbf{R}}\\
\rowcolor{Gray}
\hline
$\boldsymbol{r_1}$ & $\boldsymbol{r_2}$ \\
\hline
$x_1$ & $x_2$ \\
\hline
$x_3$ & $x_4$ \\
\hline
\end{tabular}
}
\newsavebox{\CrossProdExS}
\savebox{\CrossProdExS}{%
\scriptsize%
\begin{tabular}{l | l | l}
\multicolumn{2}{l}{\textbf{S}}\\
\rowcolor{Gray}
\hline
$\boldsymbol{s_1}$ & $\boldsymbol{s_2}$ & $\boldsymbol{s_3}$\\
\hline
$y_1$ & $y_2$ & $y_3$\\
\hline
\end{tabular}
}
\newsavebox{\CrossProdExRS}
\savebox{\CrossProdExRS}{%
\scriptsize%
\begin{tabular}{l| l | l | l | l}
\rowcolor{Gray}
\hline
$\boldsymbol{r_1}$ & $\boldsymbol{r_2}$ & $\boldsymbol{s_1}$ & $\boldsymbol{s_2}$ & $\boldsymbol{s_3}$\\
\hline
$x_1$ & $x_2$ & $y_1$ & $y_2$ & $y_3$\\
\hline
$x_3$ & $x_4$ & $y_1$ & $y_2$ & $y_3$\\
\hline
\end{tabular}
}


%
% this file contains latex boxes with tables used in the examples about
% recursion in SQL
%

%
% Example graph language influence language
%
\newsavebox{\LanguageInspiredTable}
\savebox{\LanguageInspiredTable}{%
\scriptsize%
\begin{tabular}{l | l}
\multicolumn{2}{l}{\textbf{Influence}}\\
\hline
\rowcolor{Gray}
\textbf{source} & \textbf{target}\\
\hline
Assembly & Speedcoding\\
\hline
Speedcoding & FORTRAN \\
\hline
Speedcoding & C \\
\hline
FORTRAN & C\\
\hline
C & C++\\
\hline
Smalltalk & ADA\\
\hline
ADA & C++\\
\hline
C & ADA\\
\hline
ADA & PL/SQL\\
\hline
\end{tabular}
}

%
% Example graph language influence language
%
\newsavebox{\GraphWithCycle}
\savebox{\GraphWithCycle}{%
\scriptsize%
\begin{tabular}{l | l}
\multicolumn{2}{l}{\textbf{Edge}}\\
\hline
\rowcolor{Gray}
\textbf{source} & \textbf{target}\\
\hline
$n_0$ & $n_1$ \\
\hline
$n_1$ & $n_2$ \\
\hline
$n_2$ & $n_1$\\
\hline
$n_2$ & $n_3$\\
\hline
\end{tabular}
}




\frame{\maketitle}

%
% ---------------------------------------------------------------------------------------------------
%
%%%%%%%%%%%%%%%%%%%%%%%% MUST BE THE SAME AS IN THE SQL SLIDES
\begin{frame}{The Relational Model of Data}
Relational databases were introduced in the 1970's, became prevalent in the 1980's and remain the industry standard for applications that manage large volumes of data

In CMPUT291/391 we study two relational query languages:\\
 - the algebra, which is clean and grounded on solid mathematical foundations\\
 - SQL, which is a very complex language with many nuances and intricacies introduced to make programmers lives easier

\begin{BOX}{The algebra and SQL are good friends}
While the algebra is procedural and SQL is declarative they are ``largely equivalent''. In fact, most DBMSs translate SQL queries into algebraic expressions and evaluate them that way.
\end{BOX}
\end{frame}

%
% ---------------------------------------------------------------------------------------------------
%
\begin{frame}

\begin{BOX}{Relation}
A relation is a \textbf{collection of} similar \textbf{facts} relevant to the application.

- \textbf{schema}: relation name, attribute types, constraints;\\
- \textbf{instance}: tuples.
\end{BOX}

In the context of the algebra, we ignore attribute types and constraints

\usebox{\MovieTable}
\end{frame}


%
% ---------------------------------------------------------------------------------------------------
%
%%%%%%%%%%%%%%%% MUST BE THE SAME IN THE SQL SLIDES
\begin{frame}
\vskip2em
\begin{BOX}{Relational Database}
A database is a \textbf{collection of} relations, each with its schema and constraints.

- \textbf{schema}: the schemata of the relations \emph{plus} (optionally) inter-relation constraints;\\
- \textbf{instance}: the instances of the relations.
\end{BOX}

\begin{BOX}{Legal database instance}
A database instance is \textbf{legal} if it satisfies all constraints in the schema.
\end{BOX}

\begin{BOX}{Constraint}
Logical condition over one or more tuples, derived from the application, that must be true at all times.
\end{BOX}
\end{frame}


%
% ---------------------------------------------------------------------------------------------------
%
\begin{frame} %%%% CONSTRAINTS

The relational algebra is taught as a theoretical tool that helps one understand and prove statements about relational databases and more complex languages such as SQL.

\vskip2em

We thus ignore practical issues such as data types and constraints when talking about the algebra.

\end{frame} %%%% CONSTRAINTS



%
% ---------------------------------------------------------------------------------------------------
%
\begin{frame} %%% EXAMPLE Movid DATABASE INSTANCE
\label{movie_instance}
\vspace*{0.25em}
\usebox{\FullInstance}
\end{frame} %%% EXAMPLE Movid DATABASE INSTANCE


%
%
% The relational algebra
%
%

%
% ---------------------------------------------------------------------------------------------------
%
\begin{frame}{The (Standard) Relational Algebra}

The relational model is defined on solid mathematical grounds:\\
 - relations are sets (or multisets) of tuples\\
 - query languages manipulate tuples or sets of tuples

\vskip0.5em

The Relational Algebra is an operator-based, procedural language that allows us to manipulate relational data.

\vskip0.5em

\begin{BOX}{Algebra operators result in new relations}

Every operator in the algebra takes one or two relations as input and \textbf{produces a new relation} as output.

\vskip1em

This is what allows operators to be combined to form complex queries.
\end{BOX}

\end{frame}


%
% ---------------------------------------------------------------------------------------------------
%
\begin{frame}[fragile] %%%% BASIC ALGEBRA OPERATORS

The (horizontal) \textbf{selection} operator \( \sigma_{c} (R) \):\\
 - creates a new relation with all tuples from $R$ that satisfy the boolean condition $c$\\
 - the \textbf{schema} of the resulting relation is the same as $R$

\vfill
\begin{center}
\begin{tikzpicture}[align=center,node distance=3cm,every node/.style={inner sep=0.5,outer sep=0.5}]
	\node (1) at (0,0) {\scalebox{0.75}{\usebox\BillMurrayMovies}};
	\node (2) [below of=1] {\scalebox{0.75}{\usebox\CastTable}};
	\path[commutative diagrams/.cd, every arrow, every label]
		(2) edge node[swap] {$\sigma_{\text{actor='Bill Murray'}}(\text{Cast})$} (1);   
\end{tikzpicture}
\end{center}

\end{frame} 



%
% ---------------------------------------------------------------------------------------------------
%
\begin{frame}[fragile] %%%% BASIC ALGEBRA OPERATORS

The \textbf{projection} operator \( \pi_{a_1,\ldots,a_n} (R) \):\\
 - creates a new relation with attributes $a_1,\ldots,a_n$ of tuples from $R$\\
 - the \textbf{schema} of the resulting relation is $a_1,\ldots,a_n$

\vfill
\begin{center}
\begin{tikzpicture}[align=center,node distance=3cm,every node/.style={inner sep=0.5,outer sep=0.5}]
	\node (1) at (0,0) {\scalebox{0.75}{\usebox\MovieDirectors}};
	\node (2) [below of=1] {\scalebox{0.75}{\usebox\MovieTable}};
	\path[commutative diagrams/.cd, every arrow, every label]
		(2) edge node[swap] {$\pi_{\text{director}}(\text{Movie})$} (1);   
\end{tikzpicture}
\end{center}
\end{frame} 


%
% ---------------------------------------------------------------------------------------------------
%
\begin{frame}[fragile] %%%% BASIC ALGEBRA OPERATORS

The \textbf{cross product} operator \( R \times S \):\\
 - creates a new relation with the concatenation of every tuple in $R$ with every tuple in $S$\\
 - the \textbf{schema} of the resulting relation has every attribute in $R$ followed by every attribute in $S$

\vfill
\begin{center}
\begin{tikzpicture}[align=center,node distance=1.5cm,every node/.style={inner sep=0.5,outer sep=0.5}]
    \node (1) at (0,0) {\scalebox{0.75}{\usebox\CrossProdExRS}};
    \node (2) [below of=1]  {$\times$};
    \node (3) [below left=0.35cm and 0.25cm of 2] {\scalebox{0.75}{\usebox\CrossProdExR}};
    \node (4) [below right=0.35cm and 0.25cm of 2] {\scalebox{0.75}{\usebox\CrossProdExS}};
    \path[commutative diagrams/.cd, every arrow, every label]
    	(2) edge node {} (1)
    	(3) edge node {} (2)
    	(4) edge node {} (2);
\end{tikzpicture}
\end{center}
\end{frame} 


%
% ---------------------------------------------------------------------------------------------------
%
\begin{frame} %%% JOIN ALGEBRA OPERATORS

The \textbf{join} of two relations $R \Join_{c} S$ concatenates \emph{matching} 
tuples from $R$ and $S$ and is defined as follows:
\[R \Join_{c} S \equiv \sigma_{c} \left( R \times S \right)\]

\vskip0.5em

\begin{BOX}{Equi-joins and Natural Joins}

The join condition $c$ normally involves attributes from both relations.

\vskip0.5em

An \textbf{equijoin} is a join where all conditions are equalities.

\vskip0.5em

The \textbf{natural join}\footnotemark of $R$ and $S$ written just $R \Join S$ is an equijoin defined over all attributes with the same name in $R$ and $S$.
\end{BOX}

\footnotetext{Some textbooks refer to the natural join simply as a ``join'', calling the join with a condition a theta-join: $\Join_\theta$}
\end{frame} %%% JOIN ALGEBRA OPERATORS


%
% ---------------------------------------------------------------------------------------------------
%
\begin{frame} %%%% JOIN EXAMPLES

Example: \( \text{Movie} \Join \left( \sigma_{\text{actor='Bill Murray'}}(\text{Cast}) \right) \)

\begin{center}
\begin{tikzpicture}[align=center,node distance=1.5cm,every node/.style={inner sep=0.5,outer sep=0.5}]
    \node (1) at (0,0) {\scalebox{0.75}{\usebox\JoinMoviesBillMurrayMovies}};
    \node (2) [below of=1]  {$\Join$};
    \node (3) [below left=0.35cm and 0.25cm of 2] {\scalebox{0.6}{\usebox\MovieTable}};
    \node (4) [below right=0.35cm and 0.25cm of 2] {\scalebox{0.6}{\usebox\BillMurrayMovies}};
    \node (5) [below=0.75cm of 4] {\scalebox{0.6}{\usebox\CastTable}};
    \path[commutative diagrams/.cd, every arrow, every label]
    	(2) edge node {} (1)
    	(3) edge node {} (2)
    	(4) edge node {} (2)
    	(5) edge node[swap] {$\sigma_{\text{actor='Bill Murray'}}(\text{Cast})$} (4);
\end{tikzpicture}
\end{center}

\end{frame} %%%% JOIN EXAMPLES


%
% ---------------------------------------------------------------------------------------------------
%
\begin{frame} %%% SCHEMA OF A JOIN

The \textbf{schema} of the relation resulting from a join has all attributes in the incoming relations.

\vskip1em

For convenience, in these slides, we omit repeated attributes for the sake of clarity, as in:
 \[ \text{Movie} \Join \left( \sigma_{\text{actor='Bill Murray'}}(\text{Cast}) \right) \]

 \begin{center}
 \scalebox{0.75}{\usebox\JoinMoviesBillMurrayMovies}
 \end{center}

 \vskip1em

 Different DBMSs handle these situations differently. Try an example like this on SQLite!

\end{frame} %%% SCHEMA OF A JOIN


%
% ---------------------------------------------------------------------------------------------------
%
\begin{frame} %%%%  OUTER-JOINS

The join operator ignores all tuples from either relation that do not match with any tuple in the other relation. To keep non-matching tuples we can use an \textbf{outerjoin}.\footnote{The Garcia-Molina, Ullman, Widom book uses a different notation: $R \TextbookOuterJoin S$}

The \textbf{left outerjoin} \(R \LeftOuterJoin S\) keeps unmatched tuples in $R$ with NULL values for the missing tuple in $S$.

\vskip0.5em 

Example: \(\text{Cinema} \LeftOuterJoin_{\text{name=theater}} \text{Guide}\):
\begin{center}
\scalebox{0.75}{\usebox\LeftOuterJoinCinemaGuide}
\end{center}

The \textbf{right outerjoin} \(R \RightOuterJoin S\) and the \textbf{full outerjoin} \(R \FullOuterJoin S \) are defined analogously.

\end{frame} %%%%  OUTER-JOINS


%
% ---------------------------------------------------------------------------------------------------
%
\begin{frame} %%%% SEMIJOINS

Another variant of the join operator is the \textbf{semijoin}.

\vskip1em

The \textbf{left semijoin} \(R \ltimes_{c} S \) produces a relation with tuples from $R$ that have a matching tuple in $S$.

\vskip2em

\begin{columns}[onlytextwidth]
\begin{column}{0.57\textwidth}
Example\footnotemark:\\\hspace*{2em} \(\text{Cinema} \ltimes_{\text{name=theater}} \text{Guide}\)
\end{column}
\begin{column}{0.42\textwidth}
\scalebox{0.75}{\usebox\SemijoinCinemaGuide}
\end{column}
\end{columns}

\vskip2em

\footnotetext{Note that the result does not have duplicates, even though there are two matches in Guide for the Garneu tuple in Cinema.}

The \textbf{right semijoin} \(R \rtimes_{c} S \) is defined analogously.
\end{frame} %%%% SEMIJOINS


%
% ---------------------------------------------------------------------------------------------------
%
\begin{frame}{Set operations} %%% SET OPERATORS

In the standard relational model and algebra, \textbf{relations are sets} and thus one can use the other set operators (besides cross product) on them.

\vskip1em

In this case, however, we require the relations to be \emph{compatible} (i.e., have the same schema).

 - \(R \cup S\) : all tuples in $R$ or $S$ (or both)

 - \(R \cap S\) : all tuples common to $R$ and $S$

 - \(R - S\) : all tuples in $R$ but not in $S$

\end{frame} %%% SET OPERATORS



%
% ---------------------------------------------------------------------------------------------------
%
\begin{frame}{Renaming} %%%%% RENAME OPERATOR

Consider finding actors who starred in a movie and a remake of the movie. From slide~\ref{movie_instance} we gather actress Sigourney Weaver would be an answer, as she acted on the original Ghostbusters of 1984 and the remake in 2016\footnote{Although missing from the example, Dan Aykroyd would be another answer.}

The query we seek has to concatenate \emph{two tuples from Cast} (one for the original move, and the other for the remake).

We can find the concatenations with the Cartesian product of Cast with \emph{itself}, but, we need to \textbf{rename} the attributes in one of the copies to express a selection condition that the years are different

\begin{center}
\scalebox{0.7}{\usebox\SelfJoinExampleSigourneyWeaver}
\end{center}
\end{frame} %%%%% RENAME OPERATOR


%
% ---------------------------------------------------------------------------------------------------
%
\begin{frame} %%%% RENAME OPERATOR

The renaming operator $\rho_{S(b_1,\ldots,b_n)}(R)$ creates a ``copy'' of a relation $R(a_1,\dots,a_n)$ that has a different schema but the same instance.

Thus, the query we are looking for is

\[ \pi_{\text{actor}}\left(\text{Cast} 
\Join_{\substack{\text{title=title2}\\\wedge\ \text{actor=actor2}\\\wedge\ \text{year}\neq\text{year2}}} 
\rho_{\text{Cast2(title2,year2,actor2,role2)}}(\text{Cast})\right) \]

\vskip2em

The $\wedge$ in the join condition stands for the \highlight{conjunction} (or logical ``and'') of two Boolean expression.

\end{frame} %%%% RENAME OPERATOR


%
% ---------------------------------------------------------------------------------------------------
%
\begin{frame}
\label{conjunctiveQueriesMonotonicity}
\vspace*{2em}
\begin{BOX}{Conjunctive queries}
A query is said to be conjunctive if it uses only $\sigma$, $\pi$, $\rho$ and $\Join$ operators and all selection/join conditions $c$ involve only single predicates or conjunctions of predicates.
\end{BOX}

\begin{BOX}{Positive queries}
A query is said to be positive if it does not use the set difference (-) operator.
\end{BOX}

\begin{BOX}{Positive queries are monotonic}
A monotonic query is such that \emph{inserting} new tuples in the database cannot \emph{reduce} the number of tuples in the answer to the query.
\end{BOX}
\end{frame}


%
% ---------------------------------------------------------------------------------------------------
%
\begin{frame}{Rewriting algebra expressions}

Conjunctive predicates can be rewritten as the intersection of queries... and vice-versa!

\textbf{Example}: movies with Sigourney Weaver \emph{and} from year 1984:
\[ \pi_{\text{title,year}}\left(
\sigma_{\text{actor='Sigourney Weaver'}\alert{\wedge}\ \text{year}=1984}
\left(\text{Cast} \right)\right)\]

is equivalent to
\[
\pi_{\text{title,year}}
\left(
\sigma_{\text{actor='Sigourney Weaver'}}(\text{Cast})
\alert{\cap}
\sigma_{\text{year}=1984}(\text{Cast})
\right)
\]

This property is actually useful for a DBMS developer, as it allows the query optimizer to modify the original query provided by the user in a way to achieve better performance.
\end{frame}


%
% ---------------------------------------------------------------------------------------------------
%
\begin{frame}
\label{disjunctionAndUnion}
Disjunctive predicates can be rewritten as the union of queries... and vice-versa!

\textbf{Example}: movies with Sigourney Weaver \emph{or} Bill Murray:


\[ \pi_{\text{title,year}}\left(
\sigma_{\text{actor='Sigourney Weaver'}\ \alert{\vee}\ \text{actor='Bill Murray'}}
\left(\text{Cast}\right)\right)\]

is equivalent to
\[
\pi_{\text{title,year}}
\left(
\sigma_{\text{actor='Sigourney Weaver'}}(\text{Cast})
\alert{\cup}
\sigma_{\text{actor='Bill Murray'}}(\text{Cast})
\right)
\]
\end{frame}


%
% ---------------------------------------------------------------------------------------------------
%
\begin{frame}{}

When we say~~~~ 
\(Q_1: \ \ \ \sigma_{a_1=x_1 \ \wedge\ \cdots\ \wedge a_n=x_n}(R)\)

is equivalent to ~~~~
\(Q_2: \ \ \ \sigma_{a_1=x_1}(R) \ \cap \cdots \cap \sigma_{a_n=x_n}(R)\)

We mean that, for any instance of $R$, every tuple $t$ in the result of $Q_1$ \textbf{must also} be in the result of $Q_2$

\highlight{Statements like this are useless unless \textbf{proven} correct.}\footnote{Although CMPUT391 does not cover \emph{database theory}, the good student should understand and try to prove equivalences mentioned in the course.}

Sample proof by \emph{contradiction}. Assume there is an instance of $R$ for which there is a tuple $t$ that is in the answer of $Q_1$ but not in the answer of $Q_2$. However, if $t$ is in the answer of $Q_1$ it must satisfy every predicate $a_i=x_i$; meaning it must be in the answer of $Q_2$. This contradiction proves the queries \emph{are} equivalent.

\end{frame}


%
% ---------------------------------------------------------------------------------------------------
%
\begin{frame}{Other equivalence-based rewrite rules}
\label{commutativeAssociativeRules}

Many algebra operators are both associative and commutative\footnote{Note that there is no ``order'' for the attribute names in a relation schema}:\\
- \(R\times S = S \times R\); ~~~ \((R\times S)\times T = R \times (S\times T)\)\\
- \(R\Join S = S \Join R\); ~~~ \((R\Join S)\Join T = R \Join (S\Join T)\)\\
- \(R\cap S = S \cap R\); ~~~ \((R\cap S)\cap T = R \cap (S\cap T)\)\\
- \(R\cup S = S \cup R\); ~~~ \((R\cup S)\cup T = R \cup (S\cup T)\)

\vskip1em

The standard join is commutative \(R\Join_c S = S\Join_c R\)  but generally not associative (as the attributes in the condition $c$ may not exist in all relations).
\end{frame}


%
% ---------------------------------------------------------------------------------------------------
%
\begin{frame}
\label{otherEquivalencesWithSelections}

\textbf{More equivalences involving selections}

\begin{itemize}
\item Cascading selections
\[\sigma_{c_1\ \wedge\ c_2}(R) \ = \ \sigma_{c_1}\left(\sigma_{c_2}(R)\right) \ = \ \sigma_{c_2}\left(\sigma_{c_1}(R)\right)\]

\item ``Pushing down'' through union
\[\sigma_c(R\cup S) \ = \ \sigma_c(R) \cup \sigma_c(S)\]

\item ``Pushing down'' through intersection
\[\sigma_c(R\cap S)\ = \ \sigma_c(R)\cap S \ = \ R \cap \sigma_c(S)\]

\item ``Pushing down'' through set difference
\[\sigma_c(R-S) \ = \ \sigma_c(R) - S\]
\end{itemize}
\end{frame}



%
% ---------------------------------------------------------------------------------------------------
%
\begin{frame}
\label{pushingDownSelectionsThroughJoins}

\textbf{Pushing down selections through joins}

The general rule is to push the selection condition ``down'' to the relation participating in the join that contains the attributes in the selection condition. If both relations have the attributes, the selection can be pushed both ways.

This rule applies to Cartesian products as well:

\[\sigma_c(R \Join S) \ = \ \sigma_{c^R}(R) \Join \sigma_{c^S}S\]
\[\sigma_c(R \Join_\theta S) \ = \ \sigma_{c^R}(R) \Join_\theta \sigma_{c^S}S\]
\[\sigma_c(R \times S) \ = \ \sigma_{c^R}(R) \times \sigma_{c^S}S\]

\vfill 

$c^R$ are the conditions involving only attributes of $R$; similarly for $c^S$.
\end{frame}


%
% ---------------------------------------------------------------------------------------------------
%
\begin{frame}

There aren't as many useful equivalence rules for \textbf{projections} as there are for selections, however.

\vskip0.5em

The main reason is that whenever we ``push'' a projection down, we must bring along all attributes used in other operators. For joins, we can rewrite \[\pi_{\mathbf{a}}(R\Join_\mathbf{b} S) \ = \ 
\pi_{\mathbf{a}}\left(\pi_{\mathbf{c}}(R) \Join_\mathbf{b} \pi_\mathbf{d} (S)\right)\]

Where $\mathbf{a}, \mathbf{b}, \mathbf{c}$ and $\mathbf{d}$ are \textbf{sets of attributes}.

Note that the expressions above are equivalent only if both $\mathbf{c}$ and $\mathbf{d}$ contain \textbf{all} attributes in both $\mathbf{a}$ and $\mathbf{b}$. 

Also, the outermost projection on $\mathbf{a}$ must remain unless $\mathbf{b}, \mathbf{c}$, 
and $\mathbf{d}$ form a partition $\mathbf{a}$.

\end{frame}


%
% ---------------------------------------------------------------------------------------------------
%
\begin{frame}{The Algebra is a procedural language}

Expressions in the relational algebra can be parsed (obviously and unambiguously) into trees, where the leaf nodes contain base tables and inner nodes contain algebra operators.

\vskip0.5em

This gives an immediate algorithm for computing the result of queries: starting from the root, perform a (depth-first) traversal of the tree computing the relation resulting from the expression in every node in the tree, computing the inputs recursively.

\vskip0.5em

The simple procedure is the basis of the query execution engine in most relational DBMSs out there!
\end{frame}


\end{document}