%!TEX root = lec05_query_processing.tex


%
% ------------------------------------------------------------------------
%
\begin{frame}{Goal of the Query Optimizer}

The goal of the query optimizer is to find \textbf{optimal query plans}. Because of \alert{\textbf{Rule \#1}} below, DBMSs aim for plans with the lowest possible \textbf{estimated}\footnote{Recall that to find the \textbf{actual} cost, the DBMS would have to execute the plan. It makes no sense to use actual costs of multiple plans.} cost.

In general, query optimization is NP-hard. In practice, the optimizer considers only a subset of the possible query plans for each query, and picks a plan with lowest estimated cost.

\vskip1em

\begin{block}{\alert{Rule \#1} of query optimization}
The time it takes the optimizer to find a good plan cannot exceed the time it would take to execute a ``reasonable'' plan.
\end{block}
\end{frame}

\newsavebox{\leftDeepTree}
\savebox{\leftDeepTree}{
	\begin{tikzpicture}[node distance=0.25cm,every node/.style={inner sep=1,outer sep=1,font=\footnotesize}]
	\node (0) {$\Join$};
	\node (1) [below left=of 0] {$\Join$};
	\node (2) [below left=of 1] {$\Join$};
	\node (R) [below left= of 2] {$R$};
	\node (S) [below right= of 2] {$S$};
	\node (T) [below right= of 1] {$T$};
	\node (U) [below right= of 0] {$U$};
    \path[->]
        (1) edge (0)
        (2) edge (1)
        (R) edge (2)
        (S) edge (2)
        (T) edge (1)
        (U) edge (0);
\end{tikzpicture}}

\newsavebox{\rightDeepTree}
\savebox{\rightDeepTree}{
	\begin{tikzpicture}[node distance=0.25cm,every node/.style={inner sep=1,outer sep=1,font=\footnotesize}]
	\node (0) {$\Join$};
	\node (1) [below right=of 0] {$\Join$};
	\node (2) [below right=of 1] {$\Join$};
	\node (R) [below left= of 0] {$R$};
	\node (S) [below left= of 1] {$S$};
	\node (T) [below left= of 2] {$T$};	
	\node (U) [below right= of 2] {$U$};
    \path[->]
        (1) edge (0)
        (2) edge (1)
        (R) edge (0)
        (S) edge (1)
        (T) edge (2)
        (U) edge (2);
\end{tikzpicture}}


\newsavebox{\bushyTree}
\savebox{\bushyTree}{
	\begin{tikzpicture}[node distance=0.25cm,every node/.style={inner sep=1,outer sep=1,font=\footnotesize}]
	\node (0) {$\Join$};
	\node (1) [below left=of 0] {$\Join$};
	\node (2) [below right=of 0] {$\Join$};
	\node (R) [below left= of 1] {$R$};
	\node (S) [below right= of 1,xshift=-7.5pt] {$S$};
	\node (T) [below left= of 2,xshift=7.5pt] {$T$};	
	\node (U) [below right= of 2] {$U$};
    \path[->]
        (1) edge (0)
        (2) edge (0)
        (R) edge (1)
        (S) edge (1)
        (T) edge (2)
        (U) edge (2);
\end{tikzpicture}}

\newsavebox{\wiggleLeftTree}
\savebox{\wiggleLeftTree}{
	\begin{tikzpicture}[node distance=0.25cm,every node/.style={inner sep=1,outer sep=1,font=\footnotesize}]
	\node (0) {$\Join$};
	\node (1) [below right=of 0] {$\Join$};
	\node (2) [below left=of 1] {$\Join$};
	\node (R) [below left= of 0] {$R$};
	\node (S) [below left= of 2] {$S$};
	\node (T) [below right= of 2] {$T$};	
	\node (U) [below right= of 1] {$U$};
    \path[->]
        (1) edge (0)
        (2) edge (1)
        (R) edge (0)
        (S) edge (2)
        (T) edge (2)
        (U) edge (1);
\end{tikzpicture}}

\newsavebox{\wiggleRightTree}
\savebox{\wiggleRightTree}{
	\begin{tikzpicture}[node distance=0.25cm,every node/.style={inner sep=1,outer sep=1,font=\footnotesize}]
	\node (0) {$\Join$};
	\node (1) [below left=of 0] {$\Join$};
	\node (2) [below right=of 1] {$\Join$};
	\node (R) [below left= of 1] {$R$};
	\node (S) [below left= of 2] {$S$};
	\node (T) [below right= of 2] {$T$};	
	\node (U) [below right= of 0] {$U$};
    \path[->]
        (1) edge (0)
        (2) edge (1)
        (R) edge (1)
        (S) edge (2)
        (T) edge (2)
        (U) edge (0);
\end{tikzpicture}}


%
% -----------------------------------------------------------------------------------------------
%
\begin{frame}{Query Optimization With Dynamic Programming (Sellinger, 1979)}

Dynamic programming is an algorithmic problem-solving technique suitable for optimization problems where the solution builds on solutions to sub-problems. 

\underline{Dynamic programming works bottom-up}: decompose the problem into every atomic sub-problem and solve them; the solution with $n+1$ sub-problems builds on the lowest cost solution of size $n$.

\vskip2em

\begin{center}
\begin{tikzpicture}[node distance=0.25cm,every node/.style={inner sep=1,outer sep=1,font=\footnotesize}]

\node (0) {$\pi_{a_1,\ldots,a_n}$};
\node (1) [below= of 0] {$\sigma_{c_1 \wedge c_2}$};
\node (2) [below= of 1] {$\Join$};
\node (3) [below left= of 2] {$\Join$};
\node (4) [below right= of 2] {$T$};
\node (3a) [below left= of 3] {$R$};
\node (3b) [below right= of 3] {$S$};
\path[->]
        (1) edge (0)
        (2) edge (1)
        (3) edge (2)
        (4) edge (2)
        (3a) edge (3)
        (3b) edge (3);

\node (5) [right= 1cm of 1] {$\pi_{a_1,\ldots,a_n}$};
\node (6) [right= of 5] {$\sigma_{c_1}$};
\node (7) [right= of 6] {$\sigma_{c_2}$};
\node (8) [right= of 7] {$\Join$};
\node (9) [right= of 8] {$\Join$};
\node (10) [right= of 9] {$R$};
\node (11) [right= of 10] {$S$};
\node (12) [right= of 11] {$T$};
\end{tikzpicture}
\end{center}
\end{frame}

%
% -----------------------------------------------------------------------------------------------
%
\begin{frame}{Dynamic Programming Optimizer}

\textbf{Step 1:} for each atomic sub-problem $Q$, compute:
\begin{itemize}[-,topsep=-0.5em,noitemsep]
\item $T(Q) \text{ or } B(Q)$: the estimated size of $Q$
\item $\mathit{plan}(Q)$: the lowest-cost plan to compute $Q$
\item $\alert{\mathit{cost}(Q)}$: the actual cost to compute $Q$.
\end{itemize}

\vskip1em

\textbf{Step 2:} build, recursively, all legal solutions of size $n+1$, considering all combinations of a solution of size $n$ and the solution to another sub-problem. Keep the one with least cost.

\vskip1em

Invalid partial solutions (e.g., doing a projection ``before scanning a table'') are discarded.
\end{frame}


\begin{frame}

The \alert{cost function} needs to take into account \textbf{both} \blue{I/O} \textbf{and} \blue{memory} costs. However, these two costs have different units, so we cannot just add them up.

Different systems achieve this goal differently. One simple way is to consider the number of tuples that are read and computed. 

For example:

\begin{itemize}[-,topsep=-0.5em,noitemsep]
\item $\mathit{cost}(R) = T(R)$
\item $\mathit{cost}(\rho_{S(s_1,\ldots,s_n)}(R)) = \alert{\mathit{cost}(R)}$
\item $\mathit{cost}(\sigma_C(R)) = T(\sigma_C(R)) + \alert{\mathit{cost}(R)}$
\item $\mathit{cost}(\pi_{a_1,\ldots,a_n}(R)) = T(\pi_{a_1,\ldots a_n}(R)) + \alert{\mathit{cost}(R)}$
\item $\mathit{cost}(R\Join_C S)\blue{\footnotemark} = T(R \Join_C S) + \alert{\mathit{cost}(R)+\mathit{cost}(S)}$ 
\end{itemize}

\footnotetext{This is an in-memory nested-loop-join.}
\end{frame}

% %
% % -----------------------------------------------------------------------------------------------
%
\begin{frame}{Why is Join Ordering Important?}

There is a combinatorial number of plans that join $N$ relations. 

\textbf{Example:} $N=4$. Here are the 5 ``shapes'' for a 4-way join:

\begin{center}
\begin{tikzpicture}[node distance=0.25cm]
\node (0) {\scalebox{0.75}{\usebox{\leftDeepTree}}};
\node (1) [right=of 0] {\scalebox{0.75}{\usebox{\rightDeepTree}}};
\node (2) [right=of 1] {\scalebox{0.75}{\usebox{\bushyTree}}};
\node (3) [right=of 2] {\scalebox{0.75}{\usebox{\wiggleLeftTree}}};
\node (4) [right=of 3] {\scalebox{0.75}{\usebox{\wiggleRightTree}}};
\end{tikzpicture}
\end{center}

With each shape\footnote{The number of shapes comes from \url{https://en.wikipedia.org/wiki/Catalan_number}.} we can have any of the permutations of the relations. With relations there are $4!(\frac{6!}{(4! \cdot 3!)})=120$ alternatives. 

With 5 relations, there are $5!(\frac{8!}{(5! \cdot 4!)})=1,680$ alternatives.

\end{frame}

%
% -----------------------------------------------------------------------------------------------
%
\begin{frame}
How many \textbf{actual} plans? 

The space of possible plans considers, \underline{for every single one} of the 120 ways of performing the 4-way join, \underline{all possible} \textbf{legal rewrites} of the query (i.e., pushing selections down or not?) and, for each rewrite, \underline{all possible} algorithms for each of the joins (nested-loop, hash-join, merge-join,...) and scans (table scan, index scan, if so, which one?).

Considering all these plans would almost always take longer than executing a ``reasonable'' one.

\vskip1em
\end{frame}


%
% -----------------------------------------------------------------------------------------------
%
\begin{frame}{Left Deep Plans}

\begin{columns}[onlytextwidth]
\begin{column}{0.5\textwidth}
Back to the $N=4$ case. 

\vskip1em

Among all 120 plans, only $4!=24$ are ``left-deep''
\end{column}
\begin{column}{0.3\textwidth}
{\scalebox{1}{\usebox{\leftDeepTree}}}
\end{column}
\end{columns}

\vskip1em

A ``left-deep'' binary tree is one in which all right children are leaves.

There are two main advantages of restricting the search to left-deep plans:
\begin{enumerate}[(1),topsep=-0.5em,noitemsep]
\item The search space is much smaller.
\item These plans work well in the pipeline fashion of the iterator-style query answering.
\end{enumerate}

\end{frame}

%
% -----------------------------------------------------------------------------------------------
%
\begin{frame}[fragile]
\label{worked_example}
\textbf{Worked example:} Finding the best left-deep plan based on estimated cost.

\vskip2em

\begin{columns}[onlytextwidth]
\begin{column}{0.4\textwidth}
$R(a,b,c)$ \qquad $S(c,d,e)$\\
 $T(b,f,g)$ 

\vskip2em

\begin{lstlisting}[style=SQL]
SELECT *
FROM R JOIN S 
   JOIN T JOIN  U
WHERE f > 10
\end{lstlisting}
\end{column}
\small
\begin{column}{0.6\textwidth}
\begin{tabular}{c|r|rrr}
& $T(\cdot)$ & \multicolumn{3}{c}{$V(\cdot,\cdot)$}\\
\hline
\multirow{2}{*}{$R$} & \multirow{2}{*}{10,000} & $a$ & $b$ & $c$\\
& & 500 & 1000 & 500 \\
\hline
\multirow{2}{*}{$S$} & \multirow{2}{*}{20,000} & $c$ & $d$ & $e$\\
& & 2,000 & 200 & 250 \\
\hline
\multirow{2}{*}{$T$} & \multirow{2}{*}{3,000} & $b$ & $f$ & $g$\\
& & 1,500 & 20 & 2 \\
\hline
\end{tabular}
\end{column}
\end{columns}

\vskip1em

Assume:

\begin{itemize}[-,noitemsep,topsep=-5pt]
 \item $s(f > 10, T) = 0.01$
\item $\MIN(f, T) = -20$
\item $\MAX(f, T) = 20$
\end{itemize} 

\end{frame}

\def\selectionPlan#1#2{
	\begin{tikzpicture}[node distance=0.75cm,every node/.style={inner sep=1,outer sep=1,font=\scriptsize}]
	\node (0) {#1};
	\node (1) [below of= 0] {#2};
	\path[->] (1) edge (0);
	\end{tikzpicture}
}

\def\twoRelationLDPlan#1#2{
	\begin{tikzpicture}[node distance=0.25cm,every node/.style={inner sep=1,outer sep=1,font=\scriptsize}]
	\node (0) {$\Join$};
	\node (1) [below left= of 0,xshift=5pt] {$#1$};
	\node (2) [below right= of 0,xshift=-5pt] {$#2$};
    \path[->]
        (1) edge (0)
        (2) edge (0);
	\end{tikzpicture}
}


\def\threeRelationLDPlan#1#2#3{
	\begin{tikzpicture}[node distance=0.25cm,every node/.style={inner sep=1,outer sep=1,font=\scriptsize}]
	\node (0) {$\Join$};
	\node (1) [below left= of 0,xshift=5pt] {$\Join$};
	\node (2) [below left= of 1,xshift=5pt] {$#1$};
	\node (3) [below right= of 1,xshift=-5pt] {$#2$};
	\node (4) [below right= of 0,xshift=-5pt] {$#3$};
    \path[->]
        (1) edge (0)
        (2) edge (1)
        (3) edge (1)
        (4) edge (0);
	\end{tikzpicture}
}


%
% -----------------------------------------------------------------------------------------------
%
\begin{frame}


\textbf{All} plans of size 1 (i.e., with a single node) are table scans:

\begin{tabular}{r@{\hspace*{2em}}rr}
$Q= R$ & $T(Q) = 10K$ & $\mathit{cost}(Q) = 10K$ \\
$Q= S$ & $T(Q) = 20K$ & $\mathit{cost}(Q) = 20K$ \\
$Q= T$ & $T(Q) = 3K$ & $\mathit{cost}(Q) = 3K$ \\
\end{tabular}

The nodes $\Join$, $\Join$, and $\sigma_{f>10}$ cannot be used alone and thus cannot be part of a single-node plan 


\vskip3em

There is \textbf{only one} valid plan of size 2:

\begin{tabular}{r@{\hspace*{2em}}rr}
 $Q=$ \selectionPlan{$\sigma_{f>10}$}{$T$} & $T(Q)=3K \cdot 0.01 = 30$ & $\mathit{cost}(Q) = 3.03K$
\end{tabular}

\alert{NOTE:} $V(\sigma_{f>10}(T), b) = 30$, although the original table had a lot more distinct values of $b$ (recall slide~\ref{worked_example}).

\end{frame}

\begin{frame}

Now, plans with a \textbf{single join}, starting with those of size 3:

\begin{tabular}{r@{\hspace*{2em}}rr}
 $Q=$ \twoRelationLDPlan{S}{R} & $T(Q) = \frac{10K \cdot 20K}{\max(500,2K)}=100K$ & $\mathit{cost}(Q) = 130K$ \\
 $Q=$ \twoRelationLDPlan{R}{T} & $T(Q) = \frac{10K \cdot 3K}{\max(1K,1.5K)}=20K$ & $\mathit{cost}(Q) = 33K$ \\
 $Q=$ \twoRelationLDPlan{S}{T} & $T(Q) = 20K\cdot3K =60M$ & $\mathit{cost}(Q) = 60.023M$ \\
\end{tabular}

\vskip2em

\alert{NOTE:} query plan \twoRelationLDPlan{S}{T} has a high cost because it is a \emph{cross product}: there are no attributes common to $S$ and $T$.
\begin{itemize}[-,topsep=-5pt]
\item A modern optimizer will immediately prune that sub-plan as it cannot be better than any other plan.
\end{itemize}

\end{frame}

%
% -----------------------------------------------------------------------------------------------
%

\def\joinSelectionLDplan#1{
	\begin{tikzpicture}[node distance=0.25cm,every node/.style={inner sep=1,outer sep=1,font=\scriptsize}]
	\node (0) {$\Join$};
	\node (1) [below left= of 0,xshift=5pt] {$#1$};
	\node (2) [below right= of 0,xshift=-5pt] {$\sigma_{f>10}$};
	\node (3) [below of= 2, yshift=-0.35cm] {$T$};
    \path[->]
        (1) edge (0)
        (2) edge (0)
        (3) edge (2);
	\end{tikzpicture}
}

\def\selectionJoinLDplan#1{
	\begin{tikzpicture}[node distance=0.25cm,every node/.style={inner sep=1,outer sep=1,font=\scriptsize}]
	\node (0) {$\sigma_{f>10}$};
	\node (1) [below of= 0, yshift=-0.35cm] {$\Join$};
	\node (2) [below left= of 1,xshift=5pt] {$#1$};
	\node (3) [below right= of 1,xshift=-5pt] {$T$};
    \path[->]
        (1) edge (0)
        (2) edge (1)
        (3) edge (1);
	\end{tikzpicture}
}


\begin{frame}

Now, plans of size 4 (single-join with $T$ and $\sigma_{f>10}$).

We will ignore the plans involving the cross product $S\Join T$, and consider the only two plans involving $R$ and $T$:

\vskip2em


\begin{columns}[onlytextwidth]
\begin{column}{0.3\textwidth}
$Q = $ \joinSelectionLDplan{R}
\end{column}
\begin{column}{0.3\textwidth}
\begin{align*}
T(Q) = \frac{10K\cdot 30}{\max(1K,30)} = 300\\
\mathit{cost}(Q) = 300 + 10K + 3.03K = \alert{13.33K}
\end{align*}
\end{column}
\end{columns}

\vskip2em

For the next one, we re-use the work done for plans of size 3.

\begin{columns}[onlytextwidth]
\begin{column}{0.3\textwidth}
$Q = $ \selectionJoinLDplan{R}
\end{column}
\begin{column}{0.3\textwidth}
\begin{align*}
T(Q) = 20K\cdot 0.01 = 200\\
\mathit{cost}(Q) = 200 + 33K = 33.2K
\end{align*}
\end{column}
\end{columns}


\end{frame}


%
% -----------------------------------------------------------------------------------------------
%
\begin{frame}

\vskip1em

Now for plans of size 5, which would be the best plans of size 3 plus another join (and another table).

\vskip1em

\begin{columns}[onlytextwidth]
\begin{column}{0.3\textwidth}
$Q = $ \threeRelationLDPlan{S}{R}{T} 
\end{column}
\begin{column}{0.3\textwidth}
\begin{align*}
 T(Q) &= \frac{T(R,S,\Join)T(T)}{\max(V(b,R\Join S),V(b,T))}\\
&=\frac{100K\cdot 3K}{\max(1K, 1.5K)}=200K\\
\mathit{cost(Q)} &= 200K + 130K + 3K = 363K
\end{align*}
\end{column}
\end{columns}

\vskip2em

\begin{columns}[onlytextwidth]
\begin{column}{0.3\textwidth}
$Q = $ \threeRelationLDPlan{R}{T}{S} 
\end{column}
\begin{column}{0.3\textwidth}
\begin{align*}
T(Q) &= \frac{20K \cdot 20K}{\max(500,2K)} =200K\\
\mathit{cost(Q)} &= 200K + 33K + 20K = \alert{253K}
\end{align*}
\end{column}
\end{columns}


\vskip2em

Again, we ignore the plan with $S\Join T$.


\end{frame}

%
% -----------------------------------------------------------------------------------------------
%


\begin{frame}

An now for plans of size 6, that will include all nodes.


\vskip1em

\textbf{Option 1:} add $\sigma_{f>10}$ to the \textbf{best} plan of size 5.

\vskip1em

\begin{columns}[onlytextwidth]
\begin{column}{0.3\textwidth}
$Q = $ 	\begin{tikzpicture}[node distance=0.25cm,every node/.style={inner sep=1,outer sep=1,font=\scriptsize}]
	\node (s) {$\sigma_{f>10}$};
	\node (0) [below of=s, yshift=-0.25cm] {$\Join$};
	\node (1) [below left= of 0,xshift=5pt] {$\Join$};
	\node (2) [below left= of 1,xshift=5pt] {$R$};
	\node (3) [below right= of 1,xshift=-5pt] {$T$};
	\node (4) [below right= of 0,xshift=-5pt] {$S$};
    \path[->]
        (0) edge (s)
        (1) edge (0)
        (2) edge (1)
        (3) edge (1)
        (4) edge (0);
	\end{tikzpicture}
\end{column}

\begin{column}{0.5\textwidth}
\begin{align*}
T(Q) & = 200K \cdot 0.01 = 20K\\
\mathit{cost}{Q} & = 20K + 253K = 273K
\end{align*}
\end{column}
\end{columns}


\vskip2em

\textbf{Option 2:} join $S$ to the \textbf{best} plan of size 4.

\vskip1em

\begin{columns}[onlytextwidth]
\begin{column}{0.3\textwidth}
$Q = $ \begin{tikzpicture}[node distance=0.25cm,every node/.style={inner sep=1,outer sep=1,font=\scriptsize}]
	\node (j) {$\Join$};
	\node (0) [below left= of j,xshift=5pt]{$\Join$};
	\node (5) [below right= of j,xshift=-5pt] {$S$};
	\node (1) [below left= of 0,xshift=5pt] {$R$};
	\node (2) [below right= of 0,xshift=-5pt] {$\sigma_{f>10}$};
	\node (3) [below of= 2, yshift=-0.35cm] {$T$};
    \path[->]
        (0) edge (j)
        (5) edge (j)
        (1) edge (0)
        (2) edge (0)
        (3) edge (2);
	\end{tikzpicture}
\end{column}

\begin{column}{0.5\textwidth}
\begin{align*}
T(Q) & = \frac{300\cdot 20K}{\max(500,2K)} = 3K\\
\mathit{cost}{Q} & = 3K + 13.33K + 20K = \alert{36.33K}
\end{align*}
\end{column}
\end{columns}

\vskip1em

\blue{And we are done!}
\end{frame}

%
% -----------------------------------------------------------------------------------------------
%
\begin{frame}{Notes}

The Sellinger optimizer can handle all operators in a SQL query, choosing the best place for each operator.

\begin{enumerate}[(1)]
\item Note that it ``figured out'' that pushing selections down the tree is the right thing to do. Although we would save more time by always pruning plans with ``high'' selections.

\item However, even with aggressive pruning, Dynamic programming is too expensive to be practical.
\begin{itemize}[-,noitemsep,topsep=-5pt]

\item Commercial optimizers often use many heuristics and avoid enumerating all plans as we did here. 
\end{itemize}

\item Machine learning can help by providing better cost estimates and also by helping solve the optimization problem.

\end{enumerate}
\end{frame}
