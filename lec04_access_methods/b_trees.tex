%!TEX root = ./lec04_access_methods.tex

\newlength{\btreeKeyWidth}
\setlength{\btreeKeyWidth}{\widthof{~~99~~}\relax}

\newlength{\btreeKeyHeight}
\setlength{\btreeKeyHeight}{0.75\btreeKeyWidth}

\newlength{\btreePointerBoxHeight}
\setlength{\btreePointerBoxHeight}{\dimexpr 0.85\btreeKeyHeight\relax}

\newlength{\btreeNodeHeight}
\setlength{\btreeNodeHeight}{\dimexpr(3pt + \btreePointerBoxHeight + \btreeKeyHeight)\relax}


%
% the widths of the key box and the pointer box are related
%
\newlength{\btreeNodeWidth}
\newlength{\btreePointerBoxWidth}


\def\btreeKeysPerNode#1{
	\pgfmathsetmacro{\NKEYS}{#1}
	\pgfmathsetmacro{\NPOINTERS}{\NKEYS + 1}
	\setlength{\btreeNodeWidth}{\dimexpr 1pt + ((\btreeKeyWidth + 1pt) * \NKEYS)}
	
	% the next one is trickier with \dimexpr :(
	\pgfmathsetlength{\btreePointerBoxWidth}{((\btreeNodeWidth - 1pt) / (\NPOINTERS))-1pt}
}
\btreeKeysPerNode{3} %default

\tikzset{
	btreeNodeBox/.style={
		anchor=north west,
		draw,rectangle,
		minimum width=\btreeNodeWidth,minimum height=\btreeNodeHeight
	}
}

\tikzset{
	btreeBlockPointerBox/.style={
		anchor=south west, 
		draw, rectangle, 
		minimum width=\btreePointerBoxWidth, minimum height=\btreePointerBoxHeight,
		fill=snow
	}
}

% #1 --> id of node where pointer originates
% #2 --> # of pointer box (0, 1, 2)
\def\dummyDataLink#1#2{
	\coordinate (dummy) at ([xshift=\dimexpr 1pt + 0.5\btreePointerBoxWidth + (1pt+\btreePointerBoxWidth) * #2,
							 yshift=\dimexpr -2.5em + -0.425\btreePointerBoxWidth]{#1.west});
	\draw [*->,>=stealth'] ([xshift=\dimexpr 1pt + 0.5\btreePointerBoxWidth + (1pt+\btreePointerBoxWidth) * #2,
							 yshift=\dimexpr -0.425\btreePointerBoxWidth]{#1.west}) to[out=270,in=90] (dummy);
}

% #1 --> node id
% #2 --> position within node (0,1,2)
% #3 --> key
\def\innerNodeKey#1#2#3{
	\node [draw,below left = 0pt and 0pt of {#1}.north west, 
			minimum width = \btreeKeyWidth, minimum height = \btreeKeyHeight,
			xshift={ 1.5pt + \btreeKeyWidth + (\dimexpr 1pt+ \btreeKeyWidth) * #2},
			yshift={ -1pt }] {#3};
}

% #1 --> node id
% #2 --> position within node (0,1,2)
% #3 --> key
\def\leafNodeKey#1#2#3{
	\node [draw,below left = 0pt and 0pt of {#1}.north west, 
			minimum width = \btreeKeyWidth, minimum height = \btreeKeyHeight,
			xshift={ 1.5pt + \btreeKeyWidth + (\dimexpr 1pt+ \btreeKeyWidth) * #2},
			yshift={ -1pt }] {#3};
	\dummyDataLink{#1}{#2};
}


% #1     --> block id
% #2, #3 --> block top-left coordinates
\def\btreeNode#1#2#3{
	\node ({#1}) at ({#2},{#3}) [anchor=north west, btreeNodeBox] {};
	%
	% draw the empty boxes for the keys
	%
	\pgfmathsetmacro{\TEMP}{\NKEYS - 1}
	\foreach \i in {0,...,\TEMP}{
		\innerNodeKey{#1}{\i}{};	
	}
	%
	% draw the data pointer boxes
	%
	\pgfmathsetmacro{\TEMP}{\NPOINTERS - 1}
	\foreach \i in {0,...,\TEMP}{
		\node [above left=0pt and 0pt of {#1}.south west, btreeBlockPointerBox,
				xshift=\dimexpr 1pt + (1pt + \btreePointerBoxWidth) * \i, 
				yshift=\dimexpr 1pt ] {};
	}
}

% #1 --> id of node where pointer originates
% #2 --> # of pointer box (0, 1, 2)
% #3 --> id of target node/object
\def\parentChildLink#1#2#3{
	\draw [*->,>=stealth'] ([xshift=\dimexpr 1pt + 0.5\btreePointerBoxWidth + (1pt+\btreePointerBoxWidth) * #2,
							 yshift=\dimexpr -0.425\btreePointerBoxWidth]{#1.west}) to[out=270,in=90] (#3);
}

% #1 --> id of node where pointer originates
% #2 --> # of pointer box (0, 1, 2)
% #3 --> id of target node/object
\def\LABELparentChildLink#1#2#3#4{
	\draw [*->,>=stealth'] ([xshift=\dimexpr 1pt + 0.5\btreePointerBoxWidth + (1pt+\btreePointerBoxWidth) * #2,
							 yshift=\dimexpr -0.425\btreePointerBoxWidth]{#1.west}) to[out=270,in=90] 
							 node [fill,color=background,midway] {\textcolor{red}{#4}} (#3); 
}


% #1 --> id of node where pointer originates
% #2 --> id of target node/object
\def\chainLink#1#2{
	\draw [*->,>=stealth'] ([xshift=-0.725\btreePointerBoxWidth,
							 yshift=-0.5725\btreePointerBoxWidth]{#1.east}) to[out=0,in=180] (#2);
}

\newsavebox\IIIlevelBtreeExample
\savebox{\IIIlevelBtreeExample}{
\begin{tikzpicture}
% root node
\btreeNode{root}{0}{0};
\innerNodeKey{root}{0}{17};
% left child of root
\btreeNode{inner0}{-3}{-2.5};
\innerNodeKey{inner0}{0}{7};
% right child of root
\btreeNode{inner1}{3}{-2.5};
\innerNodeKey{inner1}{0}{29};\innerNodeKey{inner1}{1}{41};
%
% leaf nodes from left to right
%
\btreeNode{leaf0}{-6}{-5};
\leafNodeKey{leaf0}{0}{3};\leafNodeKey{leaf0}{1}{5};

\btreeNode{leaf1}{-3}{-5};
\leafNodeKey{leaf1}{0}{7};\leafNodeKey{leaf1}{1}{11};\leafNodeKey{leaf1}{2}{13};

\btreeNode{leaf2}{0}{-5};
\leafNodeKey{leaf2}{0}{17};\leafNodeKey{leaf2}{1}{23};

\btreeNode{leaf3}{3}{-5};
\leafNodeKey{leaf3}{0}{29};\leafNodeKey{leaf3}{1}{31};\leafNodeKey{leaf3}{2}{37};

\btreeNode{leaf4}{6}{-5};
\leafNodeKey{leaf4}{0}{41};\leafNodeKey{leaf4}{1}{43};

% chain at leaf level
\chainLink{leaf0}{leaf1};\chainLink{leaf1}{leaf2};
\chainLink{leaf2}{leaf3};\chainLink{leaf3}{leaf4};

% links between inner nodes
\parentChildLink{root}{0}{inner0};\parentChildLink{root}{1}{inner1};

\parentChildLink{inner0}{0}{leaf0};\parentChildLink{inner0}{1}{leaf1};

\parentChildLink{inner1}{0}{leaf2};\parentChildLink{inner1}{1}{leaf3};\parentChildLink{inner1}{2}{leaf4};
\end{tikzpicture}}

%
% ---------------------------------------------------------------------------------------
%
\begin{frame}{B+ tree}

The \textbf{B+ tree} is a self-balancing search tree designed for secondary memory: \alert{each node of the tree is stored on a separate disk block}.

\begin{center}
\scalebox{0.75}{\usebox{\IIIlevelBtreeExample}}
\end{center}
\end{frame}

%
% ---------------------------------------------------------------------------------------
%
\begin{frame}

\begin{center}
\framebox{\alert{Every node holds $n$ keys and $n+1$ pointers}}
\end{center}

\vskip1em

In a leaf node:\\
 - the first $n$ pointers point to \underline{actual tuples} in the data file\\
 - the last pointer is used to form a \textbf{chain} of leaf nodes (useful for range queries)

\vskip2em

\begin{center}
\begin{tikzpicture}
\node at (0,0) [anchor=north west] {\scalebox{0.5}{\clipbox{0pt 0pt 0pt 110pt}{\usebox\IIIlevelBtreeExample}}};

\pause

\node (0) at (1.2,-1.15) [draw, color=red, rectangle, minimum width=1em, minimum height=1em] {};
\node (1) at (2.7,-1.15) [draw, color=red, rectangle, minimum width=1em, minimum height=1em] {};
\node (2) at (4.2,-1.15) [draw, color=red, rectangle, minimum width=1em, minimum height=1em] {};
\node (3) at (5.7,-1.15) [draw, color=red, rectangle, minimum width=1em, minimum height=1em] {};

\node (4) at (3,0.5) [color=red] {\small chain};
\foreach \i in {0,...,3}{
	\draw [->,color=red] (4) -- (\i);
}

\pause

\node (5) at (0.5,-1.15) [draw, color=blue, rectangle, minimum width=2em, minimum height=1em] {};
\node (6) at (2,-1.15) [draw, color=blue, rectangle, minimum width=2em, minimum height=1em] {};
\node (7) at (3.5,-1.15) [draw, color=blue, rectangle, minimum width=2em, minimum height=1em] {};
\node (8) at (5.1,-1.15) [draw, color=blue, rectangle, minimum width=2.5em, minimum height=1em] {};
\node (9) at (6.5,-1.15) [draw, color=blue, rectangle, minimum width=2.2em, minimum height=1em] {};

\node (10) at (3, -2.5) [color=blue] {to data file};
\foreach \i in {5,...,9}{
	\draw [->,color=blue] (10) -- (\i);
}
\end{tikzpicture}
\end{center}
\end{frame}

%
% ---------------------------------------------------------------------------------------
%
\begin{frame}
\begin{center}
\framebox{\alert{Every node holds $n$ keys and $n+1$ pointers}}
\end{center}

\vskip1em

In an inner node with $i$ keys the first $i+1$ pointers are used.

\vskip2em

\begin{columns}[onlytextwidth]
\begin{column}{0.55\textwidth}
A \textbf{search} for key $j$ follows pointer:
\end{column}
\begin{column}{0.5\textwidth}
\[
\begin{cases}
0 & \text{if~} j < k_0\\
i & \text{if~} k_{i-1} \leq j < k_i\\
i+1 & \text{if~} j \geq k_i
\end{cases}
\]
\end{column}
\end{columns}

\vskip2em

\begin{center}
\begin{tikzpicture}
\node at (0,0) [anchor=north west] {\scalebox{0.5}{\clipbox{165pt 0pt 0pt 50pt}{\usebox\IIIlevelBtreeExample}}};
\pause

\node (0) at (-0.75,-1) [color=red] {\footnotesize to keys $j<29$};
\draw [->,color=red] (0) -- (1,-1.5);
\pause

\node (1) at (0,0.125) [color=orange] {\footnotesize to keys $29\leq j<41$};
\draw [->,color=orange] (1) -- (2.3,-1.5);
\pause

\node (2) at (5,-1) [color=blue] {\footnotesize to keys $j\geq 41$};
\draw [->,color=blue] (2) -- (3.75,-1.5);
\end{tikzpicture}
\end{center}
\end{frame}


%
% ---------------------------------------------------------------------------------------
%
\begin{frame}

\begin{center}
\framebox{\alert{Every node holds $n$ keys and $n+1$ pointers}}
\end{center}

\vskip1em

To avoid nodes becoming ``too empty'', the minimum \alert{occupancy} of a node should be:
 
 - \textbf{inner node}: \(\ceil*{\frac{n+1}{2}}\) pointers
 
 - \textbf{leaf node}: \(\floor*{\frac{n+1}{2}} + 1\) pointers (to data file)

\vskip1em

If the node occupancy goes below that (due to deletions), the node should be merged with adjacent ones\footnote{In practice, however, most DBMSs skip this and reclaim wasted space during re-indexing maintenance processes.}
\end{frame}

%
% ---------------------------------------------------------------------------------------
%
\begin{frame}

\textbf{Example:} occupancy with \qquad \framebox{$n=3$}

\vskip2em

\begin{center}
\begin{tikzpicture}

\btreeNode{innerFull}{0}{0};
\innerNodeKey{innerFull}{0}{29};\innerNodeKey{innerFull}{1}{34};\innerNodeKey{innerFull}{2}{41};
\coordinate (if0) at (-0.15,-1.75);\coordinate (if1) at (1,-1.75);
\coordinate (if2) at (1.9,-1.75);\coordinate (if3) at (2.7,-1.75);
\parentChildLink{innerFull}{0}{if0};\parentChildLink{innerFull}{1}{if1};
\parentChildLink{innerFull}{2}{if2};\parentChildLink{innerFull}{3}{if3};

\btreeNode{innerMin}{4}{0};
\innerNodeKey{innerMin}{0}{17};
\coordinate (im0) at (3.85,-1.75);\coordinate (im1) at (5.25,-1.75);
\parentChildLink{innerMin}{0}{im0};\parentChildLink{innerMin}{1}{im1};

\btreeNode{leafFull}{0}{-3};
\leafNodeKey{leafFull}{0}{29};\leafNodeKey{leafFull}{1}{31};\leafNodeKey{leafFull}{2}{37};
\coordinate (lf0) at (2.9,-3.5);\chainLink{leafFull}{lf0};

\btreeNode{leafMin}{4}{-3};
\leafNodeKey{leafMin}{0}{39};
\leafNodeKey{leafMin}{1}{43};
\coordinate (lm0) at (6.9,-3.5);\chainLink{leafMin}{lm0};

\node (0) [left= 2em of innerFull] {Inner : };
\node (1) [left= 2em of leafFull] {Leaf : };
\node (2) [above= 2em of innerFull] {Full node};
\node (3) [above= 2em of innerMin] {Minimum occupancy};

\node (4) at (6,-5)  [color=red] {\begin{minipage}{2cm}\scriptsize\baselineskip=0.75\baselineskip plus 1fill\centering
 ``chain'' pointers count EVEN IF THEY ARE SET TO NULL\end{minipage}};
\draw[->,color=red] (4) -- (6.25,-4);
\end{tikzpicture}
\end{center}
\end{frame}


%
% ---------------------------------------------------------------------------------------
%
\begin{frame}{B+ trees are balanced}

\begin{center}
\begin{tikzpicture}
\node (tree) {\scalebox{0.5}{\usebox{\IIIlevelBtreeExample}}};
\node (0) [left=2em of tree.north west,yshift=-10pt] {level 0};
\node (1) [below=2em of 0,yshift=-2pt] {level 1};
\node (2) [below=2em of 1,yshift=-1pt] {level 2};
\end{tikzpicture}
\end{center}

The \alert{level of a node} is its distance (in edges) from the root.

The \alert{height of the tree} is the maximum level of any node.

\vskip1em

In a B+ tree \textbf{all leaves are at the same level.}

\end{frame}

%
% ---------------------------------------------------------------------------------------
%
\begin{frame}{Root nodes are special}

Often, real B+ tree nodes (i.e., disk blocks) hold a few hundred pointers. Thus, until there are more than that many keys in the index, the ``entire tree'' is just one (root \emph{and} also leaf) node.

\vskip1em

\begin{center}
\small
\begin{tabular}{c||c|c||c|c}
& \multicolumn{2}{c||}{\textbf{minimum}} & \multicolumn{2}{c}{\textbf{maximum}}\\
\cline{2-5}
& pointers & keys & pointers & keys \\
\hline
Inner & \multirow{2}{*}{$\ceil*{\frac{(n+1)}{2}}$} & \multirow{2}{*}{$\ceil*{\frac{(n+1)}{2}} - 1$} & \multirow{2}{*}{$n+1$} & \multirow{2}{*}{$n$} \\
(non-root) & & & & \\
\hline
Leaf & \multirow{2}{*}{$\floor*{\frac{(n+1)}{2}} + 1$} & \multirow{2}{*}{$\floor*{\frac{(n+1)}{2}}$} & \multirow{2}{*}{$n+1$} & \multirow{2}{*}{$n$} \\
(non-root) & & & & \\
\hline
Root & \alert{2} & 1 & $n+1$ & $n$\\
\end{tabular}
\end{center}

\textbf{\alert{NOTE}} when the tree is just the root node the minimum number of pointers is 1.
\end{frame}

%
% ---------------------------------------------------------------------------------------
%
\begin{frame}{Inserting a new key into a B+ tree}

B+ tree insertions are \underline{very similar} insertions into binary search trees: (1) find where the new key should go; (2) add the new key; (3) modify other nodes as needed.

\small
\begin{enumerate}[label=(\arabic*)]
\item \textbf{Search} for key \alert{$K$} in the tree; that is, traverse the tree from the root until we locate the leaf node \alert{$l_i$} where \alert{$K$} should go.

\item If there is space in \alert{$l_i$}, add key \alert{$K$}; \highlight{STOP}. Otherwise:
\begin{enumerate}[label=(\alph*)]
\item create a new leaf \alert{$l_{\text{new}}$} with \alert{$K$};
\item divide the keys in \alert{$l_i$} among \alert{$l_i$} and \alert{$l_{\text{new}}$} so that both satisfy the minimum occupancy requirement; 
% \item if \alert{$K$} ``splits'' the keys in \alert{$l_i$}, move the \textbf{fewest number of keys} from \alert{$l_i$} to \alert{$l_{\text{new}}$}; \underline{otherwise}, leave \alert{$K$} alone in \alert{$l_{\text{new}}$};
\end{enumerate}
\item link \alert{$l_{\text{new}}$} to \alert{$l_i$} and to its \underline{predecessor} or \underline{successor}; link \alert{$l_{\text{new}}$} to the parent of \alert{$l_i$} \textbf{recursively} inserting a new key in the parent.
\item If needed, recursively add and link parents nodes.
\end{enumerate}
\end{frame}

\normalsize

%
% ---------------------------------------------------------------------------------------
%
\begin{frame}
\textbf{Example}: insertion when there is room in \alert{$l_i$}

\vskip1em

\begin{center}
Insert key \alert{\textbf{$K=6$}} \qquad \framebox{$n=3$}

\vskip1em

\scalebox{0.6}{\begin{tikzpicture}
\btreeNode{root}{0}{0};
\innerNodeKey{root}{0}{17};
\btreeNode{inner0}{-3}{-2.5};
\innerNodeKey{inner0}{0}{7};
\coordinate (inner1) at (3,-2.5);
\btreeNode{leaf0}{-6}{-5};
\leafNodeKey{leaf0}{0}{3};\leafNodeKey{leaf0}{1}{5};
\btreeNode{leaf1}{-3}{-5};
\leafNodeKey{leaf1}{0}{7};\leafNodeKey{leaf1}{1}{11};\leafNodeKey{leaf1}{2}{13};
\chainLink{leaf0}{leaf1};\chainLink{leaf1}{leaf2};
\parentChildLink{root}{0}{inner0};\parentChildLink{root}{1}{inner1};
\parentChildLink{inner0}{0}{leaf0};\parentChildLink{inner0}{1}{leaf1};

%before update
\only<1|handout:1>{
	\LABELparentChildLink{root}{0}{inner0}{$K < 17$};
	\LABELparentChildLink{inner0}{0}{leaf0}{$K < 7$};
	\node (li) at (-6,-4) {\alert{$l_i$}};
	\draw [->,thick,color=accent] (li) -- (leaf0);
}

%after update
\only<2|handout:1>{
	\leafNodeKey{leaf0}{2}{\alert{\textbf{6}}};
}
\end{tikzpicture}}
\end{center}
\end{frame}

%
% ---------------------------------------------------------------------------------------
%
\begin{frame}

\textbf{Example}: insertion causing a node split ``to the left'' of \alert{$l_i$}

\vskip1em

\begin{center}
Insert key \alert{$K=4$} \qquad \framebox{$n=3$}
\vskip1em

\scalebox{0.6}{\begin{tikzpicture}
\btreeNode{root}{0}{0};
\innerNodeKey{root}{0}{17};
\btreeNode{inner0}{-3.75}{-2.5};
\coordinate (inner1) at (3,-2.5);
\parentChildLink{root}{0}{inner0};\parentChildLink{root}{1}{inner1};
\btreeNode{leaf0}{-6}{-5};
\btreeNode{leaf1}{-3}{-5};\leafNodeKey{leaf1}{0}{7};
\leafNodeKey{leaf1}{1}{11};\leafNodeKey{leaf1}{2}{13};
\chainLink{leaf0}{leaf1};\chainLink{leaf1}{leaf2};

% before update
\only<1-2|handout:0>{
	\parentChildLink{inner0}{0}{leaf0};\parentChildLink{inner0}{1}{leaf1};
	\innerNodeKey{inner0}{0}{7};
}

\only<1|handout:0>{
	\leafNodeKey{leaf0}{0}{3};\leafNodeKey{leaf0}{1}{5};\leafNodeKey{leaf0}{2}{6};
	\LABELparentChildLink{root}{0}{inner0}{$K < 17$};
	\LABELparentChildLink{inner0}{0}{leaf0}{$K < 7$};
}

\only<1-3|handout:1>{
	\node (li) at (-6.5,-3.5) {\large\alert{$l_i$}};
	\draw [->,thick,color=accent] (li) -- (leaf0);
}

% after update
\only<2-3|handout:1>{
	\btreeNode{newLeaf}{-9}{-5};\leafNodeKey{newLeaf}{0}{3};\leafNodeKey{newLeaf}{1}{\alert{\textbf{4}}};
	\leafNodeKey{leaf0}{0}{5};\leafNodeKey{leaf0}{1}{6};
	\chainLink{newLeaf}{leaf0};
}

\only<2-3|handout:1>{
	\node (lnew) at (-9,-3.5) {\large\alert{$l_\text{new}$}};
	\draw [->,thick,color=accent] (lnew) -- (newLeaf);
	\node (0) at (-6,-1.5) [color=blue] {update parent};
	\draw [->,thick,color=blue] (0) -- (inner0);	
}

% reveal the smallest key in the inner in the last step
\only<3|handout:1>{
	\innerNodeKey{inner0}{0}{\textcolor{red}{5}};\innerNodeKey{inner0}{1}{7};
	\parentChildLink{inner0}{0}{newLeaf};\parentChildLink{inner0}{1}{leaf0};\parentChildLink{inner0}{2}{leaf1};
}
\end{tikzpicture}}
\end{center}
\end{frame}


%
% ---------------------------------------------------------------------------------------
%
\begin{frame}

\textbf{Example}: insertion causing a node split ``to the right'' of \alert{$l_i$}

\vskip1em 

\begin{center}

Insert key \alert{$K=32$}; \only<4|handout:2>{and then insert key \alert{$K=47$};} \qquad \framebox{$n=3$}

\vskip1em

\scalebox{0.6}{\begin{tikzpicture}
\btreeNode{root}{1.5}{0};
\innerNodeKey{root}{0}{17};
\coordinate (inner0) at (1,-2.5);
\btreeNode{inner1}{4.5}{-2.5};
\innerNodeKey{inner1}{0}{29};
\parentChildLink{root}{0}{inner0};\parentChildLink{root}{1}{inner1};
\btreeNode{leaf2}{0}{-5};
\leafNodeKey{leaf2}{0}{17};\leafNodeKey{leaf2}{1}{23};
\btreeNode{leaf3}{3}{-5};
\leafNodeKey{leaf3}{0}{29};\leafNodeKey{leaf3}{1}{31};
\btreeNode{leaf4}{9}{-5};
\leafNodeKey{leaf4}{0}{41};\leafNodeKey{leaf4}{1}{43};
\parentChildLink{inner1}{0}{leaf2};
\chainLink{leaf2}{leaf3};\parentChildLink{inner1}{1}{leaf3};

% before update
\only<1|handout:1>{
	\leafNodeKey{leaf3}{2}{37};
	\chainLink{leaf3}{leaf4};
\	\LABELparentChildLink{root}{1}{inner1}{$K > 17$};
	\LABELparentChildLink{inner1}{1}{leaf3}{$29 \leq K < 41$};
};

\only<1-2|handout:1>{
	\innerNodeKey{inner1}{1}{41};
	\parentChildLink{inner1}{2}{leaf4};
	\node (li) [above left=3em and 2em of leaf3] {\alert{$l_i$}};
	\draw [->,thick,color=accent] (li) -- (leaf3);
}

\only<2-4|handout:2>{
	\btreeNode{newLeaf}{6}{-5};
	\leafNodeKey{newLeaf}{0}{\alert{32}};\leafNodeKey{newLeaf}{1}{37};
	\chainLink{leaf3}{newLeaf};\chainLink{newLeaf}{leaf4};
};

\only<2|handout:2>{
	\node (0) [above right=1em and 2em of inner1,color = blue] {update parent};
	\draw [->,thick,color=blue] (0) -- (inner1);
	\node (lnew) [above right=3em and 2em of newLeaf] {\alert{$l_\text{new}$}};
	\draw [->,thick,color=accent] (lnew) -- (newLeaf);
}

\only<3-4|handout:2>{
	\innerNodeKey{inner1}{1}{\textcolor{red}{32}};\innerNodeKey{inner1}{2}{\textcolor{red}{41}};
	\parentChildLink{inner1}{2}{newLeaf};\parentChildLink{inner1}{3}{leaf4};
}
% after update with 47
\only<4|handout:2>{\leafNodeKey{leaf4}{2}{\alert{47}};};
\end{tikzpicture}}
\end{center}

\end{frame}

%
% ---------------------------------------------------------------------------------------
%
\begin{frame}
Insertion leading to \textbf{adding a new parent}:

\vskip1em

\begin{center}

Insert key \alert{$K=53$} \qquad \framebox{$n=3$}

\vskip1em

\scalebox{0.6}{\begin{tikzpicture}
\btreeNode{root}{3.5}{0};
\innerNodeKey{root}{0}{17};
\coordinate (inner0) at (1,-2.5);
\btreeNode{inner1}{4.5}{-2.5};
\innerNodeKey{inner1}{0}{29};\innerNodeKey{inner1}{1}{32};

\btreeNode{leaf2}{0}{-5};
\leafNodeKey{leaf2}{0}{17};\leafNodeKey{leaf2}{1}{23};

\btreeNode{leaf3}{3}{-5};
\leafNodeKey{leaf3}{0}{29};\leafNodeKey{leaf3}{1}{31};

\btreeNode{leaf4}{6}{-5};
\leafNodeKey{leaf4}{0}{32};\leafNodeKey{leaf4}{1}{37};

\btreeNode{leaf5}{9}{-5};
\leafNodeKey{leaf5}{0}{41};\leafNodeKey{leaf5}{1}{43};

\parentChildLink{root}{0}{inner0};\parentChildLink{root}{1}{inner1};
\parentChildLink{inner1}{0}{leaf2};\parentChildLink{inner1}{1}{leaf3};
\parentChildLink{inner1}{2}{leaf4};

\chainLink{leaf2}{leaf3};\chainLink{leaf3}{leaf4};\chainLink{leaf4}{leaf5};

% before update
\only<1-3|handout:1>{
	\innerNodeKey{inner1}{2}{41};\parentChildLink{inner1}{3}{leaf5};
	\node (li) [above right= 3em and 0.025em of leaf5] {\alert{$l_i$}};
	\draw [->,thick,color=accent] (li) -- (leaf5);
}

\only<1|handout:0>{
	\leafNodeKey{leaf5}{2}{47};
}

\only<1|handout:1>{
	\LABELparentChildLink{root}{1}{inner1}{$K > 17$};
	\LABELparentChildLink{inner1}{3}{leaf5}{$K > 41$};
}

\only<2|handout:1>{
	\node (0) [above right=1em and 2em of inner1,color = blue] {update parent?};
	\draw [->,thick,color=blue] (0) -- (inner1);
}

\only<2-5|handout:1-2>{
	\btreeNode{newLeaf}{12}{-5};
	\leafNodeKey{newLeaf}{0}{\textcolor{red}{47}};
	\leafNodeKey{newLeaf}{1}{\alert{53}};
	\chainLink{leaf5}{newLeaf};
}

\only<2-3|handout:1>{
	\node (lnew) [above right= 3em and 0.025em of newLeaf] {\alert{$l_\text{new}$}};
	\draw [->,thick,color=accent] (lnew) -- (newLeaf);
}

\only<3-5|handout:2>{
	\btreeNode{newInner}{8.5}{-2.5};
	\innerNodeKey{newInner}{0}{\textcolor{red}{47}};
	\parentChildLink{newInner}{1}{newLeaf};
}

\only<4-5|handout:2>{
	\parentChildLink{newInner}{0}{leaf5};
}

\only<5|handout:2>{
	\innerNodeKey{root}{1}{\textcolor{red}{41}};
	\parentChildLink{root}{2}{newInner};
}
\end{tikzpicture}}
\end{center}
\end{frame}

%
% ---------------------------------------------------------------------------------------
%
\begin{frame}
Insertion leading to \textbf{adding a new root}:

\begin{center}

Insert key \alert{$K=2$} \qquad \framebox{$n=3$}

\vskip1em

\scalebox{0.6}{\begin{tikzpicture}
\btreeNode{oldRoot}{0}{0};
\btreeNode{leaf0}{-4.5}{-3};
\btreeNode{leaf1}{-1.5}{-3};
\btreeNode{leaf2}{1.5}{-3};
\btreeNode{leaf3}{4.5}{-3};
\chainLink{leaf0}{leaf1};\chainLink{leaf1}{leaf2};\chainLink{leaf2}{leaf3};

\leafNodeKey{leaf1}{0}{11};\leafNodeKey{leaf1}{1}{13};
\leafNodeKey{leaf2}{0}{17};\leafNodeKey{leaf2}{1}{23};
\leafNodeKey{leaf3}{0}{29};\leafNodeKey{leaf3}{1}{31};\leafNodeKey{leaf3}{2}{37};

\only<1-2|handout:1>{
	\parentChildLink{oldRoot}{0}{leaf0};\parentChildLink{oldRoot}{1}{leaf1};
	\parentChildLink{oldRoot}{2}{leaf2};\parentChildLink{oldRoot}{3}{leaf3};
	\innerNodeKey{oldRoot}{0}{11};\innerNodeKey{oldRoot}{1}{17};\innerNodeKey{oldRoot}{2}{29};
}

\only<1|handout:1>{
	\LABELparentChildLink{oldRoot}{0}{leaf0}{$K<11$};
	\leafNodeKey{leaf0}{0}{3};\leafNodeKey{leaf0}{1}{5};\leafNodeKey{leaf0}{2}{7};
}

\only<2-4|handout:2>{
	\btreeNode{newLeaf}{-7.5}{-3};
	\leafNodeKey{newLeaf}{0}{\alert{2}};
	\leafNodeKey{newLeaf}{1}{\textcolor{red}{3}};\leafNodeKey{leaf0}{0}{5};\leafNodeKey{leaf0}{1}{7};
	\chainLink{newLeaf}{leaf0};
}

\only<1-2|handout:1-2>{
	\node (li) [above left=3em and 2em of leaf0] {\alert{$l_i$}};
	\draw [->,thick,color=accent] (li) -- (leaf0);
}

\only<2|handout:2>{
	\node (lnew) [above left=3em and 2em of newLeaf] {\alert{$l_\text{new}$}};
	\draw [->,thick,color=accent] (lnew) -- (newLeaf);
}

\only<3-4|handout:2>{
	\btreeNode{newInner}{-5}{0};
	\innerNodeKey{newInner}{0}{\textcolor{red}{5}};
	\innerNodeKey{oldRoot}{0}{\textcolor{red}{17}};\innerNodeKey{oldRoot}{1}{\textcolor{red}{29}};
	\parentChildLink{newInner}{0}{newLeaf};\parentChildLink{newInner}{1}{leaf0};
	\parentChildLink{oldRoot}{0}{leaf1};\parentChildLink{oldRoot}{1}{leaf2};
	\parentChildLink{oldRoot}{2}{leaf3};
	
}

\only<4|handout:2>{
	\btreeNode{newRoot}{-2.5}{3};	
	\innerNodeKey{newRoot}{0}{\textcolor{red}{11}};
	\parentChildLink{newRoot}{0}{newInner};\parentChildLink{newRoot}{1}{oldRoot};

}
\end{tikzpicture}}
\end{center}
\end{frame}

%
% ---------------------------------------------------------------------------------------
%
\begin{frame}{B+tree Insertions: Wrap Up}

\textbf{Insertions have low cost. }

In most systems, nodes can hold a few hundred keys, and after the tree grows to a reasonable size (usually 2 or 3 levels), only a small fraction of insertions require adding/modifying inner nodes.

\vskip1em

\textbf{The tree always grows ``upwards'', by adding \emph{parents}.}

The worst case scenario of inserting into a tree $T$ is bound by $O(\mathit{height}(T))$, which is practically a constant (compared to the number of keys--i.e., tuples in the database).
\end{frame}

%
% for the following examples we use nKyes = 4
%
\btreeKeysPerNode{4}

%
% ---------------------------------------------------------------------------------------
%
\begin{frame}{Deletions from B+ trees}

When deleting a key from a B+ tree the main concern is ensuring the node occupancy \textbf{does not} go below the minimum allowed.\footnote{Recall that the more keys in a node (disk block) the \emph{higher} the I/O efficiency.}

If a deletion does not make the node occupancy too low, there is nothing further to do.

Dealing with a node with low occupancy (leaf or inner):\\
- \textbf{coalesce} with a neighboring node\\
- \textbf{move a key} from a neighboring node

\end{frame}

%
% ---------------------------------------------------------------------------------------
%
\begin{frame}

\textbf{Coalescing} with neighbor node.

\vskip1em

\begin{center}

Delete key \alert{$K=23$} \qquad \framebox{$n=4$; min occupancy is 2 keys}

\vskip1em

\scalebox{0.6}{\begin{tikzpicture}

\btreeNode{root}{0}0;
\btreeNode{leaf0}{-6}{-3};
\btreeNode{leaf1}{-2}{-3};
\btreeNode{leaf3}{6}{-3};
\chainLink{leaf0}{leaf1};

\innerNodeKey{root}{0}{7};

\leafNodeKey{leaf0}{0}{3};\leafNodeKey{leaf0}{1}{5};
\leafNodeKey{leaf1}{0}{7};\leafNodeKey{leaf1}{1}{11};
\leafNodeKey{leaf1}{2}{13};\leafNodeKey{leaf1}{3}{17};

\parentChildLink{root}{0}{leaf0};\parentChildLink{root}{1}{leaf1};

\only<1|handout:1>{
	\btreeNode{leaf2}{2}{-3};
	\leafNodeKey{leaf2}{0}{23};\leafNodeKey{leaf2}{1}{29};
	\innerNodeKey{root}{1}{23};\innerNodeKey{root}{2}{31};
	\leafNodeKey{leaf3}{0}{31};\leafNodeKey{leaf3}{1}{37};\leafNodeKey{leaf3}{2}{41};
	\parentChildLink{root}{3}{leaf3};
	\chainLink{leaf1}{leaf2};\chainLink{leaf2}{leaf3};

	\LABELparentChildLink{root}{2}{leaf2}{$23 \leq K < 31 $};

	\node (0) at (3.175,-3.35) [draw,very thick,dotted,circle,color=blue,minimum width=2.5em] {};
	\draw [->,very thick,color=blue,dotted] (0) to[out=45,in=135] (leaf3);
}

\only<2|handout:2>{
	\innerNodeKey{root}{1}{\textcolor{red}{29}};
	\leafNodeKey{leaf3}{0}{\textcolor{red}{29}};\leafNodeKey{leaf3}{1}{31};
	\leafNodeKey{leaf3}{2}{37};\leafNodeKey{leaf3}{3}{41};
	\parentChildLink{root}{2}{leaf3};
	\chainLink{leaf1}{leaf3};
}
\end{tikzpicture}}
\end{center}
\end{frame}

%
% ---------------------------------------------------------------------------------------
%
\begin{frame}

\textbf{Moving a key} from a neighbor node.

\vskip1em

\begin{center}

Delete key \alert{$K=23$} \qquad \framebox{$n=4$; min occupancy is 2 keys}

\vskip1em

\scalebox{0.6}{\begin{tikzpicture}

\btreeNode{root}{0}{0};
\btreeNode{leaf0}{-6}{-3};
\btreeNode{leaf1}{-2}{-3};
\btreeNode{leaf2}{2}{-3};
\btreeNode{leaf3}{6}{-3};
\chainLink{leaf0}{leaf1};\chainLink{leaf1}{leaf2};\chainLink{leaf2}{leaf3};

\innerNodeKey{root}{0}{7};\innerNodeKey{root}{2}{31};
\leafNodeKey{leaf0}{0}{3};\leafNodeKey{leaf0}{1}{5};
\leafNodeKey{leaf1}{0}{7};\leafNodeKey{leaf1}{1}{11};\leafNodeKey{leaf1}{2}{13};
\leafNodeKey{leaf2}{1}{29};
\leafNodeKey{leaf3}{0}{31};\leafNodeKey{leaf3}{1}{37};\leafNodeKey{leaf3}{2}{41};

\parentChildLink{root}{0}{leaf0};\parentChildLink{root}{1}{leaf1};
\parentChildLink{root}{2}{leaf2};\parentChildLink{root}{3}{leaf3};

\only<1|handout:1>{
	\leafNodeKey{leaf2}{0}{23};
	\leafNodeKey{leaf1}{3}{17};
	\innerNodeKey{root}{1}{23};
	
	\LABELparentChildLink{root}{2}{leaf2}{$23 \leq K < 31 $};
	\node (0) at (0.725,-3.35) [draw,very thick,dotted,circle,color=blue,minimum width=2.5em] {};
	\draw [->,very thick,color=blue,dotted] (0) to[out=45,in=135] (leaf2);
	\node (1) at (6.4,-3.35) [draw,very thick,dotted,circle,color=blue,minimum width=2.5em] {};
	\draw [->,very thick,color=blue,dotted] (1) to[out=135,in=45] (leaf2);
}

\only<2|handout:2>{
	\leafNodeKey{leaf2}{0}{\textcolor{red}{17}};
	\innerNodeKey{root}{1}{\textcolor{red}{17}};
}
\end{tikzpicture}}
\end{center}
\end{frame}

%
% ---------------------------------------------------------------------------------------
%
\begin{frame}

\textbf{Coalescing} inner nodes.

\vskip1em

\begin{center}

Delete key \alert{$K=41$} \qquad \framebox{$n=4$; min occupancy is 2 keys}

\vskip1em

\scalebox{0.5}{\begin{tikzpicture}

\btreeNode{inner0}{-4}{-3.5};\innerNodeKey{inner0}{0}{7};\innerNodeKey{inner0}{1}{13};
\btreeNode{leaf0}{-10}{-7};\btreeNode{leaf1}{-6}{-7};\btreeNode{leaf2}{-2}{-7};
\parentChildLink{inner0}{0}{leaf0};\parentChildLink{inner0}{1}{leaf1};\parentChildLink{inner0}{2}{leaf2};

\leafNodeKey{leaf0}{0}{3};\leafNodeKey{leaf0}{1}{5};
\leafNodeKey{leaf1}{0}{7};\leafNodeKey{leaf1}{1}{11};
\leafNodeKey{leaf2}{0}{13};\leafNodeKey{leaf2}{1}{17};

\btreeNode{leaf3}{2}{-7};\btreeNode{leaf5}{10}{-7};
\leafNodeKey{leaf3}{0}{23};\leafNodeKey{leaf3}{1}{29};\leafNodeKey{leaf3}{2}{31};
\leafNodeKey{leaf5}{0}{47};\leafNodeKey{leaf5}{1}{53};\leafNodeKey{leaf5}{2}{59};

\chainLink{leaf0}{leaf1};\chainLink{leaf1}{leaf2};\chainLink{leaf2}{leaf3};

\only<1-3|handout:1>{
	\btreeNode{root}{0}{0};\innerNodeKey{root}{0}{23};
	\parentChildLink{root}{0}{inner0};
	\btreeNode{inner1}{4}{-3.5};\parentChildLink{root}{1}{inner1};
	\parentChildLink{inner1}{0}{leaf3};

}

\only<1|handout:1>{
	\btreeNode{leaf4}{6}{-7};
	\leafNodeKey{leaf4}{0}{41};\leafNodeKey{leaf4}{1}{43};

	\parentChildLink{inner1}{2}{leaf5};

	\chainLink{leaf3}{leaf4};\chainLink{leaf4}{leaf5};

	\innerNodeKey{inner1}{0}{41};\innerNodeKey{inner1}{1}{47};

	\LABELparentChildLink{root}{1}{inner1}{$K>23$};	
	\LABELparentChildLink{inner1}{1}{leaf4}{$41 \leq K <47$};

	\node (0) at (7.2,-7.3) [draw,very thick,dotted,circle,color=blue,minimum width=2.5em] {};
	\draw [->,very thick,color=blue,dotted] (0) to[out=135,in=45] (leaf3);
	\draw [->,very thick,color=blue,dotted] (0) to[in=135,out=45] (leaf5);
}

\only<2-4|handout:2>{
	\leafNodeKey{leaf3}{3}{\textcolor{red}{43}};
	\chainLink{leaf3}{leaf5};
}

\only<2-3|handout:1>{
	\node (1) at (4.45,-3.75) [draw,very thick,dotted,circle,color=blue,minimum width=2.5em] {};
	\draw [->,very thick,color=blue,dotted] (1) to[out=135,in=60] (inner0);
}

\only<2-3|handout:0>{
	\parentChildLink{inner1}{1}{leaf5};
	\innerNodeKey{inner1}{0}{47};
}

\only<3|handout:1>{
	\node (3) at (2.42,-7.3) [draw,very thick,dotted,circle,color=blue,minimum width=2.5em] {};
	\draw [->,very thick,color=blue,dotted] (3) to[out=135,in=315] (inner0);

	\node (4) at (10.45,-7.3) [draw,very thick,dotted,circle,color=blue,minimum width=2.5em] {};
	\draw [->,very thick,color=blue,dotted] (4) to[out=150,in=315] ([xshift=-5pt]inner0.south east);
}

\only<3|handout:0>{
	\draw [-,very thick,dotted,color=red] (-0.25,0.25) -- (3.5,-1.5);
	\draw [-,very thick,dotted,color=red] (-0.25,-1.5) -- (3.5,0.25);

	\draw [-,very thick,dotted,color=red] (3.75,-4.95) -- (7.5,-3.25);
	\draw [-,very thick,dotted,color=red] (3.75,-3.25) -- (7.5,-4.95);
}

\only<4|handout:2>{
	\parentChildLink{inner0}{3}{leaf3};

	% copy the command this once to adjust the angles and avoid the lines crossing :(
	\draw [*->,>=stealth'] ([xshift=\dimexpr -1pt + 0.5\btreePointerBoxWidth + (1pt+\btreePointerBoxWidth) * 4,
							 yshift=\dimexpr -0.5\btreePointerBoxWidth]{inner0.west}) to[out=345,in=135] (leaf5);

	
	\innerNodeKey{inner0}{2}{\textcolor{red}{23}};\innerNodeKey{inner0}{3}{\textcolor{red}{47}};
}

\end{tikzpicture}}
\end{center}
\end{frame}

%
% ---------------------------------------------------------------------------------------
%
\begin{frame}{B+tree Maintenance in Practice}

Like with flat indexes, in practice, deletions in B+ trees are handled by marking the records \underline{at the leaf nodes} with \alert{``tombstone'' markers} (\includegraphics[height=1em]{figures/RIP.pdf}). When the database goes offline for maintenance, the index is rebuilt and the tombstones are removed.

\vskip1em

The overhead of removing nodes and rearranging pointers to disk blocks is not worth the potential savings they would introduce.

\end{frame}



%
% ---------------------------------------------------------------------------------------
%
\begin{frame}{B+tree: cost summary}

B+trees can be seen as\\
 - a dense flat index at the leaf-level which supports range searches\\
 - a hierarchical sparse index over that flat index, with height $H$

\vskip1em

\textbf{Searching} for a key requires reading \fbox{$O(H)$} disk blocks.  In practice, B+trees with 3 levels can index a very large range of search keys.

\vskip1em

\textbf{Insertions} may require adding new leaf nodes and/or new inner nodes, all the way up to the root, at cost \fbox{$O(H)$} where $H$ is the height of the tree.

\vskip 1em

\textbf{Deletions} cost the same as insertions, but many DBMSs just use ``tombstone'' markers instead for simplicity.

\end{frame}


%
% ---------------------------------------------------------------------------------------
%
\begin{frame}{How useful are B+ trees in practice?}

Indexes are used to implement primary and foreign keys and to speed up the evaluation of predicates in a \lstinline[style=SQL]{WHERE} clause of a query. Without indexes, all of these operations would require reading $O(N)$ blocks of a ``table file'' with $N$ blocks.

With B+trees, except for range queries, all costs are reduced to $O(H)$, where $h$ is the height of the B+tree, which in turn is $O(\log_k N)$, where $N$ is the number of blocks in the table file and $k$ is the number of pointers in each B+tree node.

In complexity terms, \textbf{indexes trade time for space}: they require several disk blocks (space), but save (a lot) of time in accessing tuples. Because secondary storage is ever more inexpensive, indexes are a great idea.


\end{frame}
