%!TEX root = ./lec04_access_methods.tex

%
% -----------------------------------------------------------------------
%
\begin{frame}{Index searches on ranges of values}

Whether or not an index on attribute $a$ helps answer \framebox{$\sigma_{a > x}(R)$} depends on the kind of index and on how many tuples fall in the range $a>x$.

\vskip1em

\begin{block}{\alert{Selectivity} of a predicate}
The \textbf{selectivity} of a predicate $c$ for $R$, is the fraction of \textbf{tuples} in $R$ that satisfy $c$.
\[s(R,c) = \frac{|\sigma_c(R)|}{|R|}\]
\end{block}

\vskip1em

The slides on query processing discuss how to estimate selectivity.
\end{frame}

%
% -----------------------------------------------------------------------
%
\begin{frame}

\textbf{Scenario 1}: a primary flat index on a sequential file.

Assume the index has $M$ blocks and the table file has $N$ blocks.

\vskip2em

\begin{center}
\scalebox{0.75}{\usebox\rangeSelectionQueryPrimaryIndex}
\end{center}

\vskip1em

\textbf{Cost:} \(|I_{a<x}| + |R_{a>x}|\) block reads


\fbox{$|I_{a\leq x}| = \alert{\ceil*{M \cdot \left(1 - s(R,a > x)\right)}}$} : number of \underline{index blocks} up until the first record with key $x$ 

\fbox{$|R_{a>x}| = \alert{\ceil*{N \cdot s(R,a>x)}}$} : number of \underline{data blocks} starting at $x$

\end{frame}


%
% -----------------------------------------------------------------------
%
\begin{frame}
\vskip2em
\begin{columns}[onlytextwidth]
\begin{column}{0.45\textwidth}
\textbf{Example}: 1 million tuples:\\
 - data file: 20,000 blocks \\
 - flat index: 5,000 blocks \\
 - $s(R,a>x)=0.2$
\end{column}
\begin{column}{0.55\textwidth}
\qquad\framebox{\begin{minipage}{\textwidth}\small
I/O Cost= index scan + table scan
\[\ceil*{5,000 \cdot 0.8} + \ceil*{20,000 \cdot 0.2}\]
\end{minipage}}
\end{column}
\end{columns}

\vskip1em

In practice, the DBMS decides whether or not to use the index based on an \emph{estimate} of the selectivity of the predicate.

The actual number of tuples satisfying a selection condition depends on data distributions which can be quite complex. But the DBMS cannot ``remember'' everything about the data in memory. Instead, it keeps some descriptive statistics to perform its estimates.
   
\end{frame}

%
% -----------------------------------------------------------------------
%
\begin{frame}

\textbf{Scenario 2}: multi-level index with a sparse on top of the same dense index as before:

\vskip2em

\begin{columns}[onlytextwidth]
\begin{column}{0.4\textwidth}
\textbf{Example}: 1 million tuples:\\
 - data file: 20,000 blocks \\
 - dense index: 5,000 blocks \\
 - sparse index: 25 blocks \\
 - $s(R,a>x)=0.2$
 \end{column}

\begin{column}{0.6\textwidth}
\scalebox{0.75}{\usebox{\rangeSelectionQueryStackedIndex}}
\end{column}

\end{columns}


\vskip3em

\begin{center}
\qquad\framebox{\begin{minipage}{0.8\textwidth}\small
I/O Cost= sparse index scan + \alert{dense index scan} + table scan
\[\ceil*{K \cdot 0.8} + \alert{\textbf{1}} + \ceil*{N \cdot 0.2}\]
\end{minipage}}
\end{center}
\end{frame}

%
% -----------------------------------------------------------------------
%
\begin{frame}

\textbf{Scenario 3}:  (single-level) dense \textbf{secondary} index.

\begin{center}
\scalebox{0.75}{\usebox\rangeSelectionQuerySecondaryIndex}
\end{center}

In this case, the DBMS \underline{has to read the entire index} and perform a traversal, potentially to a different block, for each pointer out of the index satisfying the predicate.

\vskip2em

\begin{columns}[onlytextwidth]
\begin{column}{0.45\textwidth}
\textbf{Example}: 1 million tuples:\\
 - data file: 20,000 blocks \\
 - dense index: 5,000 blocks \\
 - $s(R,a>x)=0.2$
\end{column}
\begin{column}{0.55\textwidth}
\qquad\framebox{\begin{minipage}{\textwidth}\small
I/O Cost= index scan + pointer lookups
\[5,000 + \ceil*{1,000,000 \cdot 0.2}\]
\end{minipage}}
\end{column}
\end{columns}
\end{frame}

