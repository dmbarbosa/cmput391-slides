%!TEX root = ./lec04_access_methods.tex

%
% -------------------------------------
%
\begin{frame}{Multi-Attribute Access}

Goal: finding records satisfying multiple selection criteria at once\\
- \textbf{Ex:} movies where \lstinline[style=cmput391]{year>2010 AND imdb=7.5}.

\vskip1em

\textbf{Option I:} use an index on \alert{either} attribute, and verify the other predicate on the other attribute for each tuple.

\vskip2em

\begin{columns}[onlytextwidth]
\begin{column}{0.5\textwidth}
\scalebox{0.8}{
\begin{tikzpicture}[semithick,align=center,node distance=0.75cm,every node/.style={inner sep=1,outer sep=1,font=\footnotesize}]
\node (0) at (0,0) {} ; %empty node with ``answer''
\node (1) [below of= 0] {$\sigma_{\text{year>2010}}$};
\node (2) [below of= 1] {\lstinline[style=cmput391]{pointer_lookup}};
\node (3) [below of= 2] {\lstinline[style=cmput391]{get_ids(IDX_Movie_imdb, imdb=7.5)}};
\path[->]
    (3) edge (2)
    (2) edge (1)
    (1) edge (0);
\end{tikzpicture}}
\end{column}

\begin{column}{0.5\textwidth}
\scalebox{0.8}{
\begin{tikzpicture}[semithick,align=center,node distance=0.75cm,every node/.style={inner sep=1,outer sep=1,font=\footnotesize}]
\node (0) at (0,0) {} ; %empty node with ``answer''
\node (1) [below of= 0] {$\sigma_{\text{imdb=7.5}}$};
\node (2) [below of= 1] {\lstinline[style=cmput391]{pointer_lookup}};
\node (3) [below of= 2] {\lstinline[style=cmput391]{get_ids(IDX_Movie_year, year>2010)}};
\path[->]
    (3) edge (2)
    (2) edge (1)
    (1) edge (0);
\end{tikzpicture}}
\end{column}
\end{columns}
\end{frame}

%
% -------------------------------------
%
\begin{frame}

\textbf{Option II:} use two indexes, each on a different attribute, and \alert{merge the lists of tuple ids}.

\vskip2em

\begin{center}
\scalebox{0.75}{
\begin{tikzpicture}[semithick,align=center,node distance=1.5cm,every node/.style={inner sep=1,outer sep=1,font=\footnotesize}]
\node (0) at (0,0) {} ; %empty node with ``answer''
\node (3) [below of= 0] {\lstinline[style=cmput391]{intersect_id_lists}};
\node (4) [below left of =3,xshift=-6em] 
	{\lstinline[style=cmput391]{get_ids(IDX_Movie_imdb, imdb=7.5}\lstinline[style=cmput391]{)}};
\node (5) [below right of =3,xshift=6em]
	 {\lstinline[style=cmput391]{get_ids(IDX_Movie_year, year>2010}\lstinline[style=cmput391]{)}};
\path[->]
    (4) edge (3)
    (3) edge (0);
\path[->]
    (5) edge (3);
\end{tikzpicture}}
\end{center}
\end{frame}

%
% -------------------------------------
%
\begin{frame}

\textbf{Option III:}  use a \alert{multidimensional} index. 

As the name suggests, a multidimensional index considers multiple attributes (the dimensions) at the same time. 

Many of them were designed to deal with numeric data, but they can work with all kinds of attributes.

\vskip2em

In the examples that follow, we will consider multidimensional indexes on \lstinline[style=SQL]{(title, year)} keys.

\end{frame}

\newsavebox{\moviesInIMDBYear}
\savebox{\moviesInIMDBYear}{
	\begin{tikzpicture}[
		every node/.style={font=\Large},
	]
	\draw (0,0) rectangle (10,10);
	\coordinate (m1) at (7.8,0.5);\fill (m1) circle [radius=3pt]; \node [above right=3pt and 1pt of m1] {$t_1$};
	\coordinate (m2) at (6.9,1);\fill (m2) circle [radius=3pt]; \node [above right=3pt and 1pt of m2] {$t_2$};
	\coordinate (m3) at (7.8,5.75);\fill (m3) circle [radius=3pt]; \node [above right=3pt and 1pt of m3] {$t_3$};
	\coordinate (m4) at (8.1,8);\fill (m4) circle [radius=3pt]; \node [above right=3pt and 1pt of m4] {$t_4$};
	\coordinate (m5) at (5.3,9);\fill (m5) circle [radius=3pt]; \node [above right=3pt and 1pt of m5] {$t_5$};
	\coordinate (m6) at (1.3,6.8);\fill (m6) circle [radius=3pt]; \node [above right=3pt and 1pt of m6] {$t_6$};
	\coordinate (m7) at (1.9,9.1);\fill (m7) circle [radius=3pt]; \node [above right=3pt and 1pt of m7] {$t_7$};
	\coordinate (m8) at (4,5.2);\fill (m8) circle [radius=3pt]; \node [above right=3pt and 1pt of m8] {$t_8$};
	\coordinate (m9) at (4.25,4);\fill (m9) circle [radius=3pt]; \node [above right=3pt and 1pt of m9] {$t_9$};
	\coordinate (m10) at (1.5,4.25);\fill (m10) circle [radius=3pt]; \node [above right=3pt and 1pt of m10] {$t_{10}$};
	\coordinate (m11) at (8.5,0.25);\fill (m11) circle [radius=3pt]; \node [above right=3pt and 1pt of m11] {$t_{11}$};
	\coordinate (m12) at (9,1.75);\fill (m12) circle [radius=3pt]; \node [above right=3pt and 1pt of m12] {$t_{12}$};
	\end{tikzpicture}
}

\newenvironment{BLOCK}
{\begin{minipage}{20pt}\baselineskip=0.75\baselineskip \centering}
{\end{minipage}}

%
% -------------------------------------
%
\begin{frame}

\textbf{Grid files} are a \emph{hashing-based} representation of multiple points in a $n$-dimensional space.

\vskip1em

Each cell of the grid points to (a chain) of \alert{bucket files}, with pointers to actual tuples (in the data file) that fall in that cell.

\begin{center}
\scalebox{0.4}{
	\begin{tikzpicture}[
		every node/.style={font=\Large},
	]
	\draw (0,0) rectangle (10,10);
	\node at (-0.75,0.125) {\LARGE 1980};
	\node at (-0.75,9.875) {\LARGE 2020};
	\node at (0.125,-0.5) {\LARGE 0};
	\node at (9.875,-0.5) {\LARGE 10};

	\node at (0,0)[anchor=south west,inner sep=0,outer sep=0,xshift=-2.5pt] {\usebox\moviesInIMDBYear};
	\draw [color=red,dashed] (3.33,0) -- (3.33,10);\draw [color=red,dashed] (6.66,0) -- (6.66,10);
	\draw [color=red,dashed] (0,3.33) -- (10,3.33);\draw [color=red,dashed] (0,6.66) -- (10,6.66);

	\node (grid) at (19,5) {
\begin{tikzpicture}[every node/.style={draw,rectangle,minimum width=4em,minimum height=3em}]
	\node (g11) at (0,0) [label=above:0-3.3,label=left:2006-2020] {};
	\node (g12) [right=0pt of g11] [label=above:3.3-6.6] {};
	\node (g13) [right=0pt of g12] [label=above:6.6-10] {};
	\node (g21) [below=0pt of g11] [label=left:1993-2006] {};
	\node (g22) [right=0pt of g21] {};
	\node (g23) [right=0pt of g22] {};
	\node (g31) [below=0pt of g21] [label=left:1980-1993] {};
	\node (g32) [right=0pt of g31] {};
	\node (g33) [right=0pt of g32] {};

	\node (bl11) at (-2,-5)  {\begin{BLOCK}$t_7$\\$t_6$\end{BLOCK}};
	\draw [densely dotted,thick,*->] ([yshift=1.5em]g11.south) -- (bl11);
	\node (bl21) at (-1,-7)  {\begin{BLOCK}$t_{10}$\\~~~\end{BLOCK}};
	\draw [densely dotted,thick,*->] ([yshift=1.5em]g21.south) -- ([xshift=12pt]bl21.north);
	\node (bl12) at (1,-5)  {\begin{BLOCK}$t_5$\end{BLOCK}};
	\draw [densely dotted,thick,*->] ([yshift=1.5em]g12.south) -- (bl12);
	\node (bl22) at (2,-7)  {\begin{BLOCK}$t_8$\\$t_9$\end{BLOCK}};
	\draw [densely dotted,thick,*->] ([yshift=1.5em]g22.south) -- ([xshift=15pt]bl22.north);
	\node (bl13) at (4,-5)  {\begin{BLOCK}$t_4$\end{BLOCK}};
	\draw [densely dotted,thick,*->] ([yshift=1.5em]g13.south) to[out=225,in=90] ([xshift=-5pt]bl13.north);
	\node (bl23) at (5,-7)  {\begin{BLOCK}$t_3$\end{BLOCK}};
	\draw [densely dotted,thick,*->] ([yshift=1.5em]g23.south) to[out=315,in=90] ([xshift=15pt]bl23.north);
	\node (bl33) at (5,-9)  {\begin{BLOCK}$t_1$\\$t_2$\end{BLOCK}};
	\draw [densely dotted,thick,*->,color=blue] ([yshift=1.5em]g33.south) to[out=345,in=0] (bl33.east);
	\node (bl33_b) at (2,-9)  {\begin{BLOCK}$t_{11}$\\$t_{12}$\end{BLOCK}};
	\draw [thick,*->,color=red] ([yshift=-5pt]bl33.north west) to[out=180,in=0] (bl33_b);
\end{tikzpicture}};
	\end{tikzpicture}}
\end{center}
\end{frame}

%
% -------------------------------------
%
\begin{frame}

The boundaries of the grid \alert{need not} be uniformly spaced across dimensions. Moving boundaries to ``balance'' the number of points per grid cell is an interesting optimization problem.

\vskip1em

\begin{center}
\scalebox{0.4}{
	\begin{tikzpicture}[
		every node/.style={font=\Large},
	]
	\draw (0,0) rectangle (10,10);
	\node at (-0.75,0.125) {\LARGE 1980};
	\node at (-0.75,9.875) {\LARGE 2020};
	\node at (0.125,-0.5) {\LARGE 0};
	\node at (9.875,-0.5) {\LARGE 10};

	\node at (0,0)[anchor=south west,inner sep=0,outer sep=0,xshift=-2.5pt] {\usebox\moviesInIMDBYear};
	\draw [color=red,dashed] (3,0) -- (3,10);
	\draw [color=red,dashed] (6.1,0) -- (6.1,10);
	
	\draw [color=red,dashed] (0,4.75) -- (10,4.75);
	\draw [color=red,dashed] (0,0.75) -- (10,0.75);

	\node (grid) at (19,5) {
\begin{tikzpicture}[every node/.style={draw,rectangle,minimum width=4em,minimum height=3em}]
	\node (g11) at (0,0) [label=above:0-3,label=left:1999-2020] {};
	\node (g12) [right=0pt of g11] [label=above:3-6.1] {};
	\node (g13) [right=0pt of g12] [label=above:6.1-10] {};
	\node (g21) [below=0pt of g11] [label=left:1985-1999] {};
	\node (g22) [right=0pt of g21] {};
	\node (g23) [right=0pt of g22] {};
	\node (g31) [below=0pt of g21] [label=left:1980-1985] {};
	\node (g32) [right=0pt of g31] {};
	\node (g33) [right=0pt of g32] {};

	\node (bl11) at (-2,-5)  {\begin{BLOCK}$t_7$\\$t_6$\end{BLOCK}};
	\draw [densely dotted,thick,*->] ([yshift=1.5em]g11.south) -- (bl11);
	\node (bl21) at (-1,-7)  {\begin{BLOCK}$t_{10}$\\~~~\end{BLOCK}};
	\draw [densely dotted,thick,*->] ([yshift=1.5em]g21.south) -- ([xshift=12pt]bl21.north);
	\node (bl12) at (1,-5)  {\begin{BLOCK}$t_5$\\$t_8$\end{BLOCK}};
	\draw [densely dotted,thick,*->] ([yshift=1.5em]g12.south) -- (bl12);
	\node (bl22) at (2,-7)  {\begin{BLOCK}$t_9$\end{BLOCK}};
	\draw [densely dotted,thick,*->] ([yshift=1.5em]g22.south) -- ([xshift=15pt]bl22.north);
	\node (bl13) at (4,-5)  {\begin{BLOCK}$t_4$\\$t_3$\end{BLOCK}};
	\draw [densely dotted,thick,*->] ([yshift=1.5em]g13.south) to[out=225,in=90] ([xshift=-5pt]bl13.north);
	\node (bl23) at (5,-7)  {\begin{BLOCK}$t_2$\\$t_{12}$\end{BLOCK}};
	\draw [densely dotted,thick,*->] ([yshift=1.5em]g23.south) to[out=315,in=90] ([xshift=15pt]bl23.north);
	\node (bl33) at (5,-9)  {\begin{BLOCK}$t_1$\\$t_{11}$\end{BLOCK}};
	\draw [densely dotted,thick,*->] ([yshift=1em]g33.south) to[out=345,in=0] (bl33.east);
\end{tikzpicture}};
	\end{tikzpicture}}
\end{center}
\end{frame}

%
% -------------------------------------
%
\begin{frame}

A \textbf{Quadtree} divides the space into four quadrants, recursively as needed, until each quadrant corresponds to a single block (of a bucket file).\footnote{\url{https://en.wikipedia.org/wiki/Quadtree}}

\begin{center}
\scalebox{0.4}{
	\begin{tikzpicture}[
		every node/.style={font=\Large},
	]
	\draw (0,0) rectangle (10,10);
	\node at (-0.75,0.125) {\LARGE 1980};
	\node at (-0.75,9.875) {\LARGE 2020};
	\node at (0.125,-0.5) {\LARGE 0};
	\node at (9.875,-0.5) {\LARGE 10};

	\node at (0,0)[anchor=south west,inner sep=0,outer sep=0,xshift=-2.5pt] {\usebox\moviesInIMDBYear};
	\draw [very thick,color=red,dashed] (5,0) -- (5,10);\draw [very thick,color=red,dashed] (0,5) -- (10,5);

	%Q1 split
	\draw [color=olive,thick,dashed] (2.5,5) -- (2.5,10);\draw [color=olive,thick,dashed] (0,7.5) -- (5,7.5);
	%Q2 split
	\draw [color=olive,thick,dashed] (7.5,5) -- (7.5,10);\draw [color=olive,thick,dashed] (5,7.5) -- (10,7.5);
	%Q4 split
	\draw [color=olive,thick,dashed] (7.5,0) -- (7.5,5);\draw [color=olive,thick,dashed] (5,2.5) -- (10,2.5);
	%Q4.4 split
	\draw [color=magenta,dashed] (8.75,0) -- (8.75,2.5);\draw [color=magenta,dashed] (7.5,1.25) -- (10,1.25);

	\node (grid) at (19,5) {
\begin{tikzpicture}[
	block/.style={draw,rectangle,minimum width=4em,minimum height=3em},
	tree/.style={thick,draw,ellipse,color=blue},
	direction/.style={fill=background,rectangle,midway}
	]
	\node (root) at (-2,0) [tree] {$5.0, 2000$};

	\node (R_SW) at (-2,-3) [block] {\begin{BLOCK}$t_{10}$\\$t_9$\end{BLOCK}};
	\node (R_NW) at (-5,-3) [tree] {$2.5, 2010$};
	\node (R_NE) at (1,-3) [tree] {$7.5, 2010$};
	\node (R_SE) at (5,-3) [tree] {$7.5, 1990$};
	\draw [->,tree] (root) -- (R_NW) node[direction] {NW};
	\draw [->,tree] (root) -- (R_NE) node[direction] {NE};
	\draw [->,tree] (root) -- (R_SW) node[direction] {SW};
	\draw [->,tree] (root) -- (R_SE) node[direction] {SE};
		
	\node (R_NW_SW) at (-7,-6) [block] {\begin{BLOCK}$t_{6}$\end{BLOCK}};
	\node (R_NW_NW) at (-5.5,-7.5) [block] {\begin{BLOCK}$t_{7}$\end{BLOCK}};
	\node (R_NW_SE) at (-4,-9) [block] {\begin{BLOCK}$t_{8}$\end{BLOCK}};
	\draw [->,tree] (R_NW) -- (R_NW_SW) node[direction] {SW};
	\draw [->,tree] (R_NW) -- (R_NW_NW) node[direction] {NW};
	\draw [->,tree] (R_NW) -- (R_NW_SE) node[direction] {SE};

	\node (R_NE_NW) at (-2,-6) [block] {\begin{BLOCK}$t_{5}$\end{BLOCK}};
	\node (R_NE_NE) at (-0.5,-7.5) [block] {\begin{BLOCK}$t_{4}$\end{BLOCK}};
	\node (R_NE_SE) at (1,-9) [block] {\begin{BLOCK}$t_{3}$\end{BLOCK}};

	\draw [->,tree] (R_NE) -- (R_NE_NW) node[direction] {NW};
	\draw [->,tree] (R_NE) -- (R_NE_NE) node[direction] {NE};
	\draw [->,tree] (R_NE) -- (R_NE_SE) node[direction] {SE};

	\node (R_SE_SE) at (5,-8) [tree] {$8.75, 1985$};
	\node (R_SE_SW) at (3,-6) [block] {\begin{BLOCK}$t_{2}$\end{BLOCK}};
	\draw [->,tree] (R_SE) -- (R_SE_SW) node[direction] {SW};
	\draw [->,tree] (R_SE) -- (R_SE_SE) node[direction] {SE};

	\node (R_SE_SE_SW) at (3,-11) [block] {\begin{BLOCK}$t_1$\\$t_{11}$\end{BLOCK}};
	\node (R_SE_SE_NE) at (4.5,-12.5) [block] {\begin{BLOCK}$t_{12}$\end{BLOCK}};

	\draw [->,tree] (R_SE_SE) -- (R_SE_SE_SW) node[direction] {SW};
	\draw [->,tree] (R_SE_SE) -- (R_SE_SE_NE) node[direction] {NE};
\end{tikzpicture}};
	\end{tikzpicture}}
\end{center}
\end{frame}

%
% -------------------------------------
%
\begin{frame}

All multidimensional indexes extend to more than two dimensions. However, as the number of dimensions grow, so does the complexity of the index \textbf{and} the number of \underline{empty regions} that need to be represented.

The costs of creating and updating these indexes depends on many design decisions of their creators, but in general they required more work than updating unidimensional B+trees or hash files.

\vskip1em

% Like with other indexes, instead of merging/coalescing nodes as the data is updated, most DBMSs periodically re-index the data instead.

One very important multidimensional index (which is covered in the labs), designed for spatial data (e.g., maps), is the ``region tree'' (\alert{\textbf{R tree}}). Like with B+trees, nodes in the R tree correspond to disk blocks and represent (MBRs of) sub-regions of the space, instead of individual points.
\end{frame}



