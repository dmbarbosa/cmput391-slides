%!TEX root = ./lec04_access_methods.tex

%
%
% tuple identifiers
%   --> pointers, virtual memory, database address space
%

\newsavebox\SQLiteRowID
\begin{lrbox}{\SQLiteRowID}\begin{minipage}{0.725\textwidth}
\lstset{style=SQL,%
  keywords=[4]{sqlite},keywordstyle=[4]\ttfamily\bfseries\color{exampleColor},%
  literate =* {>}{{\textcolor{exampleColor}{>}}}1 
              {|}{{{|}}}1} 
\begin{lstlisting}
sqlite> CREATE TABLE Temp(name CHAR(20));
sqlite> INSERT INTO Temp
   ...> VALUES ('Susan'), ('Bob'), ('Susan');
sqlite> SELECT rowid, * FROM Temp;
1|Susan
2|Bob
3|Susan
sqlite>
\end{lstlisting}
\end{minipage}
\end{lrbox}

%
% ---------------------------------------------------------------------------
%
\begin{frame}{Tuple/Row identifiers}

Most DBMSs assign \emph{identifiers} which are unique to every tuple in the same table,\footnote{Sometimes the identifiers are unique within the whole database.} called \emph{tuple} ids or \emph{row} ids.

For example, in SQLite, every tuple is assigned a 64-bit signed integer key that uniquely identifies the tuple in the table--you insert the same data twice, you get two different tuples with different ids.

\vskip1em

\begin{columns}
\begin{column}{0.4125\textwidth}
Tuple ids are typically visible to the application. 

\vskip0.5em

In SQLite, one can even update them.
\end{column}
\begin{column}{0.5\textwidth}
\hspace*{-2em}
\scalebox{0.75}{\fbox{\usebox{\SQLiteRowID}}}
\end{column}
\end{columns}
\end{frame}

\newsavebox\SQLiteDeleteRowID
\begin{lrbox}{\SQLiteDeleteRowID}\begin{minipage}{0.75\textwidth}
\lstset{style=SQL,%
  keywords=[4]{sqlite},keywordstyle=[4]\ttfamily\bfseries\color{exampleColor},%
  literate =* {>}{{\textcolor{exampleColor}{>}}}1 
              {|}{{{|}}}1} 
\begin{lstlisting}
sqlite> DELETE FROM Temp WHERE rowid IN (
   ...>    SELECT rowid FROM Temp t
   ...>    WHERE EXISTS (SELECT t2.rowid
   ...>                  FROM Temp t2
   ...>                  WHERE t2.name=t.name
   ...>                  AND t2.rowid<t.rowid));
sqlite> SELECT rowid, * FROM Temp;
1|Susan
2|Bob
sqlite>
\end{lstlisting}
\end{minipage}
\end{lrbox}

%
% ---------------------------------------------------------------------------
%
\begin{frame}
Tuple ids were initially introduced in relational systems to support the storage of objects in object-oriented programming languages, where object identity is important. Two objects are considered different if their ids are different, regardless of values.

\vskip1em

\begin{columns}
\begin{column}{0.4125\textwidth}
Tuple ids can be useful in some cases, such as eliminating duplicate tuples in relations without keys.
\vskip0.75em

In some DBMSs, tuple ids are actually symbolic references to the record
\end{column}
\begin{column}{0.5\textwidth}
\hspace*{-2em}
\scalebox{0.75}{\fbox{\usebox{\SQLiteDeleteRowID}}}
\end{column}
\end{columns}
for the tuple inside the corresponding data file.

One interesting use of tuple ids is in implementing column stores.
\end{frame}

%
%
% column stores
%
%

%
% ---------------------------------------------------------------------------
%
\begin{frame}{Column-oriented stores}

With a tuple-oriented store (slide~\ref{tuple_oriented_stores}) each record corresponds to an entire tuple of a table. For some applications, this design turns out to be sub-optimal.

In \textbf{scientific applications} it is common for some relations to have hundreds or thousands of attributes, leading to records that are quite large, and a low record/block ratio.

Typical queries in these applications involve only a small fraction of the attributes in the \lstinline[style=SQL]{WHERE} clause and in the value expressions, resulting in a high I/O ``per useful attribute'' cost.

\textbf{On Line Analytical Processing} queries apply over business data often computing aggregations and groupings defined over a small fraction of the schema attributes.

\end{frame}

%
% ---------------------------------------------------------------------------
%
\begin{frame}
The idea behind a column-store is to have separate data files for each of the attributes in the relation. 

For example, the Movie relation in the running example would be stored in four separate data files like so:

\vskip1em

\begin{columns}[onlytextwidth]
\begin{column}[t]{0.3\textwidth}
\scalebox{0.75}{\usebox{\MovieTableColTitle}}
\end{column}
\begin{column}{0.15\textwidth}
\scalebox{0.75}{\usebox{\MovieTableColYear}}
\end{column}
\begin{column}{0.15\textwidth}
\scalebox{0.75}{\usebox{\MovieTableColIMDB}}
\end{column}
\begin{column}{0.3\textwidth}
\scalebox{0.75}{\usebox{\MovieTableColDirector}}
\end{column}
\end{columns}

\vskip1em

This design has some advantages from a storage point of view:\\
 - records are very simple: one id and one attribute;\\
 - \lstinline[style=SQL]{NULL} values are easy to deal with: just ignore them;\\
 - \textbf{data compression} can be used in each file separately.

\end{frame}

%
% ---------------------------------------------------------------------------
%
\begin{frame}
What about query processing?
\vskip1em

\begin{columns}[onlytextwidth]
\begin{column}[t]{0.3\textwidth}
\scalebox{0.75}{\usebox{\MovieTableColTitle}}
\end{column}
\begin{column}{0.15\textwidth}
\scalebox{0.75}{\usebox{\MovieTableColYear}}
\end{column}
\begin{column}{0.15\textwidth}
\scalebox{0.75}{\usebox{\MovieTableColIMDB}}
\end{column}
\begin{column}{0.3\textwidth}
\scalebox{0.75}{\usebox{\MovieTableColDirector}}
\end{column}
\end{columns}

\vskip1em

It is possible to evaluate SQL quite efficiently with column stores. In fact, these systems outperform tuple-stores by quite a margin on most analytical queries.

The main idea is to process each condition in the \lstinline[style=SQL]{WHERE} clause separately, producing a list of tuple ids satisfying each condition. Then merge these lists based on the logical connector: \lstinline[style=SQL]{AND} means intersection, \lstinline[style=SQL]{OR} means union.

Is there anything this design is \textbf{bad} for?\\
 - YES: \lstinline[style=SQL]{SELECT * ...} queries.

\end{frame}