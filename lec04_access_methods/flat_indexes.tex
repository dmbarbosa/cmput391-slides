%!TEX root = ./lec04_access_methods.tex

%
% -----------------------------------------------------------------------
%
\begin{frame}[fragile]{Indexes Are Associative Data Structures}

An index is an \alert{\textbf{associative data structure}} (stored on disk) that allows the DBMS to find tuples that satisfy a predicate defined over a subset of the attributes in the tuple.

Index declarations are part of the SQL DDL:

\vskip1em

\begin{center}
\lstinline[style=SQL]{ CREATE INDEX -:name :- On -:Table:- (-:attr_1 :- [, -:attr_2 :-[, ... [,-:attr_n:-]]]);}
\end{center}

The sequence of attributes \lstinline[style=cmput391]{-:attr_1:-, -:attr_2:-, ... ,-:attr_n:-} is called the index key (a.k.a., sort key, or search key).

\vskip2em

The index \textbf{associates} every index key with a \textbf{pointer} to the tuple containing that key.

\end{frame}


\begin{frame}[fragile]

Indexes save time in answering queries by allowing the DBMS to skip tuples that do not satisfy predicates in the \lstinline[style=cmput391]{WHERE} clause\footnote{DBAs always look at the query workload to look for opportunities to add more indexes.}.

\textbf{Example:}\qquad \begin{minipage}{0.75\textwidth}
\begin{lstlisting}[style=SQL]
CREATE INDEX IDX1 ON Movie(title);

SELECT director
FROM Movie
WHERE title='Ghostbusters' AND year=1984
\end{lstlisting}
\end{minipage}

Even if there are thousands of movies, the DBMS can use \lstinline[style=cmput391]{IDX1} to quickly find only those with ``Ghostbusters'' in the title (and then check the condition on year).

\end{frame}

%
% -----------------------------------------------------------------------
%
\begin{frame}[fragile]

Index maintenance:

\vskip1em

\begin{center}
	\lstinline[style=SQL]{CREATE INDEX IDX1 ON Movie(title);}
\end{center}

\vskip2em

The DBMS keeps the index \textbf{synchronized} with the table:
\begin{itemize}[-]
\item \alert{Every tuple} of \lstinline[style=cmput391]{Movie} has a corresponding entry in \lstinline[style=cmput391]{IDX1}.
\item The DBMS updates the index \alert{after every update} (insertion, deletion or update of tuples) to the table.
\end{itemize}
 
\end{frame}

%
% -----------------------------------------------------------------------
%
\begin{frame}{Flat Indexes}

The next slides develop the basic ingredients of the B+ tree index, which is prevalent in modern DBMSs. We start with ``\textbf{flat indexes}'', corresponding to the lower level of a B+tree:

\vskip1em
\begin{block}{}
A \textbf{flat index} is a \textbf{sequential file} in which every record has two fields:\\
- the index \alert{key}: one or more attributes of tuple $t_i$ \\
- a \alert{pointer} to $t_i$
\end{block}
\vskip1em

As we will see, \textbf{hierarchical indexes} such as the B+tree are built on top of flat indexes, and this is why we will spend some time understanding them.


\end{frame}
%
% -----------------------------------------------------------------------
%
\begin{frame}
\label{index_title_movie}
Example flat index on attribute \lstinline[style=cmput391]{title} of table Movie:

\begin{center}
\lstinline[style=cmput391]{> CREATE INDEX IDX1 ON Movie(Title);}
\end{center}

\vskip2em

\begin{center}
\begin{tikzpicture}
\node [anchor=north west] at (-0.15,0) {\scalebox{0.85}{\usebox\TitleIndexOnMovie}};

\node [color=accent] at (1,0.75) {Index File};
\node [color=accent] at (5,0.75) {Data File};

\only<2-3|handout:0>{
\node (0) [draw,color=blue,rectangle,minimum width=2.7cm,minimum height=1.5cm] at (1.3,-0.75) {};
\node (1) [color=blue] at (-1.2,-0.75) {\scriptsize block 1};
\draw [color=blue,->] (1) -- (0);
}
\only<3|handout:0>{
\node (2) [draw,color=blue,rectangle,minimum width=2.7cm,minimum height=1.5cm] at (1.3,-2.7) {};
\node (3) [color=blue] at (-1.2,-2.7) {\scriptsize block 2};
\draw [color=blue,->] (3) -- (2);
}
\only<4-5|handout:0>{
\node (4) [draw,color=red,rectangle,minimum width=1.5cm,minimum height=0.3cm] at (1.15,-2.2) {};
\node (5) [color=red] at (-1,-2.2) {\scriptsize key};
\draw [color=red,->] (5) -- (4);
}

\only<5|handout:0>{
\node (4) [draw,color=red,rectangle,minimum width=0.6cm,minimum height=0.3cm] at (2.25,-2.2) {};
\node (5) [color=red] at (4,-3.1) {\scriptsize pointer};
\draw [color=red,->] (5) -- (4);
}
\end{tikzpicture}
\end{center}
\end{frame}


%
% -----------------------------------------------------------------------
%
\begin{frame}
Index keys can be more than one column. 

Example flat index on attributes \lstinline[style=cmput391]{Title} and \lstinline[style=cmput391]{Year}:

\begin{center}
\lstinline[style=cmput391]{> CREATE INDEX IDX2 ON Movie(Title,Year);}
\end{center}

\vskip2em

\begin{center}
\scalebox{0.85}{\usebox\TitleYearIndexOnMovie}
\end{center}

\vskip1em

The \alert{order of the attributes in the key matters}! An index on \lstinline[style=cmput391]{Title,Year} is very different than an index on \lstinline[style=cmput391]{Year,Title}.

\end{frame}

\newsavebox\densePrimaryIndex
\savebox{\densePrimaryIndex}{
\begin{tikzpicture}
\indexBlock{IB1}{0}{0};
\keyPointer{kp1}{IB1}{0}{3};
\keyPointer{kp2}{IB1}{1}{5};
\keyPointer{kp3}{IB1}{2}{7};
\keyPointer{kp4}{IB1}{3}{11};

\indexBlock{IB2}{0}{-2};\linkIndexBlocks{IB1}{IB2};
\keyPointer{kp5}{IB2}{0}{13};

\dataBlock{DB1}{3}{0};
\tuple{t1}{DB1}{0}{3};
\tuple{t2}{DB1}{1}{5};

\dataBlock{DB2}{3}{-1.25};\linkDataBlocks{DB1}{DB2};
\tuple{t3}{DB2}{0}{7};
\tuple{t4}{DB2}{1}{11};

\dataBlock{DB3}{3}{-2.5};\linkDataBlocks{DB2}{DB3};
\tuple{t5}{DB3}{0}{13};


\KPtoTuple{kp1}{t1};\KPtoTuple{kp2}{t2};\KPtoTuple{kp3}{t3};
\KPtoTuple{kp4}{t4};\KPtoTuple{kp5}{t5};%\KPtoTuple{kp6}{t6};
\end{tikzpicture}}

\newsavebox\sparseIndex
\savebox{\sparseIndex}{
\begin{tikzpicture}
\indexBlock{IB1}{0}{0};
\keyPointer{kp1}{IB1}{0}{3};
\keyPointer{kp2}{IB1}{1}{7};
\keyPointer{kp3}{IB1}{2}{13};

\dataBlock{DB1}{3}{0};
\tuple{t1}{DB1}{0}{3};
\tuple{t2}{DB1}{1}{5};

\dataBlock{DB2}{3}{-1.25};\linkDataBlocks{DB1}{DB2};
\tuple{t3}{DB2}{0}{7};
\tuple{t4}{DB2}{1}{11};

\dataBlock{DB3}{3}{-2.5};\linkDataBlocks{DB2}{DB3};
\tuple{t5}{DB3}{0}{13};

\KPtoTuple{kp1}{t1};\KPtoTuple{kp2}{t3};\KPtoTuple{kp3}{t5};
\end{tikzpicture}
}


%
% -----------------------------------------------------------------------
%
\begin{frame}{Terminology}

\vskip1em

\begin{columns}[onlytextwidth]
\begin{column}{0.45\textwidth}
A \textbf{dense index} has every index key in the table (meaning one pointer per tuple)
\end{column}
\begin{column}{0.55\textwidth}
\qquad\scalebox{0.75}{\usebox\densePrimaryIndex}
\end{column}
\end{columns}

\vskip3em

\begin{columns}[onlytextwidth]
\begin{column}{0.45\textwidth}
A \textbf{sparse index} has only some of the keys (and fewer pointers than tuples)

\vskip0.5em

Normally, each key in the index points to a different block of the data
\end{column}
\begin{column}{0.55\textwidth}
\qquad\scalebox{0.75}{\usebox\sparseIndex}
\end{column}
\end{columns}
\end{frame}

%
% -----------------------------------------------------------------------
%
\begin{frame}{Primary Index}

A \textbf{primary index} is a dense index on a sequential file \emph{sorted by the same keys} as the index.

\vskip1em

\begin{center}
\scalebox{0.75}{\usebox\densePrimaryIndex}
\end{center}

\vskip1em

\alert{Primary indexes speed up enforcing primary key constraints.}

\vskip1em

\end{frame}

%
% -----------------------------------------------------------------------
%
\begin{frame}{Secondary Index}
% \vskip2em
\begin{columns}[onlytextwidth]
\begin{column}{0.45\textwidth}
In a \textbf{secondary index}, the keys in the index file are not sorted in the same way as the records on the data file.

\vskip0.5em

Secondary indexes are always dense.\footnotemark
\end{column}
\begin{column}{0.55\textwidth}
\qquad\scalebox{0.75}{
\begin{tikzpicture}
\indexBlock{IB1}{0}{0};
\keyPointer{kp1}{IB1}{0}{3};
\keyPointer{kp2}{IB1}{1}{5};
\keyPointer{kp3}{IB1}{2}{7};
\keyPointer{kp4}{IB1}{3}{11};

\indexBlock{IB2}{0}{-2};\linkIndexBlocks{IB1}{IB2};
\keyPointer{kp5}{IB2}{0}{13};

\dataBlock{DB1}{3}{0};
\tuple{t5}{DB1}{0}{13};
\tuple{t4}{DB1}{1}{11};

\dataBlock{DB2}{3}{-1.25};\linkDataBlocks{DB1}{DB2};
\tuple{t3}{DB2}{0}{7};
\tuple{t2}{DB2}{1}{5};

\dataBlock{DB3}{3}{-2.5};\linkDataBlocks{DB2}{DB3};
\tuple{t1}{DB3}{0}{3};

\KPtoTuple{kp1}{t1};\KPtoTuple{kp2}{t2};\KPtoTuple{kp3}{t3};
\KPtoTuple{kp4}{t4};\KPtoTuple{kp5}{t5};%\KPtoTuple{kp6}{t6};
\end{tikzpicture}}
\end{column}
\end{columns}

\vskip0.5em

\footnotetext{It should be obvious that a sparse secondary index would be useless.}

Indexes created on attributes that are not the primary key are secondary indexes.
\end{frame}

%
% -----------------------------------------------------------------------
%
\begin{frame}{How does an index save time?}

Blocks of the index file are the same size as the blocks of the (sequential or heap) data file. 

Because the \alert{index records are smaller}, each block of the index covers more keys than a block of the data file.

Typically, a disk block can hold:\\
 - a few hundred key/pointer pairs or\\
 - dozens of tuples

\vskip2em

\begin{columns}[onlytextwidth]
\begin{column}{0.5\textwidth}
\textbf{Example}: table with 1 million tuples; each disk block can hold:\\
 - 50 tuple (records) or \\
 - 200 key-pointer pairs
\end{column}
\begin{column}{0.5\textwidth}
\qquad\framebox{\begin{minipage}{0.8\textwidth}
Sizes:\\
 - data file: 20,000 blocks \\
 - index file: 5,000 blocks 
\end{minipage}}
\end{column}
\end{columns}
\end{frame}



%
% -----------------------------------------------------------------------
%
\begin{frame}

\textbf{Binary search on an index?}

\vskip1em

\begin{center}
\framebox{\begin{minipage}{0.85\textwidth}
\textbf{Binary search} for key \alert{$k$} on a sorted list of keys \textbf{in memory}:\\
- find \alert{$m$}, the key \underline{``in the middle''} of the list\\
- stop the search if \alert{$m=k$};\\ 
- search recursively on the ``lower half'' if \alert{$k<m$}\\
- search recursively on the ``upper half'', otherwise
\end{minipage}}
\end{center}

\vskip1em

\begin{BOX}{Binary Search \textbf{does not} work on an index file on disk}
Binary Search requires $O(1)$ access time to \emph{any} of the keys, which is not possible if the blocks of the index are on disk.
\end{BOX}

\vskip2em

\begin{center}
\usebox{\genericDenseIndexWithNblocks}
\end{center}
\end{frame}

%
% -----------------------------------------------------------------------
%
\begin{frame}{Multi-level indexes}

One way to reduce the number of I/Os to search for a key in an index is to \textbf{stack} a sparse index over a dense index, of course, using the same index key.

\vskip2em

\begin{center}
\usebox{\genericMultilevelIndexSparseOnDense}
\end{center}

\vskip1em

Each record in the sparse index contains the \textbf{smallest} key in each block of the index below.
\begin{itemize}[-,topsep=-0.5em]
\item Therefore, every record in the sparse index corresponds to 200 records in the dense index ``below''.
\end{itemize}

\end{frame}

%
% -----------------------------------------------------------------------
%

\begin{frame}

\vskip1em

\begin{center}
\usebox{\genericMultilevelIndexSparseOnDense}
\end{center}

\vskip1em

To find a record with index key $k=x$:
\begin{enumerate}[(1),topsep=-0.5em,noitemsep]
\item Search the sparse index for \alert{$\max(x') \leq x$}.
\item Search the block of the dense index starting at key \alert{$x'$}.
\item If the key is found, traverse the pointer to read the tuple from the table file.
\end{enumerate}


\end{frame}



%
% -----------------------------------------------------------------------
%
\begin{frame}

How large will the sparse index be?

\vskip2em

\begin{center}
\usebox{\genericMultilevelIndexSparseOnDense}
\end{center}


\vskip2em

\begin{columns}[onlytextwidth]
\begin{column}{0.5\textwidth}
\textbf{Example}: table with 1 million tuples; each disk block can hold:\\
 - 50 tuple (records) or \\
 - 200 key-pointer pairs
\end{column}
\begin{column}{0.5\textwidth}
\qquad\framebox{\begin{minipage}{0.8\textwidth}
We would have:\\
 - data file: 20,000 blocks \\
 - dense index: 5,000 blocks \\
 - sparse index: \alert{25 blocks}
\end{minipage}}
\end{column}
\end{columns}
\end{frame}

%%%% I/O cost example

%
% -----------------------------------------------------------------------
%
\begin{frame}
\label{multilevel_index_cost_example}


\textbf{Example:} cost of finding a tuple in the table by searching for a key.

\vskip2em

\begin{columns}[onlytextwidth]
\begin{column}{0.6\textwidth}
\scalebox{0.75}{\clipbox{45 0 0 0}{\usebox{\genericMultilevelIndexSparseOnDense}}}
\end{column}
\begin{column}{0.4\textwidth}
\qquad\framebox{\begin{minipage}{0.9\textwidth}\footnotesize
 - data file: 20,000 blocks \\
 - dense index: 5,000 blocks \\
 - sparse index: \alert{25 blocks}
\end{minipage}}
\end{column}
\end{columns}


\vskip2em 

Cost of \textbf{retrieving} a tuple with key $k$:\\
- without any index: \pause 20,000 I/Os\\
- with dense index only: \pause 5,000 \pause \alert{+1} I/Os\\
- with both indexes: \pause 25 \alert{+2} I/Os
\end{frame}


%
% -----------------------------------------------------------------------
%
\begin{frame}{Access Costs With (Stacked) Flat Indexes}

Recall from slide ~\ref{costs_heap_sequential_files} that insertions, deletions and updates depended on the cost of \emph{finding} the tuples in the table files first.


Assume table $R$ has $T(R)$ tuples and is stored on $M$ blocks on disk, and that we are searching for tuples satisfying a condition $c$.

Generally, the cost has two parts:
\begin{enumerate}[(1),topsep=-0.5em,noitemsep]
\item Finding the tuples in the index.
\item Retrieving each such tuple from the table (via a pointer traversal)
\end{enumerate}

If the table is stored on a heap file, the number of pointer traversals is $\ceil*{s(R,c)\cdot T(R)}$. Otherwise, it will be $\ceil*{s(R,c)\cdot M}$.

The index cost is $\ceil*{(1-s(R,c)\cdot K) + h-1}$, where $K$ is the size (in blocks) of the top-most flat index and $h$ is the number of stacked indexes.

\vskip1em

\end{frame}


\begin{frame}

\vskip1em

I/O cost to retrieve all tuples in $R$ satisfying \alert{condition $c: k=x$}:

\begin{enumerate}[label=(\arabic*),topsep=-0.5em,noitemsep]
\item $R$ on \textbf{heap file}, \underline{no index}: $M$ block reads
\item $R$ on \textbf{sequential file}, \underline{no index}: $\ceil*{M\cdot \left(s(R, k < x) + s(R,k=x)\right)}$
\end{enumerate}

\vskip1em

If we have a \underline{dense flat index} on $k$ occupying $N$ blocks:

\begin{enumerate}[label=(\arabic*),topsep=-0.5em,noitemsep]
\item $R$ on \textbf{heap file}: $\ceil*{N\cdot s(R, k\leq x) + s(R, k=x)\cdot T(R)}$ 
\item $R$ on \textbf{sequential file}: $\ceil*{N\cdot s(R, k\leq x) + M\cdot s(R, k=x)}$
\end{enumerate}


\vskip1em

If we have \underline{stacked sparse/dense indexes} on $k$ occupying $K$ and $N$ blocks, respectively:

\begin{enumerate}[label=(\arabic*),topsep=-0.5em,noitemsep]
\item $R$ on \textbf{heap file}: $\ceil*{K\cdot s(R,k\leq x) + 1 + s(R,k=v)\cdot T(R)}$ 
\item $R$ on \textbf{sequential file}: $\ceil*{K\cdot s(R,k\leq x) + 1 + M\cdot s(R,k=x)}$
\end{enumerate}


\end{frame}

%
% -----------------------------------------------------------------------
%
\begin{frame}

Every sparse index we add \textbf{further reduces the fraction} of the blocks from an index below that we need to scan:

\begin{center}
\scalebox{0.75}{\clipbox{45 0 0 0}{\usebox{\genericMultilevelIndexSparseOnDense}}}
\end{center}

Thus, the more levels of sparse indexes the better!

\vskip1em

\alert{B+trees take this idea to the extreme}, defining a self-balanced tree of index blocks in which:

\begin{enumerate}[label=(\arabic*)]
\item the topmost level (the root) is a \textbf{single-block} sparse index
\item the leaves form a dense flat index
\item the levels in between are sparse indexes
\end{enumerate} 

\end{frame}




%
% -----------------------------------------------------------------------
%
\begin{frame}{Index maintenance}

Any index is only useful if kept up-to-date after updates to the associated table.

\vskip1em

Flat indexes are sequential files, and thus have the same update costs as in slide~\ref{costs_heap_sequential_files}.
 
Updating sparse indexes (like the higher levels of a multi-level index) is no different: all we need to look for is if there is room in the right block.

\vskip1em

The next few slides show examples of issues that arise when updating a flat index and its table.

\end{frame}

\begin{frame}

\vskip0.5em

\begin{columns}[onlytextwidth]
\begin{column}{0.6\textwidth}
\vskip0.5em 
\textbf{Example:} inserting a tuple with \textbf{index key 4}. Assume the table is kept in a \framebox{sequential file} for now.
\end{column}
\begin{column}{0.4\textwidth}
\qquad\scalebox{0.5}{\usebox\densePrimaryIndex}
\end{column}
\end{columns}

\vskip1.5em


\textbf{Option 1:} Re-arrange records to use existing free space:
\vskip1em

\begin{columns}[onlytextwidth]
\begin{column}{0.5\textwidth}
\scalebox{0.6}{
\begin{tikzpicture}
\indexBlock{IB1}{0}{0};
\keyPointer{kp1}{IB1}{0}{3};
\keyPointer{kp6}{IB1}{1}{\alert{4}};
\keyPointer{kp2}{IB1}{2}{5};
\keyPointer{kp3}{IB1}{3}{7};

\indexBlock{IB2}{0}{-2};\linkIndexBlocks{IB1}{IB2};
\keyPointer{kp4}{IB2}{0}{11};
\keyPointer{kp5}{IB2}{1}{13};

\dataBlock{DB1}{3}{0};
\tuple{t1}{DB1}{0}{3};
\tuple{t6}{DB1}{1}{\alert{4}};

\dataBlock{DB2}{3}{-1.25};\linkDataBlocks{DB1}{DB2};
\tuple{t2}{DB2}{0}{5};
\tuple{t3}{DB2}{1}{7};

\dataBlock{DB3}{3}{-2.5};\linkDataBlocks{DB1}{DB2};
\tuple{t4}{DB3}{0}{11};
\tuple{t5}{DB3}{1}{13};

\KPtoTuple{kp1}{t1};\KPtoTuple{kp2}{t2};\KPtoTuple{kp3}{t3};
\KPtoTuple{kp4}{t4};\KPtoTuple{kp5}{t5};\KPtoTuple{kp6}{t6};
\end{tikzpicture}}
\end{column}

\begin{column}{0.5\textwidth}

I/O intensive:\\
- may require moving $O(|R|)$ of tuples to either end of the file\\
- same for the index

\end{column}
\end{columns}


\end{frame}

%
% -----------------------------------------------------------------------
%
\begin{frame}

\textbf{Option 2:} Split existing blocks or add new ones as needed.
\vskip0.5em

\scalebox{0.6}{
\begin{tikzpicture}
\indexBlock{IB1}{0}{0};
\keyPointer{kp1}{IB1}{0}{3};
\keyPointer{kp6}{IB1}{1}{\alert{4}};

\indexBlock{IB2}{0}{-2}\linkIndexBlocks{IB1}{IB2};
\keyPointer{kp2}{IB2}{0}{5};
\keyPointer{kp3}{IB2}{1}{7};
\keyPointer{kp4}{IB2}{2}{11};

\indexBlock{IB3}{0}{-4};\linkIndexBlocks{IB2}{IB3};
\keyPointer{kp5}{IB3}{0}{13};

\dataBlock{DB1}{3}{0};
\tuple{t1}{DB1}{0}{3};
\tuple{t6}{DB1}{1}{\alert{4}};

\dataBlock{DB2}{3}{-1.25};\linkDataBlocks{DB1}{DB2};
\tuple{t2}{DB2}{0}{5};

\dataBlock{DB3}{3}{-2.5};\linkDataBlocks{DB2}{DB3};
\tuple{t3}{DB3}{0}{7};
\tuple{t4}{DB3}{1}{11};

\dataBlock{DB4}{3}{-3.75};\linkDataBlocks{DB3}{DB4};
\tuple{t5}{DB4}{0}{11};

\KPtoTuple{kp1}{t1};\KPtoTuple{kp2}{t2};\KPtoTuple{kp3}{t3};
\KPtoTuple{kp4}{t4};\KPtoTuple{kp5}{t5};\KPtoTuple{kp6}{t6};

\node (0) at (-10,0) [anchor=north west] {\scalebox{0.9}{\usebox\densePrimaryIndex}};

\node (1) at (-8.85,-0.8) [semithick,draw,color=red,style=dashed,rectangle,minimum width=6em,minimum height=6em,label=above:{\textcolor{red}{SPLIT}}] {};

\node (1) at (-5.5,-0.5) [semithick,draw,color=red,style=dashed,rectangle,minimum width=8.5em,minimum height=3em,label=above:{\textcolor{red}{SPLIT}}] {};
\end{tikzpicture}}

\vskip1em

\textbf{Pro:} fixed number of blocks added/split; in other words, this strategy has  $O(1)$ (constant) I/O cost

\textbf{Con:} too much empty space reduces the effectiveness of the index
\end{frame}

%
% -----------------------------------------------------------------------
%
\begin{frame}
\vskip1em

With a sparse index, the DBMS can avoid modifying the index by \emph{expanding} a data block with an \alert{overflow} block:

\vskip1em

\begin{columns}[onlytextwidth]
\begin{column}{0.4\textwidth}
\textbf{Example:} insertion of tuple with index key 4.

\vskip0.5em
Logically, there is now \emph{one} block with keys 3, 4 and 5.
\end{column}
\begin{column}{0.6\textwidth}
\scalebox{0.6}{
\begin{tikzpicture}
\node (0) at (0,0) [anchor=north west] {\usebox{\sparseIndex}};

\tikzset{dataBlockBox/.append style={color=red,thick}}
\tikzset{dataBlockPointerBox/.append style={color=red,fill=snow,thick}}

\dataBlock{overflow}{8}{-0.155};
\tuple{to}{overflow}{0}{4};
\node (1) at (6,-0.5) [outer sep=1,anchor=south west,draw,color=red,semithick,rectangle,minimum width=1em, minimum height=1em] {};

\draw [*->,>=stealth',thick,color=red] ([xshift=3pt,yshift=0pt]1.west) to[out=0,in=180] (overflow);
\end{tikzpicture}}
\end{column}
\end{columns}

\vskip1em
Many overflow blocks can be used, forming an \alert{overflow chain}.

With overflow blocks or chains the search time becomes non-uniform, as some blocks are larger than others.

During re-indexing, the DBMS removes all overflow chains and re-builds the index.
\end{frame}


%
% -----------------------------------------------------------------------
%
\begin{frame}
\vskip2em

Deletions are, logically, the opposite of insertions. Which means that if we insert a tuple that causes the table file and the index file to grow, deleting that same tuple should cause both files to shrink.

\vskip1em

\begin{block}{But is that what the DBMS should do?}

Performing deletions in this way incurs in more I/O operations.

And we may have to grow the files again in the future due to another insertion.
\end{block}

\vskip1em

In practice, the best way is to mark the deleted records with \alert{``tombstone'' marks} (\includegraphics[height=1em]{figures/RIP.pdf}), leaving the space available for future insertions.



\end{frame}

\begin{frame}



\vskip2em

\begin{columns}[onlytextwidth]
\begin{column}{0.55\textwidth}
\textbf{Example:} deletion of tuple with index key 5.

\vskip0.5em
A record with key between 3 and 7 can go where the record with 5 was.
\end{column}
\begin{column}{0.3\textwidth}
\scalebox{0.5}{
\begin{tikzpicture}
\node at (0,0) [anchor=north west] {\usebox{\sparseIndex}};
\node at (4.8, -0.75) [color=blue,draw,rectangle,minimum width=7em,minimum height=\tupleHeight,fill=blue!10,inner sep=0] {\includegraphics[height=1em]{figures/RIP.pdf}};
\end{tikzpicture}}
\end{column}
\end{columns}

\vskip2em

\begin{columns}[onlytextwidth]
\begin{column}{0.55\textwidth}
\textbf{Example:} further deletion of tuple with index key 7.

\vskip0.5em
Any record with index key between 3 and 11 can go on either block.
\end{column}
\begin{column}{0.3\textwidth}
\scalebox{0.5}{
\begin{tikzpicture}
\indexBlock{IB1}{0}{0};
\keyPointer{kp1}{IB1}{0}{3};
\keyPointer{kp2}{IB1}{1}{11};
\keyPointer{kp3}{IB1}{2}{13};

\dataBlock{DB1}{3}{0};
\tuple{t1}{DB1}{0}{3};

\dataBlock{DB2}{3}{-1.25};\linkDataBlocks{DB1}{DB2};
\tuple{t2}{DB2}{0}{7};
\tuple{t3}{DB2}{1}{11};

\dataBlock{DB3}{3}{-2.5};\linkDataBlocks{DB2}{DB3};
\tuple{t5}{DB3}{0}{13};

\KPtoTuple{kp1}{t1};\KPtoTuple{kp2}{t3};\KPtoTuple{kp3}{t5};

\node at (4.2, -0.6) [color=blue,draw,rectangle,minimum width=7em,minimum height=\tupleHeight,fill=blue!10,inner sep=0] {\includegraphics[height=1em]{figures/RIP.pdf}};
\node at (4.2, -1.44) [color=blue,draw,rectangle,minimum width=7em,minimum height=\tupleHeight,fill=blue!10,inner sep=0] {\includegraphics[height=1em]{figures/RIP.pdf}};
\end{tikzpicture}
}
\end{column}
\end{columns}
\end{frame}


%
% -----------------------------------------------------------------------
%
\begin{frame}
\vskip1em
\begin{BOX}{Updating a search key?}
Update statements that modify the index key are handled like a deletion (of the original tuple) followed by an insertion (of a ``new'' tuple with the new index key).
\end{BOX}

After many insertions and deletions causing node splits or the use of tombstones, the data and index files can become \textbf{fragmented}, leading to sub-optimal I/O performance.

\vskip1em

\begin{BOX}{Re-indexing}
Database administrators can set the DBMS to periodically re-build all data and index files, to redistribute the records across all blocks, removing largely unused blocks.
\end{BOX}
\end{frame}

%
% -----------------------------------------------------------------------
%