% ---------------------- get the tuplex from the example tables

\newsavebox\MovieI
\savebox{\MovieI}{
	\clipbox{0 45.125 0 22.125}{\usebox\MovieTable}
}
\newsavebox\MovieII
\savebox{\MovieII}{
	\clipbox{0 33.75 0 33.45}{\usebox\MovieTable}
}
\newsavebox\MovieIII
\savebox{\MovieIII}{
	\clipbox{0 22.65 0 44.75}{\usebox\MovieTable}
}
\newsavebox\MovieIV
\savebox{\MovieIV}{
	\clipbox{0 11.3 0 56}{\usebox\MovieTable}
}
\newsavebox\MovieV
\savebox{\MovieV}{
	\clipbox{0 0 0 67.25}{\usebox\MovieTable}
}

\newsavebox\MovieVI
\savebox{\MovieVI}{
	\clipbox{0 0 0 33.5}{\usebox\MovieTableTWO}
}

\def\movieTupleGHOSTBUSTERSone{\scalebox{0.75}{\usebox{\MovieI}}}
\def\movieTupleBIG{\scalebox{0.75}{\usebox{\MovieII}}}
\def\movieTupleLOSTinTRANSLATION{\scalebox{0.75}{\usebox{\MovieIII}}}
\def\movieTupleWADJDA{\scalebox{0.75}{\usebox{\MovieIV}}}
\def\movieTupleGHOSTBUSTERStwo{\scalebox{0.75}{\usebox{\MovieV}}}
\def\movieTupleGROUNDHOGday{\scalebox{0.75}{\usebox{\MovieVI}}}

\def\idxKeyTitleBIG{\scalebox{0.75}{\clipbox{5 0 140 0}{\usebox\MovieII}}}
\def\idxKeyTitleGHOSTBUSTERSone{\scalebox{0.75}{\clipbox{5 0 140 0}{\usebox\MovieI}}}
\def\idxKeyTitleGHOSTBUSTERStwo{\scalebox{0.75}{\clipbox{5 0 140 0}{\usebox\MovieV}}}
\def\idxKeyTitleLOSTinTRANSLATION{\scalebox{0.75}{\clipbox{5 0 140 0}{\usebox\MovieIII}}}
\def\idxKeyTitleWADJDA{\scalebox{0.75}{\clipbox{5 0 140 0}{\usebox\MovieIV}}}

\def\idxKeyTitleYearBIG{\scalebox{0.75}{\clipbox{8 0 111 0}{\usebox\MovieII}}}
\def\idxKeyTitleYearGHOSTBUSTERSone{\scalebox{0.75}{\clipbox{8 0 111 0}{\usebox\MovieI}}}
\def\idxKeyTitleYearGHOSTBUSTERStwo{\scalebox{0.75}{\clipbox{8 0 111 0}{\usebox\MovieV}}}
\def\idxKeyTitleYearLOSTinTRANSLATION{\scalebox{0.75}{\clipbox{8 0 111 0}{\usebox\MovieIII}}}
\def\idxKeyTitleYearWADJDA{\scalebox{0.75}{\clipbox{8 0 111 0}{\usebox\MovieIV}}}


% -------------------------- macros for drawing tuples in the data file

\tikzset{
	tuple/.style={
		anchor=north west,
		inner sep=0pt,outer sep=0pt,shift={(0pt,-2pt)}
	}
}

\tikzset{ % same as above, except for the anchor part
	tupleBelow/.style={
		inner sep=0pt,outer sep=0pt,shift={(0pt,-2pt)}
	}
}


% #1     --> id of tuple (i.e., the node with the value)
% #2, #3 --> coordinages of top-left corner of node with tuple value
% #4     --> value in the attribute of the tuple
\def\tuple#1#2#3#4{
  \node ({#1}) at ({#2},{#3}) [tuple] {{#4}} ;
}

% #1     --> id of tuple (i.e., the node with the value)
% #2, #3 --> coordinages of top-left corner of node with tuple value
% #4     --> value in the attribute of the tuple
\def\tupleBelow#1#2#3{
  \node ({#1}) [below = -2pt of {#2}] [tupleBelow] {{#3}} ;
}


% --------------------------- macros for drawing blocks of the data file

\tikzset{
	MovieDataBlock/.style={
		draw,rectangle,minimum width=5.5125cm, minimum height=0.69cm,anchor=north west,inner sep=1pt,outer sep=1pt
	}
}

\tikzset{
	MovieDataBlockPointerBox/.style={
		fill=snow,draw,rectangle,minimum width=1em,minimum height=1em,
		anchor=north west,
		inner sep=1pt,outer sep=1pt,
		shift={(5.5125cm,-0.34cm)}
	}
}


% #1    --> block id
% #2,#3 --> coordinates of top-left corner
% #4    --> id of the first tuple
% #5    --> value of the first tuple
\def\movieDataBlock#1#2#3#4#5{
	\node ({#1}) at ({#2},{#3}) [MovieDataBlock] {};
	\node at ({#2},{#3}) [MovieDataBlockPointerBox] {};
	
	\tuple{#4}{#2}{#3}{#5};
}

% #1     --> block id
% #2     --> id of block ``above'' this one
% #3, #4 --> coordinates of this block
% #5     --> id of the first tuple inside the block
% #6     --> value of the first tuple
\def\movieDataBlockBelow#1#2#3#4#5#6{
	\node ({#1}) at ({#3},{#4}) [MovieDataBlock] {};
	\node at ({#3},{#4}) [MovieDataBlockPointerBox] {};
	\draw [*->] ([xshift=4pt,yshift=8pt]{#2}.south east) to[out=270,in=90] ([xshift=-10pt]#1.north east);

	\tuple{#5}{#3}{#4}{#6};
}


% --------------------------- macros for drawing the keypointer pairs in the index

\tikzset{
	indexKeyBox/.style={
		anchor=north west,
		inner sep=0pt,outer sep=0pt,shift={(2pt,-2pt)}
	}
}

\tikzset{ %same as above, except for the anchor point
	indexKeyBelowBox/.style={ 
		inner sep=0pt,outer sep=0pt
	}
}

\tikzset{
	indexPointerBox/.style={
		fill=snow,draw,rectangle,minimum width=0.75cm,minimum height=0.3cm,
		inner sep=1pt,outer sep=1pt
	}
}

% draws two rectangles, one with an attribute value, the other with the pointer
%
% #1     --> id of key-pointer (i.e., the node with the value)
% #2, #3 --> coordinages of top-left corner of node with tuple value
% #4     --> value in the attribute of the tuple
\def\movieIndexKeyPointer#1#2#3#4{
  \node ({#1}) at ({#2},{#3}) [indexKeyBox] {#4} ;
  \node [right = -0.5pt of {#1}] [indexPointerBox] {};
}

% #1     --> id of key-pointer (i.e., the node with the value)
% #2     --> id of key-pointer immediately above
% #4     --> value in the attribute of the tuple
\def\movieIndexKeyPointerBelow#1#2#3{
  \node ({#1}) [below = -0.5pt of {#2}] [indexKeyBelowBox] {\tiny{#3}} ;
  \node [right = -0.5pt of {#1}] [indexPointerBox] {};
}


% --------------------------- macros for drawing blocks of the index file

\tikzset{
	MovieIndexBlock/.style={
		draw,rectangle,anchor=north west,inner sep=1pt,outer sep=1pt
	}
}

\tikzset{
	MovieIndexBlockPointerBox/.style={
		fill=snow,draw,rectangle,minimum height=1em,minimum width=1em,
		anchor=north west,
		inner sep=0pt,outer sep=0pt
	}
}

% #1    --> block id
% #2,#3 --> coordinates of top-left corner
% #4    --> WIDTH of block box
% #5    --> HEIGHT of the block box
% #6    --> id of the key-pointer pair
% #7    --> value of the first key-pointer pair
\def\movieIndexBlock#1#2#3#4#5#6#7{
	\node ({#1}) at ({#2},{#3}) [minimum width={#4},minimum height={#5},MovieIndexBlock] {};
	\node [below left= 0pt and 0pt of {#1}] [shift={(-0.9em,1.1em)},MovieIndexBlockPointerBox] {};

	\movieIndexKeyPointer{#6}{#2}{#3}{#7};
}

% #1    --> block id
% #2    --> id of block above
% #3,#4 --> coordinates of top-left corner
% #5    --> WIDTH of block box
% #6    --> HEIGHT of the block box
% #7    --> id of the key-pointer pair
% #8    --> value of the first key-pointer pair
\def\movieIndexBlockBelow#1#2#3#4#5#6#7#8{
	\node ({#1}) at ({#3},{#4}) [minimum width={#5},minimum height={#6},MovieIndexBlock] {};
	\node [below left= 0pt and 0pt of {#1}] [shift={(-0.9em,1.1em)},MovieIndexBlockPointerBox] {};
	\draw [*->] ([xshift=-0.4em,yshift=0.8em]{#2}.south west) to[out=270,in=90] ([xshift=10pt]{#1}.north west);

	\movieIndexKeyPointer{#7}{#3}{#4}{#8};
}


% ----------------------------- draw line between pointer into tuple

% #1 --> key-pointer id
% #2 --> tuple id
\def\KPtoTuple#1#2{
	\draw [*->] ([xshift=0.35cm,yshift=0em]{#1}.east) to[out=0,in=180] (#2.west);
}

% ======================== actual examples

\newsavebox\TitleIndexOnMovie
\savebox{\TitleIndexOnMovie}{
\begin{tikzpicture}
\movieDataBlock{Mdb1}{0}{0}{t1}{\movieTupleGHOSTBUSTERSone};
\tupleBelow{t2}{t1}{\movieTupleBIG};
\movieDataBlockBelow{Mdb2}{Mdb1}{0}{-1.25}{t3}{\movieTupleLOSTinTRANSLATION};
\tupleBelow{t4}{t3}{\movieTupleWADJDA};
\movieDataBlockBelow{Mdb3}{Mdb2}{0}{-2.5}{t5}{\movieTupleGHOSTBUSTERStwo};

\movieIndexBlock{MIb1}{-4.5}{0}{7.6em}{3.6em}{kp1}{\idxKeyTitleBIG};
\movieIndexKeyPointerBelow{kp2}{kp1}{\idxKeyTitleGHOSTBUSTERSone};
\movieIndexKeyPointerBelow{kp3}{kp2}{\idxKeyTitleGHOSTBUSTERStwo};
\movieIndexKeyPointerBelow{kp4}{kp3}{\idxKeyTitleLOSTinTRANSLATION};
\movieIndexBlockBelow{MIb2}{MIb1}{-4.5}{-2.25}{7.6em}{3.6em}{kp5}{\idxKeyTitleWADJDA};

\KPtoTuple{kp2}{t1};
\KPtoTuple{kp1}{t2};
\KPtoTuple{kp4}{t3};
\KPtoTuple{kp5}{t4};
\KPtoTuple{kp3}{t5};
\end{tikzpicture}}


\newsavebox\TitleYearIndexOnMovie
\savebox{\TitleYearIndexOnMovie}{
\begin{tikzpicture}
\movieDataBlock{Mdb1}{0}{0}{t1}{\movieTupleGHOSTBUSTERSone};
\tupleBelow{t2}{t1}{\movieTupleBIG};
\movieDataBlockBelow{Mdb2}{Mdb1}{0}{-1.25}{t3}{\movieTupleLOSTinTRANSLATION};
\tupleBelow{t4}{t3}{\movieTupleWADJDA};
\movieDataBlockBelow{Mdb3}{Mdb2}{0}{-2.5}{t5}{\movieTupleGHOSTBUSTERStwo};

\movieIndexBlock{MIb1}{-4.5}{0}{9.6em}{2.8em}{kp1}{\idxKeyTitleYearBIG};
\movieIndexKeyPointerBelow{kp2}{kp1}{\idxKeyTitleYearGHOSTBUSTERSone};
\movieIndexKeyPointerBelow{kp3}{kp2}{\idxKeyTitleYearGHOSTBUSTERStwo};

\movieIndexBlockBelow{MIb2}{MIb1}{-4.5}{-2.25}{9.6em}{2.8em}{kp4}{\idxKeyTitleYearLOSTinTRANSLATION};
\movieIndexKeyPointerBelow{kp5}{kp4}{\idxKeyTitleYearWADJDA};

\KPtoTuple{kp2}{t1};
\KPtoTuple{kp1}{t2};
\KPtoTuple{kp4}{t3};
\KPtoTuple{kp5}{t4};
\KPtoTuple{kp3}{t5};
\end{tikzpicture}}


\newsavebox\SequentialFileMovieTitleYear
\savebox{\SequentialFileMovieTitleYear}{
\begin{tikzpicture}
\movieDataBlock{Mdb1}{0}{0}{t1}{\movieTupleBIG};
\tupleBelow{t2}{t1}{\movieTupleGHOSTBUSTERSone};
\movieDataBlockBelow{Mdb2}{Mdb1}{0}{-1.25}{t3}{\movieTupleGHOSTBUSTERStwo};
\tupleBelow{t4}{t3}{\movieTupleLOSTinTRANSLATION};
\movieDataBlockBelow{Mdb3}{Mdb2}{0}{-2.5}{t5}{\movieTupleWADJDA};
\end{tikzpicture}}

\newsavebox\SequentialFileMovieTitlYearUpdated
\savebox{\SequentialFileMovieTitlYearUpdated}{
\begin{tikzpicture}
\movieDataBlock{Mdb1}{0}{0}{t1}{\movieTupleBIG};
\tupleBelow{t2}{t1}{\movieTupleGHOSTBUSTERSone};
\movieDataBlockBelow{Mdb2}{Mdb1}{0}{-1.25}{t3}{\movieTupleGHOSTBUSTERStwo};
\tupleBelow{t4}{t3}{\movieTupleGROUNDHOGday};
\movieDataBlockBelow{Mdb3}{Mdb2}{0}{-2.5}{t5}{\movieTupleLOSTinTRANSLATION};
\tupleBelow{t6}{t5}{\movieTupleWADJDA};
\end{tikzpicture}}

\newsavebox\SequentialFileMovieTitlYearUpdatedTWO
\savebox{\SequentialFileMovieTitlYearUpdatedTWO}{
\begin{tikzpicture}
\movieDataBlock{Mdb1}{0}{0}{t1}{\movieTupleBIG};
\tupleBelow{t2}{t1}{\movieTupleGHOSTBUSTERSone};
\movieDataBlockBelow{Mdb2}{Mdb1}{0}{-1.25}{t3}{\movieTupleGHOSTBUSTERStwo};
\movieDataBlockBelow{Mdb3}{Mdb2}{0}{-2.5}{t4}{\movieTupleGROUNDHOGday};
\tupleBelow{t5}{t4}{\movieTupleLOSTinTRANSLATION};
\movieDataBlockBelow{Mdb4}{Mdb3}{0}{-3.75}{t6}{\movieTupleWADJDA};
\end{tikzpicture}}

