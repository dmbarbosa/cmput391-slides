\newlength{\btreeKeyWidth}
\setlength{\btreeKeyWidth}{\widthof{~~99~~}\relax}

\newlength{\btreeKeyHeight}
\setlength{\btreeKeyHeight}{0.75\btreeKeyWidth}

\newlength{\btreePointerBoxHeight}
\setlength{\btreePointerBoxHeight}{\dimexpr 0.85\btreeKeyHeight\relax}

\newlength{\btreeNodeHeight}
\setlength{\btreeNodeHeight}{\dimexpr(3pt + \btreePointerBoxHeight + \btreeKeyHeight)\relax}


%
% the widths of the key box and the pointer box are related
%
\newlength{\btreeNodeWidth}
\newlength{\btreePointerBoxWidth}


\def\btreeKeysPerNode#1{
	\pgfmathsetmacro{\NKEYS}{#1}
	\pgfmathsetmacro{\NPOINTERS}{\NKEYS + 1}
	\setlength{\btreeNodeWidth}{\dimexpr 1pt + ((\btreeKeyWidth + 1pt) * \NKEYS)}
	
	% the next one is trickier with \dimexpr :(
	\pgfmathsetlength{\btreePointerBoxWidth}{((\btreeNodeWidth - 1pt) / (\NPOINTERS))-1pt}
}
\btreeKeysPerNode{3} %default

\tikzset{
	btreeNodeBox/.style={
		anchor=north west,
		draw,rectangle,
		minimum width=\btreeNodeWidth,minimum height=\btreeNodeHeight
	}
}

\tikzset{
	btreeBlockPointerBox/.style={
		anchor=south west, 
		draw, rectangle, 
		minimum width=\btreePointerBoxWidth, minimum height=\btreePointerBoxHeight,
		fill=snow
	}
}

% #1 --> id of node where pointer originates
% #2 --> # of pointer box (0, 1, 2)
\def\dummyDataLink#1#2{
	\coordinate (dummy) at ([xshift=\dimexpr 1pt + 0.5\btreePointerBoxWidth + (1pt+\btreePointerBoxWidth) * #2,
							 yshift=\dimexpr -2.5em + -0.425\btreePointerBoxWidth]{#1.west});
	\draw [*->,>=stealth'] ([xshift=\dimexpr 1pt + 0.5\btreePointerBoxWidth + (1pt+\btreePointerBoxWidth) * #2,
							 yshift=\dimexpr -0.425\btreePointerBoxWidth]{#1.west}) to[out=270,in=90] (dummy);
}

% #1 --> node id
% #2 --> position within node (0,1,2)
% #3 --> key
\def\innerNodeKey#1#2#3{
	\node [draw,below left = 0pt and 0pt of {#1}.north west, 
			minimum width = \btreeKeyWidth, minimum height = \btreeKeyHeight,
			xshift={ 1.5pt + \btreeKeyWidth + (\dimexpr 1pt+ \btreeKeyWidth) * #2},
			yshift={ -1pt }] {#3};
}

% #1 --> node id
% #2 --> position within node (0,1,2)
% #3 --> key
\def\leafNodeKey#1#2#3{
	\node [draw,below left = 0pt and 0pt of {#1}.north west, 
			minimum width = \btreeKeyWidth, minimum height = \btreeKeyHeight,
			xshift={ 1.5pt + \btreeKeyWidth + (\dimexpr 1pt+ \btreeKeyWidth) * #2},
			yshift={ -1pt }] {#3};
	\dummyDataLink{#1}{#2};
}


% #1     --> block id
% #2, #3 --> block top-left coordinates
\def\btreeNode#1#2#3{
	\node ({#1}) at ({#2},{#3}) [anchor=north west, btreeNodeBox] {};
	%
	% draw the empty boxes for the keys
	%
	\pgfmathsetmacro{\TEMP}{\NKEYS - 1}
	\foreach \i in {0,...,\TEMP}{
		\innerNodeKey{#1}{\i}{};	
	}
	%
	% draw the data pointer boxes
	%
	\pgfmathsetmacro{\TEMP}{\NPOINTERS - 1}
	\foreach \i in {0,...,\TEMP}{
		\node [above left=0pt and 0pt of {#1}.south west, btreeBlockPointerBox,
				xshift=\dimexpr 1pt + (1pt + \btreePointerBoxWidth) * \i, 
				yshift=\dimexpr 1pt ] {};
	}
}

% #1 --> id of node where pointer originates
% #2 --> # of pointer box (0, 1, 2)
% #3 --> id of target node/object
\def\parentChildLink#1#2#3{
	\draw [*->,>=stealth'] ([xshift=\dimexpr 1pt + 0.5\btreePointerBoxWidth + (1pt+\btreePointerBoxWidth) * #2,
							 yshift=\dimexpr -0.425\btreePointerBoxWidth]{#1.west}) to[out=270,in=90] (#3);
}

% #1 --> id of node where pointer originates
% #2 --> # of pointer box (0, 1, 2)
% #3 --> id of target node/object
\def\LABELparentChildLink#1#2#3#4{
	\draw [*->,>=stealth'] ([xshift=\dimexpr 1pt + 0.5\btreePointerBoxWidth + (1pt+\btreePointerBoxWidth) * #2,
							 yshift=\dimexpr -0.425\btreePointerBoxWidth]{#1.west}) to[out=270,in=90] 
							 node [fill,color=background,midway] {\textcolor{red}{#4}} (#3); 
}


% #1 --> id of node where pointer originates
% #2 --> id of target node/object
\def\chainLink#1#2{
	\draw [*->,>=stealth'] ([xshift=-0.725\btreePointerBoxWidth,
							 yshift=-0.5725\btreePointerBoxWidth]{#1.east}) to[out=0,in=180] (#2);
}
