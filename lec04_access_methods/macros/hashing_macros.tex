\newlength{\hashvalueWidth}
\setlength{\hashvalueWidth}{\widthof{~0000~}\relax}

\newlength{\hashvalueHeight}
\setlength{\hashvalueHeight}{1.45em}

\newlength{\pointerWidth}
\setlength{\pointerWidth}{0.75\hashvalueWidth}

\newlength{\dataBoxWidth}
\setlength{\dataBoxWidth}{2\hashvalueWidth}

\newlength{\dynamicHashBlockWidth}
\setlength{\dynamicHashBlockWidth}{\dimexpr 4pt + \hashvalueWidth + \dataBoxWidth}

\newlength{\staticHashBlockWidth}
\setlength{\staticHashBlockWidth}{\dimexpr 2pt + \dataBoxWidth}

\newlength{\hashBlockHeight}
\setlength{\hashBlockHeight}{\dimexpr 3pt + \hashvalueHeight*2\relax}

\tikzset{
	dynamicHashBlock/.style={
		anchor=north west,
		draw,rectangle,
		inner sep=0, outer sep=0,
		minimum width=\dynamicHashBlockWidth,minimum height=\hashBlockHeight
	}
}

\tikzset{
	staticHashBlock/.style={
		anchor=north west,
		draw,rectangle,
		inner sep=0, outer sep=0,
		minimum width=\staticHashBlockWidth,minimum height=\hashBlockHeight
	}
}

\tikzset{
	pointerBox/.style={
		anchor=north west, 
		draw, rectangle, 
		minimum width=\pointerWidth, minimum height=\hashvalueHeight,
		fill=snow
	}
}

\tikzset{
	dataBox/.style={
		anchor=north west, 
		draw, rectangle, 
		inner sep=0,
		minimum width=\dataBoxWidth,minimum height=\hashvalueHeight,
		fill=gray!10
	}
}


\tikzset{
	hashCodeBox/.style={
		minimum height=\hashvalueHeight,anchor=north west,%draw
	}
}

% #1     --> block id
% #2, #3 --> block top-left coordinates
% #4     --> Number of bits used
\def\extensibleHashBlock#1#2#3#4{
	\node ({#1}) at ({#2},{#3}) [anchor=north west, dynamicHashBlock] {};
	\node [color=accent,above right=0.5pt and 0pt of {#1}.north west,draw,minimum width=1.5em]{\footnotesize #4};
}

% #1     --> block id
% #2, #3 --> block top-left coordinates
% #4     --> logical bucket
\def\linearHashBlock#1#2#3#4{
	\node ({#1}) at ({#2},{#3}) [anchor=north west, dynamicHashBlock] {};
	\node [left= 1em of #1] {#4};
}

% #1     --> block id
% #2, #3 --> block top-left coordinates
% #4     --> logical bucket
\def\missingLinearHashBlock#1#2#3#4{
	\node ({#1}) at ({#2},{#3}) [anchor=north west, dynamicHashBlock,dashed,fern] {};
	\node [left= 1em of #1] {#4};
}


% #1     --> block id
% #2, #3 --> block top-left coordinates
% #4     --> bucket id
\def\staticHashBlock#1#2#3{
	\node ({#1}) at ({#2},{#3}) [anchor=north west, staticHashBlock] {};
}


\newcounter{tIdCnt}

% #1  --> block id
% #2  --> relative position (1st, 2nd)
% #3  --> hash key
% #4  --> tuple  
\def\dynamicHashBlockEntry#1#2#3#4{
	\pgfmathsetmacro{\NKEYS}{#2}
	\node (t) [hashCodeBox,below right=0.5pt and 0.5pt of {#1}.north west,
				yshift = - \dimexpr 0.5pt + (1pt + 1.4em) * \NKEYS] {#3};
	\node (t1) [right=0.5pt of t.north east,dataBox] {{#4}};
}

% #1     --> block id
% #2     --> relative position (1st, 2nd)
\def\staticHashBlockEntry#1#2#3{

	\stepcounter{tIdCnt}

	\pgfmathsetmacro{\NKEYS}{#2}
	\node (t) [dataBox,below right=0.5pt and 0.75pt of {#1}.north west,
				yshift = - \dimexpr 0.5pt + (1pt + 1.4em) * \NKEYS] {\small$t_{\arabic{tIdCnt}}$};
}


% #1     --> id
% #2, #3 --> coordinates
% #4     --> i= ?
\def\directory#1#2#3#4{
	\node ({#1}) at ({#2},{#3}) [pointerBox] {};
	\node [above=1em of #1] {$i={#4}$};

	% draw the directory boxes
	\pgfmathsetmacro{\N}{(2^#4)-1}
	\pgfmathsetbasenumberlength{#4} % set the number of digits in binary	
	\pgfmathdectobase\binaryIndex{0}{2}
	\node [left= 0.5em of #1] {\binaryIndex};

	\foreach \i in {1,...,\N}{
		\pgfmathdectobase\binaryIndex{\i}{2}
		\node (t) [below=0pt of #1.north west,pointerBox,yshift=-\dimexpr ((1pt+\hashvalueHeight) * \i)\relax] {};
		\node [left= 0.5em of t] {\binaryIndex};
	}	
}

% #1 --> id of node with directory
% #2 --> # of pointer box (0, 1, 2)
% #3 --> id of target node/object
\def\hashLink#1#2#3{
	\draw [*->,>=stealth'] 
			([yshift=-\dimexpr((1pt+\hashvalueHeight) * #2),xshift=-0.6\pointerWidth]#1.east) 
			to [out=0,in=180]  (#3);
}

% #1, #2   --> coordinates
% #3,#4,#5 --> values of i, n, r
\def\linearHashVariables#1#2#3#4#5{
	\node (t) at ({#1},{#2}) [anchor=north west] {$i={#3}$};
	\node (t1) [below=1em of t] {$n={#4}$};
	\node (t2) [below=1em of t1] {$r={#5}$};
}
