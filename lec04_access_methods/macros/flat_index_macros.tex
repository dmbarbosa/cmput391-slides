%
% Key-pointer measurements
%
\newlength{\indexKeyBoxWidth}
\setlength{\indexKeyBoxWidth}{\dimexpr 2em\relax}

\newlength{\pointerBoxWidth}
\setlength{\pointerBoxWidth}{\dimexpr 2.5em\relax}

%
% Tuple measurements
%
\newlength{\tupleHeight}
\setlength{\tupleHeight}{\dimexpr 1.1em\relax}

\newlength{\dummyTupleGreyBoxWidth}
\setlength{\dummyTupleGreyBoxWidth}{\dimexpr 5em\relax}

%
% Data Block measurements
%

\newlength{\tupleWidth}
\setlength{\tupleWidth}{\dimexpr(\indexKeyBoxWidth+\dummyTupleGreyBoxWidth)\relax}

\newlength{\dataBlockWidth}
\setlength{\dataBlockWidth}{\dimexpr(\tupleWidth+1pt)\relax}

\def\tuplesPerDataBlock{2}

\newlength{\dataBlockHeight}
\setlength{\dataBlockHeight}{\dimexpr1pt+(\tupleHeight+0.5pt)*\the\numexpr\tuplesPerDataBlock\relax}

%
% Index block measurements
%
\newlength{\indexBlockWidth}
\setlength{\indexBlockWidth}{\dimexpr(\indexKeyBoxWidth+\pointerBoxWidth+1pt)\relax}

\def\keyPointerPairsPerIndexBlock{4}

\newlength{\indexBlockHeight}
\setlength{\indexBlockHeight}{\dimexpr1pt+(\tupleHeight+0.5pt)*\the\numexpr\keyPointerPairsPerIndexBlock\relax}

\newlength{\indexBlockPointerOffset}
\setlength{\indexBlockPointerOffset}{\dimexpr(\indexBlockHeight-1em-1pt)\relax}


\tikzset{ % same as tuple, except for the anchor part
	indexKeyBox/.style={
		draw,rectangle,minimum width=\indexKeyBoxWidth,minimum height=\tupleHeight,inner sep=0,outer sep=0,font=\footnotesize
	}
}

\tikzset{
	tupleGrayBox/.style={
		draw, rectangle, minimum width=\dummyTupleGreyBoxWidth,minimum height=\tupleHeight,outer sep=0,fill=tupleBoxColor
	}
}

% #1 --> block id
% #2,#3 --> data block id and order inside block
% #4 --> key
\def\tuple#1#2#3#4{
  \node ({#1}) at ([xshift=1pt,yshift=\dimexpr -1pt-(\tupleHeight+0.5pt)*#3]{#2}.north west) [anchor=north west,indexKeyBox] {{#4}};
  \node [right = -0.5pt of {#1}] [tupleGrayBox] {};
}


% #1 --> block id
% #2,#3 --> data block id and order inside block
% #4 --> key
\def\tupleFromBox#1#2#3#4{
  \node ({#1}) at ([xshift=1pt,yshift=\dimexpr -1pt-(\tupleHeight+0.5pt)*#3]{#2}.north west) [anchor=north west,indexKeyBox] {{#4}};
}


% --------------------------------- index key-pointer tuples

\tikzset{
	keyPointerValue/.style={
		anchor=north west,
		draw,rectangle,minimum width=\indexKeyBoxWidth,minimum height=\tupleHeight,inner sep=0,outer sep=0,font=\footnotesize
	}
}

\tikzset{ 
	keyPointerPointerBox/.style={
		draw, rectangle, minimum width=\pointerBoxWidth,minimum height=\tupleHeight,outer sep=0,fill=snow
	}
}


% #1     --> id of key-pointer (i.e., the node with the value)
% #2, #3 --> id of indexBlock and ordinal within block
% #4     --> value in the attribute of the tuple
\def\keyPointer#1#2#3#4{
  \node ({#1}) at ([xshift=1pt,yshift=\dimexpr -1pt-(\tupleHeight+0.5pt)*#3]{#2}.north west) [anchor=north west,indexKeyBox] {{#4}};
  \node [right = -0.5pt of {#1}] [keyPointerPointerBox] {};
}


% --------------------------------- blocks of the data file

\tikzset{
	dataBlockBox/.style={
		xshift=-0.1em,yshift=0.1em,anchor=north west,draw,rectangle,minimum width=\dataBlockWidth,minimum height=\dataBlockHeight
	}
}

\tikzset{
	dataBlockPointerBox/.style={
		yshift=0.125pt,
		outer sep=0,anchor=south west, draw, rectangle, 
		minimum width=1em, minimum height=1em,fill=snow
	}
}

% #1     --> block id
% #2, #3 --> block top-left coordinates
\def\dataBlock#1#2#3{
	\node ({#1}) at ({#2},{#3}) [dataBlockBox] {};
	\node at ({#1}.south east) [dataBlockPointerBox] {};
}

% #1     --> block id
% #2     --> id of block ``below'' #1
\def\linkDataBlocks#1#2{
	\draw [*->,>=stealth'] ([xshift=5pt,yshift=7pt]{#1}.south east) to[out=270,in=90] ([xshift=-10pt]#2.north east);
}

% --------------------------------- blocks of the index file


\tikzset{
	indexBlockBox/.style={
		xshift=-0.1em,yshift=0.1em,anchor=north west,draw,rectangle,minimum width=\indexBlockWidth,minimum height=\indexBlockHeight
	}
}

\tikzset{
	indexBlockPointerBox/.style={
		xshift=0.375pt,
		anchor=south east, draw, rectangle, minimum width=1em, minimum height=1em,fill=snow
	}
}

% #1     --> block id
% #2, #3 --> block top-left coordinates
% #4     --> id of the first key-pointer inside the block
% #5     --> value of the first key
\def\indexBlock#1#2#3{
	\node ({#1}) at ({#2},{#3}) [indexBlockBox] {};
	\node at ({#1}.south west) [indexBlockPointerBox] {};
}

% #1     --> block id
% #2     --> id of block ``below'' #1
\def\linkIndexBlocks#1#2{
	\draw [*->,>=stealth'] ([xshift=-5pt,yshift=7pt]{#1}.south west) to[out=270,in=90] ([xshift=10pt]#2.north west);
}

% #1 --> key-pointer id
% #2 --> tuple id
\def\KPtoTuple#1#2{
	\draw [*->,>=stealth'] ([xshift=30pt,yshift=0em]{#1}.west) to[out=0,in=180] (#2.west);
}




%%%%% MACROS FOR SPARSE INDEX WITH POINTERS TO BLOCKS

\tikzset{ 
	sparseKeyPointerPointerBox/.style={
		draw, rectangle, minimum width=1.1em,minimum height=\tupleHeight, fill=snow
	}
}

% draws two rectangles, one with an attribute value, the other with the pointer
%
% #1     --> id of key-pointer (i.e., the node with the value)
% #2, #3 --> coordinages of top-left corner of node with tuple value
% #4     --> value in the attribute of the tuple
\def\sparseKeyPointer#1#2#3#4{
  \node ({#1}) at ({#2},{#3}) [keyPointerValue] {\tiny{#4}} ;
  \node [right = -0.5pt of {#1}] [sparseKeyPointerPointerBox] {};
}

\tikzset{
	sparseIndexBlockBox/.style={
		xshift=-0.1em,yshift=0.1em,anchor=north west,draw,rectangle,minimum width=3.3em,minimum height=4.6em
	}
}

% #1     --> block id
% #2, #3 --> block top-left coordinates
% #4     --> id of the first key-pointer inside the block
% #5     --> value of the first key
\def\sparseIndexBlock#1#2#3#4#5{
	\node ({#1}) at ({#2},{#3}) [sparseIndexBlockBox] {};
	\node at ({#2},{#3}) [indexBlockPointerBox] {};
	\sparseKeyPointer{{#4}}{{#2}}{{#3}}{{#5}};
}

% #1     --> block id
% #2     --> id of block ``above'' this one
% #3, #4 --> coordinates of this block
% #5     --> id of the first tuple inside the block
% #6     --> value of the first tuple
\def\sparseIndexBlockBelow#1#2#3#4#5#6{
	\node ({#1}) at ({#3},{#4}) [sparseIndexBlockBox] {};
	\node at ({#3},{#4}) [indexBlockPointerBox] {};
	\sparseKeyPointer{{#5}}{{#3}}{{#4}}{{#6}};
	\draw [*->,>=stealth'] ([xshift=-5pt,yshift=7pt]{#2}.south west) to[out=270,in=90] ([xshift=10pt]#1.north west);
}

% #1 --> key-pointer id
% #2 --> tuple id
\def\sparseKPtoTuple#1#2{
	\draw [*->,>=stealth'] ([xshift=22.5pt,yshift=0em]{#1}.west) to[out=0,in=180] (#2.west);
}



%================================ illustration of multilevel indexes

\def\denseIndexMultilevel#1#2#3#4{
    \node ({#1}) at ({#2},{#3}) [font=\footnotesize,draw,fill=accent!25,rectangle,minimum width=2.5em,minimum height=1em] {{#4}};
    \node (t) [below = 1em of {#1}] [rectangle,minimum width=2.5em] {};
    \draw [->,>=stealth'] ({#1}) -- (t);
    \draw [->,>=stealth'] ([xshift=-2pt]{#1}.south) -- ([xshift=-6pt]t.north);
    \draw [->,>=stealth'] ([xshift=2pt]{#1}.south) -- ([xshift=6pt]t.north);
    \draw [->,>=stealth'] ([xshift=-6pt]{#1}.south) -- (t.north west);
    \draw [->,>=stealth'] ([xshift=6pt]{#1}.south) -- (t.north east);
}

\def\denseIndexMultilevelAfter#1#2#3{
    \node ({#1}) [right of= {#2}] [font=\footnotesize,draw,fill=accent!25,rectangle,minimum width=2.5em,minimum height=1em] {{#3}};
        \node (t) [below = 1em of {#1}] [rectangle,minimum width=2.5em] {};
    \draw [->,>=stealth'] ({#1}) -- (t);
    \draw [->,>=stealth'] ([xshift=-2pt]{#1}.south) -- ([xshift=-6pt]t.north);
    \draw [->,>=stealth'] ([xshift=2pt]{#1}.south) -- ([xshift=6pt]t.north);
    \draw [->,>=stealth'] ([xshift=-6pt]{#1}.south) -- (t.north west);
    \draw [->,>=stealth'] ([xshift=6pt]{#1}.south) -- (t.north east);
}


\newsavebox\genericDenseIndexWithNblocks
\savebox{\genericDenseIndexWithNblocks}{
\begin{tikzpicture}[
	every node/.append style={node distance=4em},
	every edge/.append style={>=stealth'}]
\node at (-1.5,0) [color=accent,font=\footnotesize] {Index:};

\denseIndexMultilevel{0}{0}{0}{$b_0$};
\denseIndexMultilevelAfter{1}{0}{$b_1$};
\denseIndexMultilevelAfter{2}{1}{$b_2$};
\node (3) [right of=2] {...};
\denseIndexMultilevelAfter{4}{3}{$b_{N-1}$};
\denseIndexMultilevelAfter{5}{4}{$b_{N}$};

\path [->] (0) edge (1) 
   (1) edge (2)
   (2) edge (3)
   (3) edge (4)
   (4) edge (5);
\end{tikzpicture}
}

\newsavebox\genericMultilevelIndexSparseOnDense
\savebox{\genericMultilevelIndexSparseOnDense}{
\begin{tikzpicture}[
	every node/.append style={node distance=4em},
	every edge/.append style={>=stealth'}]
\tikzset{
    block/.style={font=\footnotesize,draw,fill=snow,rectangle,minimum width=2.5em,minimum height=1em}
    }
\tikzset{
    sparseBlock/.style={font=\footnotesize,draw,fill=snow,rectangle,minimum width=2.5em,minimum height=1em}
    }

\node at (-1.5,0) [color=accent,font=\footnotesize] {\begin{minipage}{1cm}\baselineskip=0.75\baselineskip \centering
Dense Index: \end{minipage}};

\denseIndexMultilevel{0}{0}{0}{$b_0$};
\denseIndexMultilevelAfter{1}{0}{$b_1$};
\denseIndexMultilevelAfter{2}{1}{$b_2$};
\node (3) [right of=2] {...};
\denseIndexMultilevelAfter{4}{3}{$b_{N-1}$};
\denseIndexMultilevelAfter{5}{4}{$b_{N}$};

\path [->] (0) edge (1) 
   (1) edge (2)
   (2) edge (3)
   (3) edge (4)
   (4) edge (5);

\node at (-1.5,1.5) [color=accent,font=\footnotesize] {\begin{minipage}{1cm}\baselineskip=0.75\baselineskip \centering
Sparse Index: \end{minipage}};

\node [sparseBlock] (6) [xshift=1em,above of=1] {$\mathit{b}_0$};
\node (7) [xshift=1em,right of=6] {...};
\node [sparseBlock] (8) [xshift=1em,right of=7] {$\mathit{b}_M$};

\path [->] (6) edge (7)
    (7) edge (8);

\draw [->,>=stealth'] ([xshift=-9pt]6.south) to[out=270,in=90] ([xshift=2pt]0.north west);
\draw [->,>=stealth'] ([xshift=-5pt]6.south) to[out=270,in=90] ([xshift=2pt]1.north west);
\draw [->,>=stealth'] ([xshift=-1pt]6.south) to[out=270,in=90] ([xshift=2pt]2.north west);
\draw [->,>=stealth'] ([xshift=5pt]8.south) to[out=270,in=90] ([xshift=2pt]4.north west);
\draw [->,>=stealth'] ([xshift=9pt]8.south) to[out=270,in=90] ([xshift=2pt]5.north west);

\node (inv1) [xshift=2em,yshift=-1em,above right of=1,rectangle]  {...};
\draw [->,>=stealth'] ([xshift=2pt]6.south) -- (inv1.north west);
\draw [->,>=stealth'] ([xshift=9pt]6.south) -- (inv1.north east);

\node (inv2) [xshift=-2em,yshift=-1em,above right of=3,rectangle]  {...};
\draw [->,>=stealth'] ([xshift=-4pt]8.south) -- (inv2.north east);
\draw [->,>=stealth'] ([xshift=-8pt]8.south) -- (inv2.north west);
\end{tikzpicture}
}


%
% illustration scanning cost of range selection on dense primary index
%

\newsavebox{\rangeSelectionQueryPrimaryIndex}
\savebox{\rangeSelectionQueryPrimaryIndex}{
\begin{tikzpicture}
%index blocks
\node (0) at (0,0) [draw,rectangle,fill=gray!35,minimum width=1.75em,minimum height=1em] {};
\node (1) at (1,0) [rectangle,minimum width=1em,minimum height=1em] {...};
\foreach \i in {2,...,4}{
  \node (\i) at (\i,0) [draw,rectangle,fill=gray!35,minimum width=1.75em,minimum height=1em] {};  
}
\node (5) at (5,0) [rectangle,minimum width=1em,minimum height=1em] {...};    
\node (6) at (6,0) [draw,rectangle,minimum width=1.75em,minimum height=1em] {};
%chain of index blocks
\foreach \i in {1,...,6}{
    {\pgfmathsetmacro{\j}{\i - 1}
    \draw [->,>=stealth'] ({\j}) -- (\i);}
}

%data blocks
\node (7) at (0,-1) [draw,rectangle,minimum width=1.75em,minimum height=1em] {};
\node (8) at (1,-1) [rectangle,minimum width=1em,minimum height=1em] {...};
\foreach \i in {9,...,10}{
  \pgfmathsetmacro{\j}{\i - 7}
  \node (\i) at (\j,-1) [draw,rectangle,minimum width=1.75em,minimum height=1em] {};  
}
\node (11) at (4,-1) [draw,rectangle,fill=gray!35,minimum width=1.75em,minimum height=1em] {};
\node (12) at (5,-1) [rectangle,minimum width=1em,minimum height=1em] {...};    
\node (13) at (6,-1) [draw,rectangle,fill=gray!35,minimum width=1.75em,minimum height=1em] {};
%chain of data blocks
\foreach \i in {8,...,13}{
    {\pgfmathsetmacro{\j}{\i - 1}
    \draw [->,>=stealth'] ({\j}) -- (\i);}
}

%index and tuple keys
\node (k) at (3.85,0) [inner sep=0,outer sep=0,draw,fill=accent,rectangle,minimum width=0.25em,minimum height=1em] {};
\node (t) at (4.15,-1) [inner sep=0,outer sep=0,draw,fill=accent,rectangle,minimum width=0.25em,minimum height=1em] {};

%labels
\node (i) [outer sep=0pt,above left of=k,draw,color=accent] {\scriptsize $\min (x') > x$};
\node at (6.5,-1) [anchor= west]{\scriptsize Data: $N$ blocks}; 
\node at (6.5,0) [anchor= west]{\scriptsize Index: $M$ blocks}; 

\draw [color=accent,->] (i.south) to[out=270,in=90] (k.north);
\draw [->,>=stealth'] (k.south) to[out=270,in=90] (t.north);
\end{tikzpicture}
}


%
%
% --------------------------------------------------------------
%
%


\newsavebox{\rangeSelectionQueryStackedIndex}
\savebox{\rangeSelectionQueryStackedIndex}{
\begin{tikzpicture}
%index blocks
\node (0) at (0,0) [draw,rectangle,,minimum width=1.75em,minimum height=1em] {};
\node (1) at (1,0) [rectangle,minimum width=1em,minimum height=1em] {...};
\foreach \i in {2,...,3}{
  \node (\i) at (\i,0) [draw,rectangle,minimum width=1.75em,minimum height=1em] {};  
}
\node (4) at (4,0) [draw,rectangle,fill=gray!35,minimum width=1.75em,minimum height=1em] {};
\node (5) at (5,0) [rectangle,minimum width=1em,minimum height=1em] {...};    
\node (6) at (6,0) [draw,rectangle,minimum width=1.75em,minimum height=1em] {};
%chain of index blocks
\foreach \i in {1,...,6}{
    {\pgfmathsetmacro{\j}{\i - 1}
    \draw [->,>=stealth'] ({\j}) -- (\i);}
}

%sparse index on top of dense index
\node (14) at (1,1) [draw,rectangle,fill=gray!35,minimum width=1.75em,minimum height=1em] {};
\node (15) at (2,1)  {...};
\node (16) at (3,1) [draw,rectangle,fill=gray!35,minimum width=1.75em,minimum height=1em] {};
\node (17) at (4,1)  {...};
\node (18) at (5,1) [draw,rectangle,minimum width=1.75em,minimum height=1em] {};
\foreach \i in {15,...,18}{
    {\pgfmathsetmacro{\j}{\i - 1}
    \draw [->,>=stealth'] ({\j}) -- (\i);}
}

%data blocks
\node (7) at (0,-1) [draw,rectangle,minimum width=1.75em,minimum height=1em] {};
\node (8) at (1,-1) [rectangle,minimum width=1em,minimum height=1em] {...};
\foreach \i in {9,...,10}{
  \pgfmathsetmacro{\j}{\i - 7}
  \node (\i) at (\j,-1) [draw,rectangle,minimum width=1.75em,minimum height=1em] {};  
}
\node (11) at (4,-1) [draw,rectangle,fill=gray!35,minimum width=1.75em,minimum height=1em] {};
\node (12) at (5,-1) [rectangle,minimum width=1em,minimum height=1em] {...};    
\node (13) at (6,-1) [draw,rectangle,fill=gray!35,minimum width=1.75em,minimum height=1em] {};
%chain of data blocks
\foreach \i in {8,...,13}{
    {\pgfmathsetmacro{\j}{\i - 1}
    \draw [->,>=stealth'] ({\j}) -- (\i);}
}

%index and tuple keys
\node (k) at (3.9,0) [inner sep=0,outer sep=0,draw,fill=accent,rectangle,minimum width=0.25em,minimum height=1em] {};
\node (s) at (2.9,1) [inner sep=0,outer sep=0,draw,fill=accent,rectangle,minimum width=0.25em,minimum height=1em] {};
\node (t) at (4.15,-1) [inner sep=0,outer sep=0,draw,fill=accent,rectangle,minimum width=0.25em,minimum height=1em] {};

%labels
\node (i) [outer sep=0pt,above left=1.5cm and -0.5cm of s,draw,color=accent] {\scriptsize $\max (x') \leq x$};
\node (j) [outer sep=0pt,above right=1.5cm and -0.5cm of k,draw,color=accent] {\scriptsize $\min (x') > x$};
\node at (6.5,-1) [anchor= west]{\scriptsize Data: $N$ blocks}; 
\node at (6.5,0) [anchor= west]{\scriptsize Dense Index: $M$ blocks}; 
\node at (6.5,1) [anchor= west]{\scriptsize Sparse Index: $K$ blocks}; 

\draw [color=accent,->] (j.south) to[out=270,in=90] (k.north);
\draw [color=accent,->] (i.south) to[out=270,in=90] (s.north);

\draw [->,>=stealth'] (s.south) to[out=270,in=115] (4.north west);
\draw [->,>=stealth'] (k.south) to[out=270,in=90] (t.north);
\end{tikzpicture}
}


%
%
% --------------------------------------------------------------
%
%

\newsavebox{\rangeSelectionQuerySecondaryIndex}
\savebox{\rangeSelectionQuerySecondaryIndex}{
\begin{tikzpicture}
%index blocks
\node (0) at (0,0) [draw,rectangle,fill=gray!35,minimum width=1.75em,minimum height=1em] {};
\node (1) at (1,0) [rectangle,minimum width=1em,minimum height=1em] {...};
\foreach \i in {2,...,4}{
  \node (\i) at (\i,0) [draw,rectangle,fill=gray!35,minimum width=1.75em,minimum height=1em] {};  
}
\node (5) at (5,0) [rectangle,minimum width=1em,minimum height=1em] {...};    
\node (6) at (6,0) [draw,rectangle,fill=gray!35,minimum width=1.75em,minimum height=1em] {};
%chain of index blocks
\foreach \i in {1,...,6}{
    {\pgfmathsetmacro{\j}{\i - 1}
    \draw [->,>=stealth'] ({\j}) -- (\i);}
}

%data blocks
\node (7) at (0,-1) [draw,rectangle,fill=gray!35,minimum width=1.75em,minimum height=1em] {};
\node (8) at (1,-1) [rectangle,minimum width=1em,minimum height=1em] {...};
\foreach \i in {9,...,10}{
  \pgfmathsetmacro{\j}{\i - 7}
  \node (\i) at (\j,-1) [draw,rectangle,fill=gray!35,minimum width=1.75em,minimum height=1em] {};  
}
\node (11) at (4,-1) [draw,rectangle,fill=gray!35,minimum width=1.75em,minimum height=1em] {};
\node (12) at (5,-1) [rectangle,minimum width=1em,minimum height=1em] {...};    
\node (13) at (6,-1) [draw,rectangle,minimum width=1.75em,minimum height=1em] {};
%chain of data blocks
\foreach \i in {8,...,13}{
    {\pgfmathsetmacro{\j}{\i - 1}
    \draw [->,>=stealth'] ({\j}) -- (\i);}
}

%index and tuple keys
\node (k) at (3.85,0) [inner sep=0,outer sep=0,draw,fill=accent,rectangle,minimum width=0.25em,minimum height=1em] {};
\node (t) at (4.15,-1) [inner sep=0,outer sep=0,draw,fill=accent,rectangle,minimum width=0.25em,minimum height=1em] {};

%labels
\node (i) [outer sep=0pt,above left of=k,draw,color=accent] {\scriptsize $\min (x') > x$};
\node at (6.5,-1) [anchor= west]{\scriptsize Data: $N$ blocks}; 
\node at (6.5,0) [anchor= west]{\scriptsize Index: $M$ blocks}; 

\draw [color=accent,->] (i.south) to[out=270,in=90] (k.north);
\draw [->,>=stealth'] (k.south) to[out=270,in=90] (t.north);

% out of order keys/tuples
\node (k2) at (4,0) [inner sep=0,outer sep=0,draw,fill=accent,rectangle,minimum width=0.25em,minimum height=1em] {};
\node (t2) at (0.15,-1) [inner sep=0,outer sep=0,draw,fill=accent,rectangle,minimum width=0.25em,minimum height=1em] {};
\draw [->,>=stealth'] (k2.south) to[out=235,in=45] (t2.north);

\node (k3) at (4.15,0) [inner sep=0,outer sep=0,draw,fill=accent,rectangle,minimum width=0.25em,minimum height=1em] {};
\node (t3) at (2.15,-1) [inner sep=0,outer sep=0,draw,fill=accent,rectangle,minimum width=0.25em,minimum height=1em] {};
\draw [->,>=stealth'] (k3.south) to[out=235,in=45] (t3.north);

\node (k4) at (6.15,0) [inner sep=0,outer sep=0,draw,fill=accent,rectangle,minimum width=0.25em,minimum height=1em] {};
\node (t4) at (3.25,-1) [inner sep=0,outer sep=0,draw,fill=accent,rectangle,minimum width=0.25em,minimum height=1em] {};
\draw [->,>=stealth'] (k4.south) to[out=235,in=45] (t4.north);

\end{tikzpicture}
}



%
%
% --------------------------------------------------------------
%
%

\newsavebox{\motivatingBucketFiles}
\savebox{\motivatingBucketFiles}{
\begin{tikzpicture}
%index blocks
\node (0) at (0,0) [draw,rectangle,fill=accent!15,minimum width=1.75em,minimum height=1em] {};
\node (1) at (1,0) [rectangle,minimum width=1em,minimum height=1em] {...};
\foreach \i in {2,...,4}{
  \node (\i) at (\i,0) [draw,rectangle,fill=accent!15,minimum width=1.75em,minimum height=1em] {};  
}
\node (5) at (5,0) [rectangle,minimum width=1em,minimum height=1em] {...};    
\node (6) at (6,0) [draw,rectangle,fill=accent!15,minimum width=1.75em,minimum height=1em] {};
%chain of index blocks
\foreach \i in {1,...,6}{
    {\pgfmathsetmacro{\j}{\i - 1}
    \draw [->,>=stealth'] ({\j}) -- (\i);}
}

%data blocks
\node (7) at (0,-1.5) [draw,rectangle,minimum width=1.75em,minimum height=1em] {};
\node (8) at (1,-1.5) [rectangle,minimum width=1em,minimum height=1em] {...};
\foreach \i in {9,...,10}{
  \pgfmathsetmacro{\j}{\i - 7}
  \node (\i) at (\j,-1.5) [draw,rectangle,minimum width=1.75em,minimum height=1em] {};  
}
\node (11) at (4,-1.5) [draw,rectangle,minimum width=1.75em,minimum height=1em] {};
\node (12) at (5,-1.5) [rectangle,minimum width=1em,minimum height=1em] {...};    
\node (13) at (6,-1.5) [draw,rectangle,minimum width=1.75em,minimum height=1em] {};
%chain of data blocks
\foreach \i in {8,...,13}{
    {\pgfmathsetmacro{\j}{\i - 1}
    \draw [->,>=stealth'] ({\j}) -- (\i);}
}

%index and tuple keys
\node (k) at (3.85,0) [inner sep=0,outer sep=0,draw,fill=accent,rectangle,minimum width=0.25em,minimum height=1em] {};
\node (t) at (4.15,-1.5) [inner sep=0,outer sep=0,draw,fill=accent,rectangle,minimum width=0.25em,minimum height=1em] {};

%labels
\node at (6.5,-1.5) [anchor= west]{\scriptsize Data: $N$ blocks}; 
\node at (6.5,0) [anchor= west]{\scriptsize Index: $M$ blocks}; 

\draw [->,>=stealth'] (k.south) to[out=270,in=90] (t.north);

% out of order keys/tuples
\node (k2) at (4,0) [inner sep=0,outer sep=0,draw,fill=accent,rectangle,minimum width=0.25em,minimum height=1em] {};
\node (t2) at (0.15,-1.5) [inner sep=0,outer sep=0,draw,fill=accent,rectangle,minimum width=0.25em,minimum height=1em] {};
\draw [->,>=stealth'] (k2.south) to[out=235,in=45] (t2.north);

\node (k3) at (4.15,0) [inner sep=0,outer sep=0,draw,fill=accent,rectangle,minimum width=0.25em,minimum height=1em] {};
\node (t3) at (2.15,-1.5) [inner sep=0,outer sep=0,draw,fill=accent,rectangle,minimum width=0.25em,minimum height=1em] {};
\draw [->,>=stealth'] (k3.south) to[out=235,in=45] (t3.north);

\node (k4) at (6.15,0) [inner sep=0,outer sep=0,draw,fill=accent,rectangle,minimum width=0.25em,minimum height=1em] {};
\node (t4) at (3.25,-1.5) [inner sep=0,outer sep=0,draw,fill=accent,rectangle,minimum width=0.25em,minimum height=1em] {};
\draw [->,>=stealth'] (k4.south) to[out=235,in=45] (t4.north);

\end{tikzpicture}
}
